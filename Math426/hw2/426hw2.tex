\documentclass[10.5pt]{article}
\usepackage{amsmath, amsfonts, amssymb,amsthm}
\usepackage{centernot}
\usepackage[includeheadfoot,margin=0.5in]{geometry} % For page dimensions
\usepackage{fancyhdr}
\usepackage{enumerate} % For custom lists

\fancyhf{}
\lhead{Math 426hw1}
\rhead{Tighe McAsey - 37499480}
\pagestyle{fancy}

% Page dimensions
\geometry{a4paper}

\theoremstyle{definition}
\newtheorem{pb}{}
\usepackage{color}

% Commands:

\newcommand{\set}[1]{\{#1\}}
\newcommand{\abs}[1]{\lvert#1\rvert}
\newcommand{\norm}[1]{\lvert\lvert#1\rvert\rvert}
\newcommand{\tand}{\text{ and }}
\newcommand{\tor}{\text{ or }}
\newcommand{\ism}{\simeq}

\begin{document}
    \begin{pb}
        
    \end{pb}
    \begin{pb}
        
    \end{pb}

    \textbf{Lemma.} I will use the following lemma to streamline my proofs for problems 3 and 4. \newline
    If \(\psi: X \to Y\) is a homeomorphism, and \(\sim\) is an equivalence relation on \(X\), and \(\approx\) a equivalence relation on \(Y\), such that
    \(\psi(a) \approx \psi(b) \iff a \sim b\), then \(X/\sim \ism Y/\approx\) \newline
    \textbf{proof.} Define \(\overline{\psi}: X/\sim \to Y/\approx\), by \(\overline{\psi}: \overline{x} \mapsto \overline{\psi(x)}\), this is surjective since \(\psi\) is surjective and \(\overline{\psi}\) is
    well defined/injective by definition of \(\approx\). We can define \(\overline{\psi^{-1}}: Y/\approx \to X/\sim\), in the same way. This is the inverse of \(\overline{\psi}\), since
    \(\overline{\psi} \tand \overline{\psi^{-1}}\) are just restrictions to equivalence classes of \(\psi \tand \psi^{-1}\). To show \(\overline{\psi}\) is continuous, note that
    \(\overline{\psi} = \pi_\approx \psi\). Let \(U\) be open in \(Y/\approx\), then the preimage of \(U\) under \(\pi_\approx \) is open by definition, so continuity follows from continuity of \(\psi\).
    The proof for continuity of \(\overline{\psi^{-1}}\) is the same.

    \begin{pb}
        
    \end{pb}
    \begin{pb}
        Note that the triangle is homeomorphic to the disc. We can insribe the triangle in a circle with radius \(R\). Then for each point \(p\), let \(q\) be the intersection
        of the ray through \(p \) and the origin with the boundary of the triangle. For each of these points we can map \(p \mapsto \frac{Rp }{\abs{q }}\) this is a homeomorphism
        since \(q\) varies smoothly with \(p\) and we have inverse \(p \mapsto \frac{\abs{q}p}{R}\), where \(q\) comes from inscribing the triangle in the circle, which is also continuous.
        It follows that the equivalence relation induced on \(D^2\) is \(e^{ix} \sim e^{ix}e^{\frac{2\pi}{3}} \sim e^{-ix}\), which can be seen by the picture and lemma. So that
        the dunce cap can be written as \(D^2/\sim\).

        \textcolor{red}{Include Images HERE}

        Now consider the maps \(\mathbf{1}_{S^1}\) and
        \begin{align*}
            &f:S^1 \to S^1 \\
            &e^{ix} \mapsto \begin{cases}
                e^{3ix} & 0 \leq x < \frac{4\pi}{3} \\
                e^{-3ix} & \frac{4\pi}{3} \leq x < 2\pi
            \end{cases}
        \end{align*}
        Take the mapping cone 
        \begin{align*}
            C_f = S^1 \times I / (x,0) \sim f(x), (x,1) \sim (y,1)
        \end{align*}
        For each \(x\), we have \(f^{-1}(x) = \set{e^{ix/3},e^{i(x + 2\pi)/3},e^{-ix/3}}\). We can then take the map \(C_{\mathbf{1}_{S^1}} \to D^2\), where \((x,t) \mapsto (x,1-t)\), this is a homeomorphism between the
        cone and disc, with the quotients in \(D^2/\sim\) being the image of quotients of \(C_f\) under this map. Hence by the lemma \(C_f \ism D^2/\sim\) the dunce cap.

        We have that \(C_{\mathbf{1}_{S^1}}\) is contractible, using the homotopy \(H((x,t),s) = (x,t(1-s))\), so it will suffice to show that \(C_{f} \ism C_{\mathbf{1}_{S^1}}\), and we 
        have proven in class that homotopic maps have homotopic cones. I will show \(f \sim \rho \sim \mathbf{1}_{S^1}\), where \[\rho: e^{ix} \mapsto 
        \begin{cases} 
        e^{3ix} & 0 < x < 2\pi/3 \\ 
        1 & 2\pi/3 \leq x < 2\pi
        \end{cases}\]
        I will provide \(H_1\) for the first equivalence \(f \sim \rho\) and \(H_2\) for the second \(\rho \sim \mathbf{1}_{S^1}\).
        \begin{align*}
            &H_1(x,t):
            \begin{cases}
                x \mapsto f(x) & x < \frac{2}{3} - \frac{1}{3}t \tor x > \frac{2}{3} + \frac{1}{3}t \\ 
                x \mapsto f(\frac{2}{3} - \frac{1}{3}t) & \frac{2}{3} - \frac{1}{3}t \leq x \leq \frac{2}{3} + \frac{1}{3}t
            \end{cases} \\
            &H_2(x,t):
            \begin{cases}
                x \mapsto f(\frac{x}{1 + 2t})
            \end{cases}
        \end{align*}

    \end{pb}
\end{document}