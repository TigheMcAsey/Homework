\documentclass[10.5pt]{article}
\usepackage{graphicx}
\usepackage{amsmath, amsfonts, amssymb,amsthm}
\usepackage{centernot}
\usepackage[includeheadfoot,margin=0.5in]{geometry} % For page dimensions
\usepackage{fancyhdr}
\usepackage{enumerate} % For custom lists
\usepackage{tikz-cd}

\fancyhf{}
\lhead{Math 426hw3}
\rhead{Tighe McAsey - 37499480}
\pagestyle{fancy}

% Page dimensions
\geometry{a4paper}

\theoremstyle{definition}
\newtheorem{pb}{}
\usepackage{color}

% Commands:

\newcommand{\set}[1]{\{#1\}}
\newcommand{\abs}[1]{\lvert#1\rvert}
\newcommand{\norm}[1]{\lvert\lvert#1\rvert\rvert}
\newcommand{\gen}[1]{\langle #1 \rangle}
\newcommand{\tand}{\text{ and }}
\newcommand{\tor}{\text{ or }}
\newcommand{\ism}{\simeq}

\begin{document}
    \begin{pb}
        We first check that \(f_*\) is well defined, by definition of being a strong deformation retract we have \(f_*: A \to A\) as the identity map on objects, furthermore if \(\gamma\) is a path, then
        \(f(\gamma,1)\subset Y\), so that we only need check that \(f_*\) is well defined on equivalence classes of paths to see that \(f_*\) is a map from \(\Pi(X,A)\) to \(\Pi(Y,A)\).
        Let \(\gamma\) and \(\gamma'\) be homotopic paths (with respect to \(A\)), then since \(f_*\) is a strong deformation retract onto \(Y\), we have
        \(f_*(\gamma) \sim_A \gamma \tand f_*(\gamma') \sim_A \gamma'\) are homotopic. It follows that by transitivity
        \begin{align*}
            f_*(\gamma) \sim_A \gamma \sim_A \gamma' \sim_A f_*(\gamma')
        \end{align*}
        are homotopic with respect to \(A\).

        To show \(f_*\) is an isomorphism of groupoids it will suffice to provide an inverse. Define \(g\) as the embedding of \(Y\) into \(X\), it is clear that \(g_*\) is identity on objects and
        well defined on paths. \(f_*g_* = 1_{\Pi(Y,A)}\) since both are identity on \(A\), and if \(\gamma\) is a path in \(Y\), then both \(g \tand f\) fix \(\gamma\), hence
        \(f_*g_*([\gamma]) = [\gamma]\). Now considering \(g_*f_*\), we once again have both being identity on \(A\). Now if \(\gamma\) is a path in \(X\), \(f\) being a strong
        deformation retract implies that \(\gamma\) is homotopic in \(X\) to \(f(\gamma,1)\) with respect to \(A\), but then \(g(f(\gamma,1),1) = f(\gamma,1)\), so that
        \(g(f(\gamma,1),1) \sim_A \gamma\). This implies that \(g_*f_*([\gamma]) = [\gamma]\).
    \end{pb}
    \begin{pb}
        \textbf{(a)} There is an obvious strong deformation retract from \(U \cap V\) to two disconnected circles, one with base point \(p\), and one with basepoint
        \(q\), since there are no paths between the two points we know what this groupoid looks like since we know the fundamental group of the circle.
        Hence by problem 1, we have \(\Pi(U \cap V, \set{p,q}) = \cdots\)

        \textbf{(b)} Here is an explicit strong deformation retract of \(U\) (you can ignore this and look at the picture if you like).
        We can identify \(U\) as having coordinates \((t,x)\), where \(t\) spans the circular portion of \(U\),
        and \(x\) the horizontal. Now assume that \(p\) is at \((0,0)\) and \(q\) at \((0,\epsilon)\) where \(x \in (-1,1), t \in [0,1]\), so that \(\epsilon \in (0,1)\), in this case
        we can see that \(t = 1\) at the identified boundary. Then we cant take the strong deformation retract
        \begin{align*}
            &H: U \times I \to U \\
            &((t,x),s) \mapsto \begin{cases}
                (t,xe^{t(1-\frac{1}{1-s})}) & s \neq 1\\
                (t,0) & s = 1 \tand t \neq 0 \\
                (t,x) & s = 1 \tand t = 0
            \end{cases}
        \end{align*}
        This is clearly continuous, away from \(s = 1\), but it is continuous there as well since \(\lim_{s\to1}e^{1-\frac{1}{1-s}} = 0\). This gives us a strong deformation 
        retract from \(U\) to a circle union the line through \(p,q\) where clearly we can retract the portion of the line to the left of \(p\) onto \(p\). Alternatively, here is a picture of the
        retract:
            
        \textcolor{red}{Picture of SDR Here}

        From the lemma, this retracted space has the same fundamental groupoid as \(U\). Hence we have that
        \(\Pi(U,\set{p,q}) = \cdots\)

        \textbf{(c)} We use the same method as in part (b) to conclude that the fundamental groupoid \(\Pi(V, \set{p,q})\).
        \(\Pi(V, \set{p,q}) = \cdots\)

        \textbf{(d)} We first notice that in \(U\) we have \([b] = [d^{-1}ad]\), and in \(V\) we have \([b] = [cac^{-1}]\). I claim that the category \(X\), with
        \(\text{Ob}(X) = \set{p,q}\) and
        \[\text{Mor}_X(p,p) = \gen{a,dc \vert a = (dc)^{-1}a(dc)} \; \text{Mor}_X(q,q) = \gen{cad, cd \vert cad = (cd)^{-1}} \; \text{Mor}_X(p,q) = \gen{d,c^{-1}}\]
    \end{pb}
    \begin{pb}
        \textbf{(a)}
        Consider \(F = \gen{a,b}\) to be the free group on 2 generators. Then we can take the group homomorphism defined on generators, 
        \(\varphi: F \to G, \;a \mapsto xy, b \mapsto yxy\). To check that this is onto, we need only check \(x,y \in \varphi(F)\), but this is straightforward, since
        \begin{align*}
            &\varphi(ba^{-1}) = \varphi(b)\varphi(a)^{-1} = yxyy^{-1}x^{-1} = y &\varphi(a^2b^{-1}) = \varphi(a)\varphi(ba^{-1})^{-1} = xyy^{-1} = x
        \end{align*}
        By definition of \(G\), we have \(\ker \varphi = \set{\alpha \in F \vert \varphi(\alpha) \in \gen{xyxy^{-1}x^{-1}y^{-1}}}\), where
        \(xyxy^{-1}x^{-1}y^{-1} = \varphi(a^3b^{-2})\), so that \(\varphi(\alpha) \in \gen{\varphi(a^3b^{-2})}\) exactly when \(\alpha \in \gen{a^3b^{-2}}\).
        Hence we have
        \(\ker \varphi = \gen{a^3b^{-2}}\), so that by the first isomorphism theorem \[H \ism F/\gen{a^3b^{-2}} = F/\ker \varphi \ism G\]

        \textbf{(b)} We have the relation \(xy^2x^{-1} = y^3\), then writing conjugation by \(x\) as \(\phi\), we have in general \(\phi(y^{2n}) = \phi(y^2)^n = y^{3n}\).
        Applying two conjugations it is easy to see that \(x^2y^4x^{-2} = xy^6x^{-1} = y^9\). A little harder is
        \begin{align*}
            x^3y^4x^{-3} = (x^3y)y^4(x^3y)^{-1} = (yx^2)y^4(yx^2)^{-1} = y(x^2y^4x^{-2})y^{-1} = y(y^9)y^{-1} = y^9 = x^2y^4x^{-2}
        \end{align*}
        This implies that \(y^6 = xy^4x^{-1} = y^4\) and hence \(y^2 = 1\). Our original relations then give us \(x = yx\) implying \(y=1\) which implies that \(x^2 = x^3\), so that \(x = 1\) as well.
    \end{pb}
    \begin{pb}
        
    \end{pb}
\end{document}