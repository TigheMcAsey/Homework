\documentclass[11pt]{article}
\usepackage{amsmath, amsfonts, amssymb,amsthm}
\usepackage{centernot}
\usepackage[includeheadfoot,margin=0.5in]{geometry} % For page dimensions
\usepackage{fancyhdr}
\usepackage{enumerate} % For custom lists

\fancyhf{}
\lhead{Math 426hw1}
\rhead{Tighe McAsey - 37499480}
\pagestyle{fancy}

% Page dimensions
\geometry{a4paper}

\theoremstyle{definition}
\newtheorem{pb}{}

% Commands:

\newcommand{\set}[1]{\{#1\}}
\newcommand{\abs}[1]{\lvert#1\rvert}
\newcommand{\norm}[1]{\lvert\lvert#1\rvert\rvert}
\newcommand{\tand}[1]{\text{ and }}
\newcommand{\tor}[1]{\text{ or }}

\begin{document}
    \begin{pb}
        \textbf{(a)} Let \(x \in \set{a\in X \vert \exists U \text{ open, such that } a \in U \subset A}\), then \(U \cup A^\circ\) is an open subset of \(A\) containing \(A^\circ\),
        hence by maximality \(x \in U \cup A^\circ = A^\circ\). If \(a \in A^o\), then \(a\) is in a open subset contained in \(A\) proving the other set inclusion.

        let \(x \in \set{x \in X \vert \forall U \text{ open with } x \in U \tand UU \cap A \neq \emptyset}^c\), then there exists some open \(U \subset A^c\) containing \(x\),
        so that \(A \subset U^c\) is closed this implies \(\overline{A} \subset U^c\) and hence \(x \not \in \overline{A}\). Conversely, if \(x \in \overline{A}^c\), then
        \(\overline{A}^c\) is an open set disjoint from \(A\) containing \(x\), so that \(x \in \set{x \in X \vert \forall U \text{ open with } x \in U \tand UU \cap A \neq \emptyset}^c\).

        \textbf{(b)} \(U^\circ\) is open by definition, so \(U^\circ = U\) implies \(U\) open. If \(U\) is open, then \(U\) is an open set contained in \(U\), so that \(U \subset U^\circ\) 
        and hence \(U = U^\circ\).

        \(\overline{A}\) is closed, hence \(A = \overline{A}\) implies \(A\) is closed. Now suppose that \(A\) is closed, then \(A\) is a closed set containing \(A\), hence
        \(A \supset \overline{A}\), which implies \(A = \overline{A}\).

        \textbf{(c)} The compliment of \(A^\circ\) is closed, and \(A^\circ \subset A\) implies that \(\left(A^\circ\right)^c \supset A^c\), implying that
        \(\overline{A^c} \supset \left(A^\circ\right)^c\). Conversely, if \(x \in \overline{A^c}\), then by part (a), any open set containing \(x\) has non-empty
        intersection with \(A^c\), hence there does not exist an open set \(U\) containing \(x\), such that \(U \subset A\), applying (a) again ,this means that
        \(x \not \in A^\circ\)

        \(\overline{A}^c\) is an open set contained in \(A^c\), hence \(\overline{A}^c \subset (A^c)^\circ\). Conversely, if \(x \in \overline{A}\), then from (a), any open set containing
        \(x\) has non-trivial intersection with \(A\), hence applying part (a) again we get that \(x \not \in (A^c)^\circ\), hence \(\overline{A} \subset \left((A^c)^\circ\right)^c\),
        contraposing this gives the desired equality.
    \end{pb}
    \begin{pb}
        Consider the collection \(\mathcal{I}\) of closed sets in \(X\), which are not finite unions of irreducibles. Every descending chain being eventually constant is equivalent to every descending chain having a lower bound
        (i.e. If \(\cap_i F_i = F_j\), then \(F_j\) is a lower bound on the chain).
        Thus we can apply Zorn's lemma which furnishes a minimal element \(Z\) in \(\mathcal{I}\), if \(Z\) were not irreducible, then it would need to be a union of closed subsets \(Z_1\cup Z_2\),
        since \(Z\) is not a finite union of irreducibles, the same must apply to one of \(Z_1 \tor ZZ_2\), but this contradicts the minimality of \(Z \in \mathcal{I}\). It follows that
        \(\mathcal{I} = \emptyset\), so that \(X\) is a finite union of irreducible elements.

        let \(\set{Y_i}_{i = 1}^m \neq \set{Z_i}_{i = 1}^n\) be two collections
        of irreducible sets, such that no set is contained in the union of the rest of the collection, and
        \begin{align*}
            \bigcup_i Y_i = X = \bigcup_i Z_i
        \end{align*}
        Then there must exist some \(Y_i, Z_j\), such that \(Y_i \cap Z_j \neq \emptyset \tand Y_i \neq Z_j\)
        (explicitly choose some \(Y_i \not \in \set{Z_j}_j\), but \(\emptyset \neq Y_i = Y_i \cap \cup_j Z_j = \cup_j Y_i \cap Z_j\)
         cannot all be empty). We may assume WLOG \(Y_i \not \subset Z_j\), but this contradicts the Zarisky condition, since 
        \(Y_i = (Y_i \cap Z_j) \cup (Y_i \cap \cup_{i \neq j}Z_i)\) is a union of closed proper subsets of \(Y_i\).
    \end{pb}
\end{document}