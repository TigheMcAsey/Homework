\documentclass[10.5pt]{article}
\usepackage{amsmath, amsfonts, amssymb,amsthm}
\usepackage[includeheadfoot]{geometry} % For page dimensions
\usepackage{fancyhdr}
\usepackage{enumerate} % For custom lists

\fancyhf{}
\lhead{Math 421hw1}
\pagestyle{fancy}

% Page dimensions
\geometry{a4paper, margin=1in}

\theoremstyle{definition}
\newtheorem{pb}{}

% Commands:

\newcommand{\set}[1]{\{#1\}}
\newcommand{\abs}[1]{\left\vert#1\right\vert}
\newcommand{\norm}[1]{\lvert\lvert#1\rvert\rvert}
\newcommand{\tand}{\text{ and }}
\newcommand{\tor}{\text{ or }}
\newcommand{\floor}[1]{\left\lfloor #1 \right\rfloor}
\newcommand{\ceil}[1]{\left\lceil #1 \right\rceil}
\newcommand{\re}{\text{Re}}
\newcommand{\im}{\text{Im}}

\begin{document}
    \begin{pb}
        \textbf{a} Let \(x \in A_1^\circ \cup A_2^\circ\), then without loss of generality it will suffice to show \(A_1^\circ \subset (A_1 \cup A_2)^\circ\). We have that \(A_1^\circ \subset A_1 \subset A_1 \cup A_2\), which is open, then
        \[(A_1 \cup A_2)^\circ \subset (A_1 \cup A_2)^\circ \cup A_1^\circ \subset A_1 \cup A_2\]
        is open, and since  \((A_1 \cup A_2)^\circ\) is the largest open set contained in \(A_1 \cup A_2\), it must be the case that
        \[(A_1 \cup A_2)^\circ = (A_1 \cup A_2)^\circ \cup A_1^\circ\]
        so that \(A_1^\circ \subset (A_1 \cup A_2)^\circ\).
        
        To show that equality need not hold, consider the usual (metric) topology on \(\mathbb{R}\), with \(A_1 = [-1,0], A_2 = [0,1]\), then
        \begin{align*}
            A_1^\circ \cup A_2^\circ = (-1,0) \cup (0,1) \subsetneq (-1,1) = (A_1 \cup A_2)^\circ
        \end{align*}

        \textbf{(b)} Once again, it will suffice to show that \(\overline{A_1} \supset \overline{A_1 \cap A_2}\), then \(\overline{A_2} \supset \overline{A_1 \cap A_2} \implies \overline{A_1} \cap \overline{A_2} \supset \overline{A_1 \cap A_2}\) will follow by symmetry. Note that \(\overline{A_1} \supset A_1 \supset A_1\cap A_2\) is closed, so that in particular
        \begin{align*}
            \overline{A_1 \cap A_2} \supset \overline{A_1} \cap \overline{A_1 \cap A_2} \supset A_1 \cap (A_1 \cap A_2) = A_1 \cap A_2
        \end{align*}
        is closed, implying that since \(\overline{A_1 \cap A_2}\) is smallest closed set containing \(A_1 \cap A_2\), we must have \(\overline{A_1 \cap A_2} = \overline{A_1} \cap \overline{A_1 \cap A_2}\) which, in particular, implies that \(\overline{A_1 \cap A_2} \subset \overline{A_1}\).

        To show equality need not hold, consider the usual (metric) topology on \(\mathbb{R}\), with \(A_1 = (-1,0), A_2 = (0,1)\), then
        \begin{align*}
            \overline{A_1}\cap \overline{A_2} = \set{0} \supsetneq \emptyset = \overline{A_1 \cap A_2}
        \end{align*}
    \end{pb}
    \begin{pb}
        An immediately equivalent condition to nowhere density is that the closure contains no (non-empty) open sets. Furthermore, notice that if \(A\) is nowhere dense, then so is \(\overline{A}\) which follows from \(\overline{A} = \overline{\overline{A}}\).

        With the above in mind, let \(U\) be open, then \(V = U \setminus \overline{A} = U \cap \overline{A}^c\) is nonempty by nowhere density of \(A\) (and hence \(\overline{A}\)), and is furthermore the intersection of two open sets thus open. It follows that \(V \setminus \overline{B} = V \cap \overline{B}^c \neq \emptyset\) by nowhere density of \(B\). This implies that
        \begin{align*}
            &U \setminus \overline{A\cup B} = U \cap (\overline{A}\cup \overline{B})^c = (U \cap \overline{A}^c) \cap \overline{B}^c = V \cap \overline{B}^c \neq \emptyset \\
            \implies U \not \subset \overline{A\cup B}
        \end{align*}
        Since \(U\) was arbitrary, we can conclude that \(\overline{A\cup B}\) contains no non-empty open sets and is thus nowhere dense.
    \end{pb}
    \begin{pb}
        \(\emptyset, X \in \mathcal{T}(\mathcal{E})\) is immediate. Now let \(\set{U_\alpha}_{\alpha\in A} \subset \mathcal{T}(\mathcal{E})\), if each \(U_\alpha\) is empty, then we are done, otherwise if \(X = U_{\alpha'}\) for some \(\alpha'\) we have 
        \begin{align*}
            \bigcup_{\alpha \in A}U_\alpha = X \bigcup_{\alpha \in A \setminus \alpha'} U_\alpha = X \in \mathcal{T}(\mathcal{E})
        \end{align*}
        Now we may assume without loss of generality that \(X \neq U_\alpha, \forall \alpha\), and that \(U_\alpha \neq \emptyset \forall \alpha\), since this doesn't affect the union, then
        each of the sets in the union \(\bigcup_{\alpha\in A}U_\alpha\) are unions of finite intersections of sets in \(\mathcal{E}'\), so we can rewrite
        \begin{align*}
            \bigcup_{\alpha \in A}U_\alpha = \bigcup_{\alpha \in A} \bigcup_{\beta \in B_\alpha} U_{\alpha,\beta} = \bigcup_{(\alpha,\beta) \in S} U_{\alpha,\beta}
        \end{align*}
        where \(U_\alpha = \bigcup_{\beta \in B_\alpha} U_{\alpha,\beta}\), and each \(U_{\alpha,\beta} \in \mathcal{E}'\). And where \(S\) is defined as \(\set{(\alpha,\beta) \vert \alpha \in A, \beta \in B_\alpha}\) which suffices to show that \(\bigcup_{\alpha \in A} U_\alpha \in \mathcal{T}(\mathcal{E})\) by definition of \(\mathcal{T}(\mathcal{E})\).
    \end{pb}
\end{document}