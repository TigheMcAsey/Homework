\documentclass[10.5pt]{article}
\usepackage{amsmath, amsfonts, amssymb,amsthm}
\usepackage[includeheadfoot]{geometry} % For page dimensions
\usepackage{fancyhdr}
\usepackage{enumerate} % For custom lists

\fancyhf{}
\lhead{Math 421hw1}
\rhead{Tighe McAsey - 37499480}
\pagestyle{fancy}
% Page dimensions
\geometry{a4paper, margin=1in}

\theoremstyle{definition}
\newtheorem{pb}{}

% Commands:

\newcommand{\set}[1]{\{#1\}}
\newcommand{\abs}[1]{\left\vert#1\right\vert}
\newcommand{\norm}[1]{\lvert\lvert#1\rvert\rvert}
\newcommand{\tand}{\text{ and }}
\newcommand{\tor}{\text{ or }}
\newcommand{\floor}[1]{\left\lfloor #1 \right\rfloor}
\newcommand{\ceil}[1]{\left\lceil #1 \right\rceil}
\newcommand{\re}{\text{Re}}
\newcommand{\im}{\text{Im}}

\begin{document}
    \begin{pb}
        \textbf{(a)} Without loss of generality it will suffice to show \(A_1^\circ \subset (A_1 \cup A_2)^\circ\), then the case \(A_2^\circ \subset (A_1 \cup A_2)^\circ\) is symmetrical. We have that \(A_1^\circ \subset A_1 \subset A_1 \cup A_2\), which is open, then
        \[(A_1 \cup A_2)^\circ \subset (A_1 \cup A_2)^\circ \cup A_1^\circ \subset A_1 \cup A_2\]
        is open, and since  \((A_1 \cup A_2)^\circ\) is the largest open set contained in \(A_1 \cup A_2\), it must be the case that
        \[(A_1 \cup A_2)^\circ = (A_1 \cup A_2)^\circ \cup A_1^\circ\]
        so that \(A_1^\circ \subset (A_1 \cup A_2)^\circ\). \qed
        
        To show that equality need not hold, consider the usual (metric) topology on \(\mathbb{R}\), with \(A_1 = [-1,0], A_2 = [0,1]\), then
        \begin{align*}
            A_1^\circ \cup A_2^\circ = (-1,0) \cup (0,1) \subsetneq (-1,1) = (A_1 \cup A_2)^\circ
        \end{align*}

        \textbf{(b)} Once again, it will suffice to show that \(\overline{A_1} \supset \overline{A_1 \cap A_2}\), then \(\overline{A_2} \supset \overline{A_1 \cap A_2} \implies \overline{A_1} \cap \overline{A_2} \supset \overline{A_1 \cap A_2}\) will follow by symmetry. Note that \(\overline{A_1} \supset A_1 \supset A_1\cap A_2\) is closed, so that in particular
        \begin{align*}
            \overline{A_1 \cap A_2} \supset \overline{A_1} \cap \overline{A_1 \cap A_2} \supset A_1 \cap (A_1 \cap A_2) = A_1 \cap A_2
        \end{align*}
        is closed, implying that since \(\overline{A_1 \cap A_2}\) is smallest closed set containing \(A_1 \cap A_2\), we must have \(\overline{A_1 \cap A_2} = \overline{A_1} \cap \overline{A_1 \cap A_2}\) which, in particular, implies that \(\overline{A_1 \cap A_2} \subset \overline{A_1}\). \qed

        To show equality need not hold, consider the usual (metric) topology on \(\mathbb{R}\), with \(A_1 = (-1,0), A_2 = (0,1)\), then
        \begin{align*}
            \overline{A_1}\cap \overline{A_2} = \set{0} \supsetneq \emptyset = \overline{A_1 \cap A_2}
        \end{align*}
    \end{pb}
    \begin{pb}
        An immediately equivalent condition to nowhere density is that the closure contains no (non-empty) open sets. Furthermore, notice that if \(A\) is nowhere dense, then so is \(\overline{A}\) which follows from \(\overline{A} = \overline{\overline{A}}\).

        With the above in mind, let \(U\) be open, then \(V = U \setminus \overline{A} = U \cap \overline{A}^c\) is nonempty by nowhere density of \(A\) (and hence \(\overline{A}\)), and is furthermore the intersection of two open sets thus open. It follows that \(V \setminus \overline{B} = V \cap \overline{B}^c \neq \emptyset\) by nowhere density of \(B\). This implies that
        \begin{align*}
            &U \setminus \overline{A\cup B} = U \cap (\overline{A}\cup \overline{B})^c = (U \cap \overline{A}^c) \cap \overline{B}^c = V \cap \overline{B}^c \neq \emptyset \\
            \implies &U \not \subset \overline{A\cup B}
        \end{align*}
        Since \(U\) was arbitrary, we can conclude that \(\overline{A\cup B}\) contains no non-empty open sets and is thus nowhere dense.
    \end{pb}
    \begin{pb}
        \(\emptyset, X \in \mathcal{T}(\mathcal{E})\) is immediate. Now let \(\set{U_\alpha}_{\alpha\in A} \subset \mathcal{T}(\mathcal{E})\), if each \(U_\alpha\) is empty, then we are done, otherwise if \(X = U_{\alpha'}\) for some \(\alpha'\) we have 
        \begin{align*}
            \bigcup_{\alpha \in A}U_\alpha = X \bigcup_{\alpha \in A \setminus \alpha'} U_\alpha = X \in \mathcal{T}(\mathcal{E})
        \end{align*}
        Now we may assume without loss of generality that \(X \neq U_\alpha, \forall \alpha\), and that \(U_\alpha \neq \emptyset, \forall \alpha\), since this doesn't affect the union, then
        each of the sets in the union \(\bigcup_{\alpha\in A}U_\alpha\) are unions of finite intersections of sets in \(\mathcal{E}'\), so we can rewrite
        \begin{align*}
            \bigcup_{\alpha \in A}U_\alpha = \bigcup_{\alpha \in A} \bigcup_{\beta \in B_\alpha} U_{\alpha,\beta} = \bigcup_{(\alpha,\beta) \in S} U_{\alpha,\beta}
        \end{align*}
        where \(U_\alpha = \bigcup_{\beta \in B_\alpha} U_{\alpha,\beta}\), and each \(U_{\alpha,\beta} \in \mathcal{E}'\). And where \(S\) is defined as \(\set{(\alpha,\beta) \vert \alpha \in A, \beta \in B_\alpha}\) which suffices to show that \(\bigcup_{\alpha \in A} U_\alpha \in \mathcal{T}(\mathcal{E})\) by definition of \(\mathcal{T}(\mathcal{E})\). To show that \(\mathcal{T}(\mathcal{E})\) is closed under finite intersections, it suffices to show that it is closed under intersections of two elements (then induction may be applied to show it for finite intersections). So let \(U, V \in \mathcal{T}(\mathcal{E})\), if either is \(X \tor \emptyset\), then we are done trivially, so assume not. Then we can write \(U = \bigcup_{\alpha\in A}E_\alpha, V = \bigcup_{\beta\in B}E_\beta\), for \(E_\alpha, E_\beta \in \mathcal{E}'\). Then
        \begin{align*}
            U \cap V &= \left(\bigcup_{\alpha\in A} E_\alpha\right)\cap \left(\bigcup_{\beta\in B} E_\beta\right) = \bigcup_{\alpha\in A}\left( E_\alpha \cap \bigcup_{\beta\in B} E_\beta\right) = \bigcup_{\alpha\in A}\bigcup_{\beta\in B}E_\alpha\cap E_\beta = \bigcup_{(\alpha,\beta) \in A \times B} E_\alpha\cap E_\beta
        \end{align*}
        the intersection of two sets in \(\mathcal{E}'\) is still a finite intersection of sets in \(\mathcal{E}\), so \(U \cap V\) is still in \(\mathcal{T}(\mathcal{E})\). This suffices to show that \(\mathcal{T}(\mathcal{E})\) is a topology.
        

        To show that \(\mathcal{T}(\mathcal{E})\) is the topology generated by \(\mathcal{E}\), we need to show that every topology \(\mathcal{T}\) on \(X\), such that \(\mathcal{E} \subset \mathcal{T}\) is such that \(\mathcal{T} \supset \mathcal{T}(\mathcal{E})\). Let \(\mathcal{T}\) be a topology on \(X\) containing \(\mathcal{E}\), and suppose that \(U \in \mathcal{T}(\mathcal{E})\), if \(U = X \tor \emptyset\), then \(X \in \mathcal{T}\) by definition of a topology. Otherwise \(U = \bigcup_{\alpha \in A}\bigcap_1^{N_\alpha} E^\alpha_i\) for some \(E^{\alpha}_i \in \mathcal{E}\), but then since \(\mathcal{E} \subset \mathcal{T}\), so are finite intersections hence \(\set{\bigcap_1^{N_\alpha} E^\alpha_i \vert \alpha \in A} \subset \mathcal{T}\), and since \(\mathcal{T}\) is closed under arbitrary unions, we have that
        \begin{align*}
            U = \bigcup_{\alpha \in A} \bigcap_1^{N_\alpha} E^\alpha_i \in \mathcal{T}
        \end{align*}
        so that \(\mathcal{T}(\mathcal{E}) \subset \mathcal{T}\) as desired, this suffices to show that \(\mathcal{T}(E) = \bigcap_{\text{Topologies on }X \text{ containing }\mathcal{E}} \mathcal{T}\) (equality since \(\mathcal{T}(\mathcal{E})\) is one of the elements of the intersection, and is included in all sets in the intersection).
    \end{pb}
    \begin{pb}
        \textbf{(a)} Consider a collection of sets \(\set{S_j}_{j=1}^\infty\), such that each \(S_j\) is a neighborhood of \(p\), then each \(S_j\) contains some open set, \(V_j\) which contains \(p\). If \(S\) were a neighborhood base, then for any open set \(U \in \mathcal{T}\), such that \(p \in U\), we would have some \(V_j \subset S_j \subset U\), so it will suffice to show that there exists an open set \(U\), such that \(V_j \not \subset U\) for any \(j\) to conclude that \(\set{S_j}_1^\infty\) is not a neighborhood base for \(p\). Then since \(\set{S_j}_1^\infty\) was arbitrary this suffices to show that \(p\) does not have a countable neighborhood base.

        In problem (3), we provided a closed form for the open sets in the topology generated by a set. Applying this result, we may write
        \begin{align*}
            V_j = \bigcup_{\beta \in B_j}\bigcap_{i = 1}^{N_{\beta}} X_{a^i_{\beta},k^i_{\beta}}
        \end{align*}
        where for each \(B_j\), we have some \(\beta^j\) such that \(k^i_{\beta^j} = 1\) for all \(i\) (this follows from \(V_j\) being a neighborhood of \(p\)).

        By countability of the \(V_j\), it follows that \(\set{a^i_{\beta^j}}_{1 \leq j, \; 1 \leq i \leq N_j}\) is countable, and thus there is some \(a \in A\), such that \(a \not \in \set{a^i_{\beta^j}}_{1 \leq j, \; 1 \leq i \leq N_j}\). Considering the set \(U = X_{a,1}\), it is immediate that \(U\) is a neighborhood of \(p\), and furthermore \(U\) does not contain \(V_j\) for any \(j\), as proof consider
        \begin{align*}
            f:x \mapsto \begin{cases}
                1 & x \neq a \\
                0 & x = a
            \end{cases}
        \end{align*}
        since \(a \neq a^i_{\beta^j}, \; \forall j\), we have that
        \begin{align*}
            f \in \bigcap_{i = 1}^{N_{\beta^j}} X_{a^i_{\beta^j},k^i_{\beta^j}} \subset V_j, \tand
            f \not \in U
        \end{align*}

        \textbf{(b)} Suppose such a metric exists, then consider the collection
            \(\mathcal{B} := \set{B_{\frac{1}{n}}(p)}_{n=1}^\infty\), this constitutes a neighborhood base for \(p\), (since any open set containing \(p\) must contain a ball of radius \(\epsilon\) around \(p\), and we can choose \(n > \frac{1}{\epsilon}\)) contradicting part (a).

        \textbf{(c)} Suppose for contradiction such a neighborhood base exists \(\set{V_\beta}_{\beta \in B}\), then let \(\set{a_i}_{i=1}^\infty\) be a countable subset of \(A\). Define sets \(\set{U_n}_1^\infty\), such that
        \begin{align*}
            U_n = \bigcap_{i=1}^n X_{a_i,1}
        \end{align*}
        then by definition of neighborhood base, for each \(U_n\), we may choose some \(V_{\beta^n}\), such that \(V_{\beta^n} \subset U_n\). By part (a),
        \(\set{V_{\beta^n}}_{n=1}^\infty\) cannot be a neighboorhood base for \(p\) since it is a countable set. Then there is some open set \(U\), such that
        \(V_{\beta^n} \not \subset U\) for any \(n\). So there must be some \(V_{\beta'} \not \in \set{V_{\beta^n}}_{n=1}^\infty, \text{ such that } V_{\beta'} \subset U\). If \(V_{\beta^n} \subset V_{\beta'}\) for some \(n\), then \(V_{\beta^n} \subset U\) is a contradiction, so by our total order condition, we must have that \(V_{\beta'} \subset V_{\beta^n}, \; \forall n\). This is a contradiction since \(\left(\bigcap_1^\infty V_{\beta^n}\right)^\circ \subset \left(\bigcap_1^\infty U_n\right)^\circ = \emptyset\), which is immediate from our characterization of the product topology from question 3. The details are, any open set must contain some finite intersection of basis elements (question (3)), but any \(\bigcap_{i=1}^N X_{a_i',k}\) contains the function \(f: \begin{cases}
            x \mapsto 1 & x \in \set{a_i'}_{i=1}^N \\
            x \mapsto 0 & \text{else}
        \end{cases}\), which maps finitely many values to \(1\), but for any \(f \in \bigcap_1^\infty U_n, \; f^{-1}(\set{1})\) is an infinite set.

        \textbf{(d)} It will suffice to show that every point of \(X\) is an either in \(K\), or an accumulation point of \(K\), equivalently fixing a point \(g \in X\), we have for any open set \(U\) containing \(g\), \(U \cap K \neq \emptyset\). Let \(g \in X\), and \(U\) an open set containing \(g\). Then from question 3, we know that there is some non-empty \(\bigcap_{i=1}^N X_{a_i,k_i} \subset U\). Then we may define the function
        \begin{align*}
            f: \begin{cases}
                x \mapsto k_i & x \in \set{a_i}_1^N \\
                x \mapsto 0 & \text{else}
            \end{cases}, \text{ by construction } f \in K\cap U
        \end{align*}

        \textbf{(e)} By countability of \(H\), we have \(S := \bigcup_{f \in H}\bigcup_{\set{a \in A\mid f(a) = 1}} a\) is a countable union of countable sets and hence countable. Hence 
        \begin{align}
            &H \subset \bigcap_{a \in A \setminus S} X_{a,0} \subset K \\
            &\bigcap_{a \in A \setminus S} X_{a,0} = \bigcap_{a \in A \setminus S} X_{a,1}^c
        \end{align}
        By (2), we have that \(\bigcap_{a \in A \setminus S} X_{a,0}\) is equal to an intersection of closed sets hence closed, implying that \(\overline{H} \subset \bigcap_{a \in A \setminus S} X_{a,0}\) and by (1), we have that \(\overline{H} \subset \bigcap_{a \in A \setminus S} X_{a,0} \subset K\).
    \end{pb}
\end{document}