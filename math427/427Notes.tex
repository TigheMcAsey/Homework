\documentclass[11pt]{article}
\usepackage{amsmath, amsfonts, amssymb,amsthm}
\usepackage[includeheadfoot]{geometry} % For page dimensions
\usepackage{fancyhdr}
\usepackage{enumerate} % For custom lists
\usepackage{tikz-cd}
\usepackage{xr}
\externaldocument{427Exercises}

\fancyhf{}
\lhead{427 Notes}
\rhead{Tighe McAsey}
\pagestyle{fancy}

% Page dimensions
\geometry{a4paper, margin=1in}

\theoremstyle{definition}
\newtheorem{pb}{}

% Commands:

\newcommand{\set}[1]{\{#1\}}
\newcommand{\abs}[1]{\lvert#1\rvert}
\newcommand{\norm}[1]{\lvert\lvert#1\rvert\rvert}
\newcommand{\gen}[1]{\left\langle #1 \right\rangle}
\newcommand{\tand}{\text{ and }}
\newcommand{\tor}{\text{ or }}
\newcommand{\falg}{F^{\text{alg}}}
\newcommand{\gal}{\text{Gal}}
\newcommand{\mor}{\text{Mor}}
\newcommand{\floor}[1]{\left\lfloor #1 \right\rfloor}
\newcommand{\im}{\text{Im}}

\begin{document}
    \section{Chain Complexes}\label{CCNotes} \ref{CCEx}

    \textbf{CAUTION!! - } The connection (this is what we call the diagonal maps often denoted \(\mathbf{h}\)) diagram need to not commute.

    \section{Homology}

    \subsection{The Cellular Chain Complex}

    \textbf{Definition of the cellular chain complex: } The cellular chain complex is defined on a CW complex, given by \(C_n = H_n(X^n/X^{n-1})\), 
    it is easy to see that this is free on n-cells since this is the homology of that many n-spheres.

    \textbf{Calculating differential maps: } Given the above, construction of cellular homology from an ES or reduced ES theory, we can calculate the image of the
    chain map \(d_n\), by simply calculating the degree on every component. I.e. Let \(\set{e_\alpha^{n-1}}_A \tand \set{e_\beta^{n-1}}_B\) be the \(n \tand n-1\)-cells in \(X\), 
    then \(d_n\) is determined by its degree on each of the \(\del e_\alpha^n\). In particular we have the familiar formula (the quotient in the image is commonly called \(S^{n-1}_\beta\)).
    \[d_n (e^n_\alpha) = \deg (q_\beta\varphi_\alpha): \partial e_\alpha \to \frac{X^{n-1}}{X^{n-1} \setminus e^{n-1}_\beta}\]

    \section{Cohomology}
\end{document}