\documentclass[11pt]{article}
\usepackage{amsmath, amsfonts, amssymb,amsthm}
\usepackage[includeheadfoot]{geometry} % For page dimensions
\usepackage{fancyhdr}
\usepackage{enumerate} % For custom lists
\usepackage{tikz-cd}

\fancyhf{}
\lhead{Math 423hw3}
\rhead{Tighe McAsey - 37499480}
\pagestyle{fancy}

% Page dimensions
\geometry{a4paper, margin=1in}

\theoremstyle{definition}
\newtheorem{pb}{}

% Commands:

\newcommand{\set}[1]{\{#1\}}
\newcommand{\abs}[1]{\lvert#1\rvert}
\newcommand{\norm}[1]{\lvert\lvert#1\rvert\rvert}
\newcommand{\gen}[1]{\left\langle #1 \right\rangle}
\newcommand{\tand}{\text{ and }}
\newcommand{\tor}{\text{ or }}
\newcommand{\falg}{F^{\text{alg}}}
\newcommand{\gal}{\text{Gal}}
\newcommand{\floor}[1]{\left\lfloor #1 \right\rfloor}
\newcommand{\spec}{\text{Spec}}
\newcommand{\nil}{\text{Nil}}
\newcommand{\jac}{\text{Jac}}
\newcommand{\ho}{\text{Hom}}

\begin{document}
    \color{red} RED = Answer with errors, \color{blue} BLUE = Corrected answer, \color{black} BLACK = Correct original answer

    \begin{pb}
        Let \(P \subset A\) be prime, then it will suffice to show that \(A/P\) is a field which is equivalent to maximality of \(P\) by the correspondence theorem. Consider \(0 \neq x \in A/P\), then choose \(n \geq 2\) such that \(x^n = x\), it follows that \(x(1-x^{n-1}) = x-x = 0\), and since \(P\) is prime \(A/P\) is a domain which implies that \(1-x^{n-1} = 0\), so that \(x^{n-1} = 1\) in \(A/P\). \qed
    \end{pb}
    \begin{pb}
        \color{red}
        Suppose that \(M\) is not flat, then we can fix modules \(A,B\), such that
        \begin{equation*}
            \begin{tikzcd}
                0 \arrow[r,"{}"] & A \arrow[r,"{f}"] & B
            \end{tikzcd}
        \end{equation*}
        is exact, but
        \begin{equation*}
            \begin{tikzcd}
                0 \arrow[r,"{}"] & A\otimes M \arrow[r,"{f\otimes1_M}"] & B \otimes M
            \end{tikzcd}
        \end{equation*}
        is not. It follows that there is some \(0 \neq \sum_1^n a_i \otimes x_i \in A \otimes M\), such that \(f\otimes1_M(\sum_1^n a_i\otimes x_i) = 0\). I claim that \(M_0 = (x_1,\hdots,x_n)\) is the desired submodule. To see this, note \((f \times 1)\vert_{A \times M_0} = f\times 1_{M_0}\), and if \(j\) is the map \(A \times M \to A \otimes M\) in the definition of the tensor, then \(j\vert_{A \times M_0}\) is equal the map \(A \times M_0 \to A \otimes M_0\) in the definition of the tensor. It follows that for any \(v \in A \otimes M_0\), \(v = j\vert_{A \times M_0}(u), \; u \in A \times M_0\), so that
        \begin{align*}
            f \otimes 1_{M_0}(v) = f \otimes 1_{M_0}j\vert_{A \times M_0}(u) = f \times 1_{M_0}(u) = f \times 1_M (u) = f\otimes1_Mj(u)
        \end{align*}
        and hence \(f \otimes 1_{M_0}j\vert_{M_0}(\sum_1^n (a_i, x_i)) = f\otimes1_Mj(\sum_1^n (a_i,x_i)) = f\otimes1_M(\sum_1^n a_i \otimes x_i) = 0\), where \(0 \neq \sum_1^n a_i \otimes x_i =  j(\sum_1^n (a_i,x_i)) = j\vert_{A \times M_0}(\sum_1^n (a_i,x_i))\) which suffices to show that \(f \otimes 1_{M_0}\) is not injective, and hence \(M_0\) is not flat, with the following sequence as witness.
        \begin{equation*}
            \begin{tikzcd}
                0 \arrow[r,"{}"] &A \otimes M_0 \arrow[r,"{f \otimes 1_{M_0}}"] &B \otimes M_0 \qed
            \end{tikzcd}
        \end{equation*}
    \end{pb}
    \begin{pb}
        Since \(C[X]\) is a PID, it satisfies Bezout's identity. So assume \(f_1, f_2\) are coprime polynomials, it follows that there exist \(g,h \in \mathbb{C}[X]\), such that \(f_1h + f_2g = 1\). Now let \(m \otimes n \in M_1 \otimes M_2\), it follows that
        \begin{align*}
            m \otimes n &= (f_1h + f_2g)(m \otimes n) = f_1h(m \otimes n) + f_2g(m \otimes n) = h(f_1m \otimes n) + g(m\otimes f_2n) \\
            &= h(0 \otimes n) + g(m \otimes 0) = 0
        \end{align*}
        Conversely, let \(a \in \mathbb{C}\), such that \(f_1(a) = f_2(a) = 0\). Let \(I = (X-a)\) and consider the map multiplication map \[m: \mathbb{C}[X] \times \mathbb{C}[X] \to \mathbb{C}[X]/(X-a),\; (f,g) \mapsto fg + I\]
        To see that this defines a bilinear map \(M_1 \times M_2 \to \mathbb{C}[X]/I\) it will suffice to check that \(m\) is well defined on cosets so that we can take the induced bilinear map \[\overline{m}: M_1 \times M_2 \to \mathbb{C}[X]/I, \;(f+(f_1),g+(f_2)) \mapsto fg + I\] Let \(g_1, g_2, h_1,h_2 \in \mathbb{C}[X]\), then
        \begin{align*}
            m(g_1 + h_1f_1,g_2 + h_2f_2) = g_1g_2 + g_1h_2f_2 + g_2h_1f_1 + h_1h_2f_1f_2 + I = g_1g_2 + I
        \end{align*}
        the last equality following since both \(f_i \in I\). It follows that \(\overline{m}: M_1 \times M_2 \to \mathbb{C}(X)/I\) is a nonzero (since \((1,1) \mapsto 1\)) bilinear map, so \(\overline{m} = \eta j\) where \(j\) is the map from the definition of the tensor product and \(\eta: M_1 \otimes M_2 \to \mathbb{C}[X]/I\). Since \(\overline{m}\) is non-zero, it follows that \(\eta\) is nonzero and hence \(M_1 \otimes M_2 \neq 0\) since \(\eta \not \in \set{0} = \text{Hom}(0,\mathbb{C}[X]/I)\). \qed
    \end{pb}

    \begin{pb}
        \color{red}
        Consider the exact sequence of \(A\) modules
        \begin{equation*}
            \begin{tikzcd}
                0 \arrow[r,"{}"] &(t) \arrow[r,"{\iota}"] & A
            \end{tikzcd}
        \end{equation*}
        Where \(\iota: t\mapsto t\), injectivity and therefore exactness is clear. To see \(N\) is not flat, tensor the above sequence to get
        \begin{equation*}
            \begin{tikzcd}
                0 \arrow[r,"{}"] &(t)\otimes_A N \arrow[r,"{\iota_*}"] &A \otimes_A N
            \end{tikzcd}
        \end{equation*}
        Here we have
        \begin{align*}
            \iota_*(t\otimes e_3) = t(1\otimes e_3) = 1\otimes t e_3 = 1 \otimes 0 = 0
        \end{align*}
        So it will suffice to show that \(0 \neq t \otimes e_3 \in (t)\otimes_A N\) to conclude that \(\iota_*\) is not injective. Consider \(\phi: N \to N\), \(\phi: e_i \mapsto \delta_{i3}e_2\) and extending linearly (here \(\delta_{i3}\) is the Kronecker delta). Define the map \(\varphi: (t) \times N \to N\) via \(\varphi: (x,y) \mapsto x\phi(y)\), it is immediate that \(\varphi\) is \(A\)-bilinear, hence \(\varphi\) factors through \(j: (t)\times N \to (t)\otimes_A N\). We have that \(\varphi(t,e_3) = te_2 = e_1 \neq 0\), so that since \(\varphi\) factors through \(j\) we have \(t \otimes e_3 = j(t,e_3) \neq 0\). \qed
        \color{blue}

        Consider the exact sequence of \(A\) modules
        \begin{equation*}
            \begin{tikzcd}
                0 \arrow[r,"{}"] &(t) \arrow[r,"{\iota}"] & A
            \end{tikzcd}
        \end{equation*}
        Where \(\iota: t\mapsto t\), injectivity and therefore exactness is clear. To see \(N\) is not flat, tensor the above sequence to get
        \begin{equation*}
            \begin{tikzcd}
                0 \arrow[r,"{}"] &(t)\otimes_A N \arrow[r,"{\iota_*}"] &A \otimes_A N
            \end{tikzcd}
        \end{equation*}
        Here we have
        \begin{align*}
            \iota_*(t\otimes e_3) = t(1\otimes e_3) = 1\otimes t e_3 = 1 \otimes 0 = 0
        \end{align*}
        So it will suffice to show that \(0 \neq t \otimes e_3 \in (t)\otimes_A N\) to conclude that \(\iota_*\) is not injective. Consider the map \(\varphi: (t) \times N \to N, \; (tx,y) \mapsto xy\), bilinearity is a consequence of bilinearity of multiplication. By the universal property of the tensor product we know that \(\varphi\) factors through \(j: (t) \times N \to (t) \otimes_A N\), furthermore \(\varphi(t,e_3) = e_3 \neq 0\) implies that \(t \otimes e_3 = j(t,e_3) \neq 0\). \qed
    \end{pb}

    \color{black}
    \begin{pb}
        Suppose that \(r \leq n\), and \(g_1,g_2,\hdots,g_r\) generate \(I\) as an \(A\) module. It is immediate that \(I^2\) is the ideal generated by all degree 2 monomials of \(A\), it follows that by assumption each monomial in \(f_1,\hdots f_m\) is divisible by some element of \(I^2\), and hence \((f_i)_1^m/I^2 = 0\). Furthermore, \(\set{g_i}_1^r\) generating \(I\) as an \(A\)-module implies that \(\set{g_i + I^2}_1^r\) generate \(I/I^2\) as an \(A/I\) module, since \(A/I \otimes_A I \cong I/I^2\) (here the bilinear map inducing isomorphism is multiplication). Applying the third isomorphism theorem, \[A/I \cong \frac{\mathbb{R}[X_1,\hdots,X_n]/I}{(f_1,\hdots,f_m)/I} \cong \mathbb{R}[X_1,\hdots,X_n]/I \cong \mathbb{R}\]
        so that in fact \(\set{g_i + I^2}_1^r\) span \(I/I^2\) as an \(A/I\) vectorspace. Here
        \begin{align*}
            I/I^2 \cong \bigoplus_1^n X_iA/I
        \end{align*}
        where both the spanning and zero intersection properties are obvious, implying that \(I/I^2\) has dimension \(n\) as an \(A/I\) vectorspace, since any spanning set must have atleast \(n\) elements, we conclude that that \(r = n\) \qed
    \end{pb}

    \begin{pb}
        \color{red}
        \(A[X] = \bigoplus_0^\infty AX^i\) as an \(A\)-module, assume for contradiction that \(\bigoplus_0^\infty AX^i \cong \bigoplus_0^\infty A\) is not flat, then applying problem 2, there is some finitely generated submodule \(M_0\), such that \(M_0\) is not flat. Since submodules of free modules are free, we know that \(M_0 \cong \bigoplus_1^n A\), implying that \(\bigoplus_1^n A\) is not flat, but this is a contradiction, since this is only the case if
        \begin{equation*}
            \begin{tikzcd}
                0 \arrow[r,"{}"] &K \arrow[r,"{f}"] &L
            \end{tikzcd}
        \end{equation*}
        is exact, but the following sequence is not
        \begin{equation*}
            \begin{tikzcd}
                0 \arrow[r,"{}"] &K \otimes \bigoplus_1^n A \arrow[r,"{f \otimes 1_{\bigoplus_1^n A}}"] &L \otimes \bigoplus_1^n A
            \end{tikzcd}
        \end{equation*}
        but this is equivalent to the following sequence not being exact
        \begin{equation*}
            \begin{tikzcd}
                0 \arrow[r,"{}"]& \bigoplus_1^n K\otimes A \arrow[r,"{\bigoplus_1^n f \otimes 1_A}"] &\bigoplus_1^n L\otimes A
            \end{tikzcd}
        \end{equation*}
        which once again is equivalent to the following not being exact
        \begin{equation*}
            \begin{tikzcd}
                0 \arrow[r,"{}"] & \bigoplus_1^n K \arrow[r,"{\bigoplus_1^n f}"] & \bigoplus_1^n L
            \end{tikzcd}
        \end{equation*}
        where \(\bigoplus_1^n f\) is injective since \(f\) is. \qed

        \color{blue} \textbf{Atiyah \& Macdonald 2.5.} [A may not be a PID, however we have that tensor commutes with arbitrary direct sums.] \(A[X] = \bigoplus_0^\infty AX^i\) as an \(A\)-module, assume for contradiction that \(\bigoplus_0^\infty AX^i \cong \bigoplus_0^\infty A\) is not flat, then there must exist some modules \(K,L\) and some \(f: K \to L\), such that
        \begin{equation*}
            \begin{tikzcd}
                0 \arrow[r,"{}"] &K \arrow[r,"{f}"] &L
            \end{tikzcd}
        \end{equation*}
        is exact, but the following sequence is not
        \begin{equation*}
            \begin{tikzcd}
                0 \arrow[r,"{}"] &K \otimes \bigoplus_0^\infty A \arrow[r,"{f \otimes 1_{\bigoplus_0^\infty A}}"] &L \otimes \bigoplus_0^\infty A
            \end{tikzcd}
        \end{equation*}
        but this is equivalent to the following sequence not being exact
        \begin{equation*}
            \begin{tikzcd}
                0 \arrow[r,"{}"]& \bigoplus_0^\infty K\otimes A \arrow[r,"{\bigoplus_0^\infty f \otimes 1_A}"] &\bigoplus_0^\infty L\otimes A
            \end{tikzcd}
        \end{equation*}
        which once again is equivalent to the following not being exact
        \begin{equation*}
            \begin{tikzcd}
                0 \arrow[r,"{}"] & \bigoplus_0^\infty K \arrow[r,"{\bigoplus_0^\infty f}"] & \bigoplus_0^\infty L
            \end{tikzcd}
        \end{equation*}
        where \(\bigoplus_0^\infty f\) is injective since \(f\) is. \qed

        \textbf{Atiyah \& Macdonald 3.5.} Assume  for contradiction that \(A\) has a nilpotent element, \(0 \neq x \in A\), such that \(x^n = 0\) (we may WLOG take \(n\) to be the smallest such exponent). Then \(1 \not \in \text{ann}(x) \subset A\), hence there is some maximal ideal \(\mathfrak{m} \supset \text{ann}(x) \supset x^{n-1}\). Since \(\left(\frac{x}{1}\right)^n = 0\) in \(A_{\mathfrak{m}}\) which has no nilpotent elements, it must be the case that \(\frac{x}{1} = 0\) in \(A_{\mathfrak{m}}\), so there is some \(p \in A \setminus\mathfrak{m}\), such that \(px = 0\), but this implies that \(p \in \text{ann}(x) \subset \mathfrak{m}\) which is a contradiction. \qed

        \(\mathbb{C}^2\) is a counterexample (\((1,0)(0,1) = 0\)). It is immediate that the only ideals of \(\mathbb{C}^2\) are \(0, \mathbb{C} \times \set{0} \tand \set{0} \times \mathbb{C}\), the latter two are prime. It will suffice to show that \[\mathbb{C}^2_{\set{0} \times \mathbb{C}} = (\mathbb{C}^\times \times \mathbb{C})^{-1}\mathbb{C}^2\] is a domain by symmetry. Consider the map \(\mathbb{C} \to \mathbb{C}^2_{\set{0} \times \mathbb{C}}\) given by \(a \mapsto \frac{(a,a)}{(1,1)}\), this is injective since for any \((b,c) \in \mathbb{C}^\times \times \mathbb{C}\) we have \((b,c)(a,a) = 0 \implies ba = 0 \implies a = 0\).
        It is also surjective since if \(\frac{(a,b)}{(c,d)} \in \mathbb{C}^2_{\set{0} \times \mathbb{C}}\), then take \(x = \frac{c}{a}\), so that
        \begin{align*}
            (1,0)((x,x)(c,d) - (a,b)(1,1)) = (a,0) - (a,0) = 0
        \end{align*}
        this implies that \(x \mapsto \frac{(a,b)}{(c,d)}\) so that the map is a surjection and hence an isomorphism. This implies that both localizations are isomorphic to \(\mathbb{C}\), and hence both are integral domains without \(\mathbb{C}^2\) being an integral domain. \qed

        For a simpler example, \(\mathbb{Z}/(6)\) also works (the argument is similar). \qed
    \end{pb}

\end{document}