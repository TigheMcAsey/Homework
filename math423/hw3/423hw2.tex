\documentclass[11pt]{article}
\usepackage{amsmath, amsfonts, amssymb,amsthm}
\usepackage[includeheadfoot]{geometry} % For page dimensions
\usepackage{fancyhdr}
\usepackage{enumerate} % For custom lists
\usepackage{tikz-cd}

\fancyhf{}
\lhead{Math 423hw1}
\rhead{Tighe McAsey - 37499480}
\pagestyle{fancy}

% Page dimensions
\geometry{a4paper, margin=1in}

\theoremstyle{definition}
\newtheorem{pb}{}

% Commands:

\newcommand{\set}[1]{\{#1\}}
\newcommand{\abs}[1]{\lvert#1\rvert}
\newcommand{\norm}[1]{\lvert\lvert#1\rvert\rvert}
\newcommand{\gen}[1]{\left\langle #1 \right\rangle}
\newcommand{\tand}{\text{ and }}
\newcommand{\tor}{\text{ or }}
\newcommand{\falg}{F^{\text{alg}}}
\newcommand{\gal}{\text{Gal}}
\newcommand{\floor}[1]{\left\lfloor #1 \right\rfloor}
\newcommand{\spec}{\text{Spec}}
\newcommand{\nil}{\text{Nil}}
\newcommand{\jac}{\text{Jac}}
\newcommand{\ho}{\text{Hom}}

\begin{document}
    \begin{pb}
        Let \(P \subset A\) be prime, then it will suffice to show that \(A/P\) is a field which is equivalent to maximality of \(P\) by the correspondence theorem. Consider \(0 \neq x \in A/P\), then choose \(n \geq 2\) such that \(x^n = x\), it follows that \(x(1-x^{n-1}) = x-x = 0\), and since \(P\) is prime \(A/P\) is a domain which implies that \(1-x^{n-1} = 0\), so that \(x^{n-1} = 1\) in \(A/P\). \qed
    \end{pb}
    \begin{pb}
        Suppose that \(M\) is not flat, then we can fix modules \(A,B\), such that
        \begin{equation*}
            \begin{tikzcd}
                0 \arrow[r,"{}"] & A \arrow[r,"{f}"] & B
            \end{tikzcd}
        \end{equation*}
        is exact, but
        \begin{equation*}
            \begin{tikzcd}
                0 \arrow[r,"{}"] & A\otimes M \arrow[r,"{(1\times f)_*}"] & B \otimes M
            \end{tikzcd}
        \end{equation*}
        is not. It follows that there is some \(0 \neq \sum_1^n a_i \otimes x_i \in A \otimes M\), such that \((1\times f)_*(\sum_1^n a_i\otimes x_i) = 0\). I claim that \(M_0 = (x_1,\hdots,x_n)\) is the desired submodule. To see this, note \((1 \times f)\vert_{A \times M_0} = 1\times f\vert_{M_0}\), and if \(j\) is the map \(A \times M \to A \otimes M_0\) in the definition of the tensor, then \(j\vert_{A \times M_0}\) is equal the map \(A \times M_0 \to A \otimes M_0\) in the definition of the tensor. It follows that for any \(v \in A \otimes M_0\), \(v = j\vert_{A \times M_0}(u), \; u \in A \times M_0\), so that
        \begin{align*}
            (1 \times f\vert_{M_0})_*(v) = (1 \times f\vert_{M_0})_*j\vert_{A \times M_0}(u) = 1 \times f\vert_{M_0}(u) = 1 \times f (u) = (1\times f)_*j(u)
        \end{align*}
        and hence \((1 \times f\vert_{A \times M_0})_*j\vert_{M_0}(\sum_1^n (a_i, x_i)) = (1 \times f)_*j(\sum_1^n (a_i,x_i)) = (1\times f)_*(\sum_1^n a_i \otimes x_i) = 0\), where \(0 \neq \sum_1^n a_i \otimes x_i =  j(\sum_1^n (a_i,x_i)) = j\vert_{A \times M_0}(\sum_1^n (a_i,x_i))\) which suffices to show that \(1 \times f\vert_{M_0}\) is not injective, and hence \(M_0\) is not flat, with the following sequence as witness.
        \begin{equation*}
            \begin{tikzcd}
                0 \arrow[r,"{}"] &A \otimes M_0 \arrow[r,"{(1 \times f\vert_{M_0})_*}"] &B \otimes M_0
            \end{tikzcd}
        \end{equation*}
        !!! NEED TO DEAL WITH \(B \otimes M_0\) !!!
    \end{pb}
    \begin{pb}
        Since \(C[X]\) is a PID, it satisfies Bezout's identity. So assume \(f_1, f_2\) are coprime polynomials, it follows that there exist \(g,h \in \mathbb{C}[X]\), such that \(f_1h + f_2g = 1\). Now let \(m \otimes n \in M_1 \otimes M_2\), it follows that
        \begin{align*}
            m \otimes n &= (f_1h + f_2g)(m \otimes n) = f_1h(m \otimes n) + f_2g(m \otimes n) = h(f_1m \otimes n) + g(m\otimes f_2n) \\
            &= h(0 \otimes n) + g(m \otimes 0) = 0
        \end{align*}
        Conversely, let \(a \in \mathbb{C}\), such that \(f_1(a) = f_2(a) = 0\). Let \(I = (X-a)\) and consider the map multiplication map \[m: \mathbb{C}[X] \times \mathbb{C}[X] \to \mathbb{C}[X]/(X-a),\; (f,g) \mapsto fg + I\]
        To see that this defines a bilinear map \(M_1 \times M_2 \to \mathbb{C}[X]/I\) it will suffice to check that \(m\) is well defined on cosets so that we can take the induced bilinear map \[\overline{m}: M_1 \times M_2 \to \mathbb{C}[X]/I, \;(f+(f_1),g+(f_2)) \mapsto fg + I\] Let \(g_1, g_2, h_1,h_2 \in \mathbb{C}[X]\), then
        \begin{align*}
            m(g_1 + h_1f_1,g_2 + h_2f_2) = g_1g_2 + g_1h_2f_2 + g_2h_1f_1 + h_1h_2f_1f_2 + I = g_1g_2 + I
        \end{align*}
        the last equality following since both \(f_i \in I\). It follows that \(\overline{m}: M_1 \times M_2 \to \mathbb{C}(X)/I\) is a nonzero (since \((1,1) \mapsto 1\)) bilinear map, so \(\overline{m} = \eta j\) where \(j\) is the map from the definition of the tensor product and \(\eta: M_1 \otimes M_2 \to \mathbb{C}[X]/I\). Since \(\overline{m}\) is non-zero, it follows that \(\eta\) is nonzero and hence \(M_1 \otimes M_2 \neq 0\) since \(\eta \not \in \set{0} = \text{Hom}(0,\mathbb{C}[X]/I)\). \qed
    \end{pb}
    \begin{pb}
        Consider the exact sequence of \(\mathbb{R}\) modules
        \begin{equation*}
            \begin{tikzcd}
                0 \arrow[r,"{}"] &\mathbb{R} \arrow[r,"{\iota}"] & A
            \end{tikzcd}
        \end{equation*}
        Where \(\iota: 1\mapsto t\), exactness is clear. The following is well defined by extesion of scalars.
        \begin{equation*}
            \begin{tikzcd}
                0 \arrow[r,"{}"] &\mathbb{R}\otimes_A N \arrow[r,"{\iota_*}"] &A \otimes_A N
            \end{tikzcd}
        \end{equation*}
        Furthermore, \[\iota_* 1 \otimes e_2 = \iota_*1\otimes te_1 = t\iota_*1\otimes e_1 = t(t \otimes e_1) = t^2(1 \otimes e_1) = 0\]
        Now it will suffice to check that \(0 \neq 1 \otimes e_2 \in \mathbb{R} \otimes_A N\). When regarding \(\mathbb{R}\) as an \(A\)-module, via extension of scalars, we are taking \(\mathbb{R} \cong A \otimes_\mathbb{R} \mathbb{R} \cong A\), so we define our \(A\)-bilinear map on \(A \times N\). Simply consider the multiplication map \(m:(a,b) \mapsto ab\), then \(0 \neq e_2 = m(1,e_2)\), since this map factors is bilinear, it factors through \(j\), so that \(1 \otimes e_2 \overset{\text{def}}{=} j(1,e_2) \neq 0\) which suffices to show that \(\ker \iota_* \neq 0\) so that the tensored sequence is not exact and hence \(N\) is not flat. \qed
    \end{pb}
    \begin{pb}
        Suppose that \(r \leq n\), and \(g_1,g_2,\hdots,g_r\) generate \(I\) as an \(A\) module. Let \[J := (X_1^2,X_1X_2,\hdots,X_1X_n,X_2^2,X_2X_3 \hdots, X_n^2)\]
        be the ideal generated by all degree 2 monomials in \(\mathbb{R}[X_1,\hdots,X_n]\), it follows that by assumption each monomial in \(f_1,\hdots f_m\) is divisible by some element of \(J\), and hence \((f_i)_1^m/J = 0\).
        It follows that \(\overline{g_1},\hdots,\overline{g_r}\) generate \(I/J\) as an \(A\)-module, now note that no element of \(I\) has a term with degree \(0\), hence each monomial of \(g_i\) (and hence \(\overline{g_i}\)) has degree atleast one. Since all monomials degree larger than or equal to \(2\) are annihilated in \(I/J\) we may conclude that each \(\overline{g_i} = X_j\) up to units. It follows that each \(\overline{g_i} = X_j\), and we may reindex without loss of generality so that \((g_1,\hdots,g_r) = (X_1,\hdots,X_r)\).

        Applying the third isomorphism theorem, \[A/I \cong \frac{\mathbb{R}[X_1,\hdots,X_n]/I}{(f_1,\hdots,f_m)/I} \cong \mathbb{R}[X_1,\hdots,X_n]/I \cong \mathbb{R}\] is a field, and \((X_1,\hdots,X_r)/J = I/J\) as an \(A\)-module implies that
        \begin{align*}
            \bigoplus_1^r A/I \cong (X_1,\hdots,X_r)/J/(X_1,\hdots,X_r)I = I/J/I^2 \cong \bigoplus_1^n A/I
        \end{align*}
        since the rank of isomorphic vectorspaces must be equal this implies that \(r = n\). \qed
    \end{pb}
    \begin{pb}
        \(A[X] = \bigoplus_0^\infty AX^i\) as an \(A\)-module, since \(A[X^i] \otimes_A M \cong A \otimes_A M \cong M\), it is immediate that \(AX^i\) is flat for each \(i\), assume for contradiction that \(\oplus_0^\infty AX^i \cong \oplus_0^\infty A\) is not flat, then applying problem 2, there is some finitely generated submodule \(M_0\), such that \(M_0\) is not flat. Since submodules of free modules are free, we know that \(M_0 \cong \bigoplus_1^n A\), implying that \(\bigoplus_1^n A\) is not flat, but this is a contradiction, since this is only the case if
        \begin{equation*}
            \begin{tikzcd}
                0 \arrow[r,"{}"] &K \arrow[r,"{f}"] &L
            \end{tikzcd}
        \end{equation*}
        is exact, but the following sequence is not
        \begin{equation*}
            \begin{tikzcd}
                0 \arrow[r,"{}"] &K \otimes \bigoplus_1^n A \arrow[r,"{f \otimes 1_{\oplus_1^n A}}"] &L \otimes \bigoplus_1^n A
            \end{tikzcd}
        \end{equation*}
        but this is equivalent to the following sequence not being exact
        \begin{equation*}
            \begin{tikzcd}
                0 \arrow[r,"{}"]& \bigoplus_1^n K\otimes A \arrow[r,"{\oplus_1^n f \otimes 1_A}"] &\bigoplus_1^n L\otimes A
            \end{tikzcd}
        \end{equation*}
        which once again is equivalent to the following not being exact
        \begin{equation*}
            \begin{tikzcd}
                0 \arrow[r,"{}"] & \bigoplus_1^n K \arrow[r,"{\oplus_1^n f}"] & \bigoplus_1^n L
            \end{tikzcd}
        \end{equation*}
        where \(\bigoplus_1^n f\) is injective since \(f\) is. \qed
    \end{pb}
    

\end{document}