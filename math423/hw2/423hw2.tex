\documentclass[11pt]{article}
\usepackage{amsmath, amsfonts, amssymb,amsthm}
\usepackage[includeheadfoot]{geometry} % For page dimensions
\usepackage{fancyhdr}
\usepackage{enumerate} % For custom lists

\fancyhf{}
\lhead{Math 423hw1}
\rhead{Tighe McAsey - 37499480}
\pagestyle{fancy}

% Page dimensions
\geometry{a4paper, margin=1in}

\theoremstyle{definition}
\newtheorem{pb}{}

% Commands:

\newcommand{\set}[1]{\{#1\}}
\newcommand{\abs}[1]{\lvert#1\rvert}
\newcommand{\norm}[1]{\lvert\lvert#1\rvert\rvert}
\newcommand{\gen}[1]{\left\langle #1 \right\rangle}
\newcommand{\tand}{\text{ and }}
\newcommand{\tor}{\text{ or }}
\newcommand{\falg}{F^{\text{alg}}}
\newcommand{\gal}{\text{Gal}}
\newcommand{\floor}[1]{\left\lfloor #1 \right\rfloor}
\newcommand{\spec}{\text{Spec}}
\newcommand{\nil}{\text{Nil}}
\newcommand{\jac}{\text{Jac}}
\newcommand{\ho}{\text{Hom}}

\begin{document}
    \begin{pb}
        It suffices to show that \(\spec(A)\) satisfies the finite intersection property (FIP) for closed sets, since if given any collection of closed sets \(\set{V_i}_{i \in I}\) we have
        \begin{align*}
            \bigcap_{i \in I}V_i = \emptyset \implies \exists V_{i_1},\hdots,V_{i_N}, \text{ such that } \bigcap_{j=1}^N V_{i_j} = \emptyset
        \end{align*}
        Then given any collection \(\set{U_i}_{i \in I}\) of open sets we have
        \begin{align*}
            \bigcup_{i \in I}U_i = \spec(A) &\iff \bigcap_{i \in I}U_i^c = \emptyset \implies \exists U_{i_1}^c,\hdots,U_{i_N}^c, \text{ such that } \bigcap_{j=1}^N U_{i_j}^c = \emptyset \\
            &\iff \left(\bigcup_{j=1}^N U_{i_j}\right)^c = \emptyset \iff \bigcup_{j=1}^N U_{i_j} = \spec(A)
        \end{align*}
        
        Now let \(\set{V_i}_{i \in I}\) be a collection of closed sets, such that \(\bigcap_{i \in I}V_i = \emptyset\), then by the characterization of Zariski closed sets, \(V_i = V(S_i)\), for some \(S_i \subset A\), and \(\emptyset = \bigcap_{i \in I}V_i = V(\bigcup_{i \in I}S_i)\). This suffices to show that \(\gen{\bigcup_{i \in I}S_i} = A\), since if \(\gen{\bigcup_{i \in I}S_i}\) were a proper ideal of \(A\), then there would exist some maximal ideal \(\mathfrak{m} \supset \gen{\bigcup_{i \in I}S_i}\), and since maximal ideals are prime we would have \(\mathfrak{m} \in V(\bigcup_{i \in I} S_i)\) which is impossible since it is empty. Since \(\gen{\bigcup_{i \in I}S_i} = A\), there exist \(\set{s_k}_{k=1}^n \subset \bigcup_{i \in I}S_i \tand \set{a_k}_{k=1}^n \subset A\), such that \(\sum_{k=1}^n a_ks_k = 1\), each \(s_k\) lies in some \(S_{i_k}\) which implies that \(\gen{\bigcup_{k=1}^N S_{i_k}} = A\), in particular
        \begin{align*}
            \emptyset = V(A) = V\left(\bigcup_{k=1}^N S_{i_k}\right) = \bigcap_{k=1}^N V_{i_k}
        \end{align*}
        This suffices to show that \(\spec(A)\) satisfies the FIP and is hence quasi-compact. \qed
    \end{pb}
    \begin{pb}
        First suppose that \(\nil(A)\) is prime, and let \(V(S_1), V(S_2)\) be Zariski closed sets, such that \(V(S_1) \cup V(S_2) = \spec(A)\), then since \(\nil(A)\) is prime it must be contained in one of the two closed sets, without loss of generality assume that \(\nil(A) \subset V(S_1)\), then
        \begin{align*}
            S_1 \subset \nil(A) = \bigcap_{\substack{P \subset A \\ P \text{ is a prime Ideal}}} P
        \end{align*}
        Implying that \(P \in V(S_1)\) for all prime ideals \(P \subset A\), but this is equivalent to \(V(S_1) = \spec(A)\), since \(V(S_1),V(S_2)\) were arbitrary this suffices to show that \(\spec(A)\) is irreducible.

        I will prove the converse using the contrapositive. Assume that \(\nil(A)\) is not prime, then there are \(x,y \in A\), such that \(x,y \not \in \text{Nil}(A)\), and \(xy \in \nil(A)\). It follows that
        \begin{align*}
            V((x))\cup V((y)) = V((xy)) \supset V(\nil(A)) = \spec(A)
        \end{align*}
        where \(V(\nil(A)) = \spec(A)\) is proven in the previous part of the problem. So it will suffice to show that \(V((x)), V((y)) \subsetneq \spec(A)\) to conclude that \(\spec(A)\) is irreducible. Since \[x,y \not \in \nil(A) = \bigcap_{\substack{P \subset A \\ P \text{ is a prime Ideal}}} P\]
        there are prime ideals \(x \not \in P_x, y \not \in P_y\), so that \(P_x \not \in V((x)), P_y \not \in V((y))\) hence neither can be all of \(\spec(A)\). \qed
    \end{pb}
    \begin{pb}
        \textbf{Lemma.} \emph{\(M\) is finitely generated implies M satisfies the aescending chain condition (ACC)}. Assume that \(M = \gen{x_1,\hdots,x_n}\), then for any chain of submodules, \((N_i)_I\) we have \(\bigcup_I N_i = M\) implies that \(N_j = M\) for some \(j \in I\) (and hence all \(i \geq j\)).

        \textbf{Proof of Lemma.} Since \(\bigcup_I N_i = M\), for each \(k = 1,\hdots,n\), there is some \(N_{i_k}\), such that \(x_k \in N_{i_k}\), since this is a chain of submodules it is totally ordered, implying that there is some \(j \in \set{i_1,\hdots,i_n}\), such that \(N_{i_k} \subset N_j\) for all \(k \in \set{1,\hdots,n}\), hence \(M = \gen{x_1,\hdots,x_n} \subset N_j\). \qed

        \textbf{(a)}  Let \(M \neq 0\) be a finitely generated \(A\)-module. We can use the ACC proven in the lemma to apply Zorn's lemma. Consider the set \(X := \set{N \subsetneq M \mid N \text{ is a submodule}}\), ordered by inclusion. \(X\) contains \(0\), hence is nonempty. Let \((N_i)_{i\in I}\) be a chain in \(X\), then \(\bigcup_{i \in I}N_i \neq M\) by the ACC, to see that \(\bigcup_{i \in I}N_i\) is a submodule, let \(a,b \in A\), \(n_1, n_2 \in \bigcup_{i \in I}N_i\). Then \(n_1 \in N_{i_1}, n_2 \in N_{i_2}\), and since it is a chain we may assume without loss of generality \(N_{i_2} \subset N_{i_1}\) it follows that \(an_1 + bn_2 \in N_{i_1} \subset \bigcup_{i \in I}N_{i}\), thus proving that \(\bigcup_{i\in I}N_i \in X\) is an upper bound for the chain. This satisfies the conditions for Zorn's lemma, so there exists some maximal element \(N \in X\), so \(N \subsetneq M\) is a proper submodule which is not contained in any other proper submodules. \qed

        \textbf{(b)} Suppose for contradiction that \(N \subsetneq \mathbb{Q}\) is maximal, then by the correspondance theorem \(\mathbb{Q}/N\) has no proper submodules. Hence for any \(0 \neq x \in \mathbb{Q}/N\) (there is always such an \(x\) since \(N\) is a proper submodule), it must be the case that \(\gen{x} = \mathbb{Q}/N\), since \(\mathbb{Q}/N\) is generated by a single element as a \(\mathbb{Z}\) module, it must be a homomorphic image of \(\mathbb{Z}\), furthermore since it is generated by any of its elements as a \(\mathbb{Z}\) module, it must be a finite cyclic group- in other words \(\mathbb{Q}/N \cong \mathbb{Z}/(p)\) for some prime \(p\). Hence by the first isomorphism theorem, we have a surjective \(\mathbb{Z}\) module homomorphism \(\varphi: \mathbb{Q} \to \mathbb{Z}/(p)\), with \(\ker \varphi = N\). I claim that \(\varphi\) is the zero map, contradicting that it is surjective, as proof, let \(x \in \mathbb{Q}\), then \(\frac{x}{p} \in \mathbb{Q}\), then \[0 + (p) = p\varphi(\frac{x}{p}) + (p) = \varphi(p\frac{x}{p}) + (p) = \varphi(x) + (p)\]
        Since \(x\) was arbitrary this shows that \(\varphi(x) = 0\) for all \(x \in \mathbb{Q}\) \qed
    \end{pb}
    \begin{pb}
        \textbf{Prop 2.4.} Let \(A = \mathbb{R}\), \(M = \oplus_{0}^\infty \mathbb{R}\), and \(\phi: (x_0,x_1,\hdots) \mapsto (0,x_0,x_1, \hdots)\). Now let \(n \in \mathbb{N}\), and \(a_1, \hdots, a_n \in \mathbb{R}\), since these are arbitrary, it will suffice to show that there is some \(\mathbf{x} \in \oplus_{0}^\infty \mathbb{R}\), such that
        \begin{align*}
            \mathbf{y} = \phi^n(\mathbf{x}) + a_1\phi^{n-1}(\mathbf{x}) + \cdots + a_{n-1}\phi(\mathbf{x}) + a_n\mathbf{x} \neq 0
        \end{align*}
        Choosing \(\mathbf{x} = (1,0,\hdots)\) we get
        \begin{align*}
            \mathbf{y}_n = \mathbf{x}_0 + \sum_{i=1}^n a_i\mathbf{x}_i = 1 \implies \mathbf{y} \neq 0
        \end{align*}

        \textbf{Cor 2.5.} Consider \(A = \mathbb{Z}\), \(M = \mathbb{Q}\), then \((2)\mathbb{Q} = \mathbb{Q}\), since for any \(x \in \mathbb{Q}\), we have \(\frac{x}{2} \in \mathbb{Q}\). However for any \(k \in \mathbb{Z}\), we have \(0 \neq 1 + 2k \in \mathbb{Q}\), it follows that since \(\mathbb{Q}\) is a field, for any \(x \in \mathbb{Q} \setminus \set{0}\) we have \((1+2k)x \neq 0\), it follows that \((1 + 2k)\mathbb{Q} \neq 0\) for any \(k \in \mathbb{Z}\).

        \textbf{Prop 2.6.} Consider the local ring \((A_2,\mathfrak{m})\), where \(A_2 = \set{\frac{m}{n} \mid m,n \in \mathbb{Z}, 2 \nmid n}\) and \(\mathfrak{m} = \set{\frac{m}{n} \in A_2 \mid 2 \vert m}\). We can consider \(\mathbb{Q}\) as an \(A_2\) module, i.e. \(A = A_2, M = \mathbb{Q}\). 
        Then \(\jac(A)M = \mathfrak{m}M = M\), since for any \(x \in \mathbb{Q}\), we have \(\frac{x}{2} \in \mathbb{Q}\) and \(2 \in \mathfrak{m}\). However, \(\mathbb{Q} \neq 0\).
    \end{pb}
    \begin{pb}
        \begin{align}
            M' \overset{\mu}{\longrightarrow} &M \overset{\nu}{\longrightarrow} M'' \to 0 \\
            0 \to \ho(M'',N) \overset{\nu^*}{\longrightarrow}&\ho(M,N)\overset{\mu^*}{\longrightarrow}\ho(M',N) \\
            0 \to N' \overset{\mu}{\longrightarrow} &N \overset{\nu}{\longrightarrow} N'' \\
            0 \to \ho(M,N') \overset{\mu_*}{\longrightarrow} &\ho(M,N) \overset{\nu_*}{\longrightarrow} \ho(M,N'')
        \end{align}

        \textbf{(1) is exact \(\mathbf{\implies}\) (2) is exact for all \(A\)-Modules \(N\)}: Let \(N\) be an \(A\)-module, \(\nu^*f = 0 \iff f \circ \nu = 0\), but since \(\nu\) is surjective by exactness of (1), this implies that \(f = 0\) and hence \(\nu^*\) is injective. Now let \(f \in \text{Im}\nu^*\), then \(f = g\nu\), so that \(\mu^*f = g\nu\mu  = g \circ 0 = 0\), since \(\ker \nu = \text{Im} \mu\) from exactness of (1), this suffices to show that \(\text{Im} \nu^* \subset \ker \mu^*\). Now let \(f \in \ker \mu^*\), then \(f \mu = 0\), so \(\text{Im}\mu = \ker \nu \subset \ker f\), this implies by the first isomorphism theorem that \(f\) factors through \(\pi: M \to M/\ker\nu\), i.e. \(\exists h : M/\ker\nu \to N\), such that \(f = h \pi\). By the first isomorphism theorem, there is an isomorphism \(s: M/\ker \nu \to M''\), such that \(s \pi = \nu\), equivalently \(\pi = s^{-1}\nu\), this implies that \(f = hs^{-1}\nu = \nu^*(hs^{-1})\in \text{Im}\nu^*\), hence \(\ker \mu^* \subset \text{Im}\nu^*\), so that they are in fact equal, and exactness is proven. \qed

        \textbf{(2) is exact for all \(A\)-Modules \(N\) \(\mathbf{\implies}\) (1) is exact}:
        To show that \(\nu\) is surjective, let \(N = M''/\nu(M)\), we have the quotient map \(\pi: M'' \to M''/\nu(M)\), \(\pi \in \ho(M'',N)\), it is immediate that \(\nu^*\pi = 0\), which by exactness of (2) means that \(\pi = 0\) implying that \(M'' = \nu(M)\), so that \(\nu\) is surjective. To show that \(\text{Im}\mu \subset \ker \nu\), let \(N = M''\), then \(1_{M''} \in \ho(M'',N)\), by exactness of (2) we have \(0 = \mu^*(\nu^*1_{M''}) = 1_{M''}\nu \mu\), then since \(1_{M''}\) is an isomorphism, this implies that \(\nu\mu(M') = 0\), and hence \(\text{Im}\mu \subset \ker \nu\). To show the opposite inclusion, take \(N = M/\mu(M')\), then we have the quotient map \(\pi: M \to M/\mu(M')\), \(\pi \in \ho(M,N)\) and \(\pi \in \ker \mu^*\), hence \(\pi \in \text{Im}\nu^*\) by exactness of (2), so for some \(f: M'' \to N\), \(\pi = \nu^*f = f\nu\). For any \(m \in M\), \(\nu(m) = 0 \implies f\nu(m) = f(0) = 0\), hence \(\ker \nu \subset \ker f\nu = \text{Im}\mu\), this suffices to show exactness of (1). \qed

        \textbf{(3) is exact \(\mathbf{\implies}\) (4) is exact for all \(A\)-Modules \(M\)}: Let \(M\) be an arbitrary \(A\) module.
        To see that \(\mu_*\) is injective, let \(f,g \in \ho(M,N')\) and suppose that \(\mu_*f = \mu_*g\), then for any \(m \in M\), \(\mu f(m) = \mu g(m)\), by exactness of (3), \(\mu\) is injective so that \(f(m) = g(m)\) but then since \(m\) was arbitrary \(f = g\) proving injectivity of \(\mu_*\). Suppose that \(f \in \text{Im}\mu_*\), then \(f = \mu_*g\), \(g \in \ho(M,N')\). It follows that \(\nu_*f = \nu_* \mu_* g = \nu\mu g = 0 \circ g = 0\), so that \(\text{Im}\mu_* \subset \ker \nu_*\). Now let \(f \in \ker \nu_*\), then \(f(M) \subset \ker \nu = \mu(N')\) by exactness of (3), futhermore we know that \(\mu\) is injective so in particular (taking \(\mu'\) to be \(\mu\) with restricted codomain) \(\mu': N' \to \mu(N')\) is inverible, then \(f = \mu \mu'^{-1}f = \mu_*(\mu'^{-1}f)\), so that \(f \in \text{Im}\mu_*\) which gives us \(\ker \nu_* \subset \text{Im}\mu_*\) so that (4) is exact.

        \textbf{(4) is exact for all \(A\)-Modules \(M\) \(\mathbf{\implies}\) (3) is exact}:
        We first show \(\ker \mu = 0\), let \(m \in N'\), and let \(M = A\), then consider the map \(f:a \to am\), \(f \in \ho(M,N')\), we have that \(\mu_*f = 0 \iff \mu_*f(1) = 0 \iff \mu(m) = 0\) then by exactness of (4), \(\mu_*f = 0 \iff f = 0 \iff f(1) = 0 \iff m = 0\), so taken together \(\mu(m) = 0 \iff m = 0\), since \(m\) was chosen arbitrarily this suffices to show that \(\mu\) is injective. Now let \(M = N'\), so that \(1_{N'} \in \ho(M,N')\), then exactness of (4) implies that \(\nu_*\mu_*1_{N'} = \nu\mu = 0\), hence \(\text{Im}\mu \subset \ker \nu\). If \(\mu\) is surjective, then \(\ker \nu \subset \text{Im} \mu\) is trivial, so assume not. Let \(m \in N \setminus \mu(N')\), then let \(M = A\), then the map \(f: a \mapsto am\) is such that \(f \in \ho(M,N)\). Then \(m \in f(M) \setminus \mu(N')\) implies that \(f(M) \not \subset \mu(N')\), so that \(f \not \in \text{Im}(\mu_*)\), and hence by exactness of (4), \(f \not \in \ker \nu_*\), it follows that \(\nu_*f \neq 0\), and since \(\nu_*f(a) = a\nu f(1)\), for any \(a \in A\) this implies that \(0 \neq \nu^*f(1) = \nu(m)\), hence by choice of \(m\), we have \(m \not \in \text{Im} \mu \implies m \not \in \ker \nu\), contraposing gives the desired result \(\ker \nu \subset \text{Im}\mu\) which suffices to show that (3) is exact. \qed
    \end{pb}
    \begin{pb}
        In this problem I will denote \(e_\ell \in A^k\) to have \(\ell\)-th coordinate \(1\), and all other coordinates \(0\).

        \textbf{(a)} Suppose for contradiction that \(n > m\), and \(f: A^m \to A^n\) is surjective, then consider the module homomorphism \[\pi: A^n \to A^m, \begin{cases}
            e_i \mapsto e_i & i \leq m \\
            e_i \mapsto 0 & m < i \leq n
        \end{cases}\]
        It is immediate that \(\pi \circ f: A^m \to A^m\) is surjective, hence by the second corollary in Nakayama's lemma it must be injective, but this is a contradiction, since by surjectivity of \(f\), for some \(0 \neq x \in A^m\), we have \(f(x) = e_n\), so that \(\pi \circ f (x) = 0\), implying \(\ker (\pi \circ f) \neq 0\). \qed

        \textbf{(b)} For the sake of contradiction, assume that \(m > n\), and \(f:A^m \to A^n\) is injective. Denote the inclusion map \(\iota : A^n \hookrightarrow A^m\), so that \(\iota \circ f: A^m \to A^m\) is injective, to simplify the notation, define \(\varphi = \iota \circ f\). Since \(\varphi(A^m) \subset \iota(A^n)\), \(m > n\) implies that \(a e_m \not \in \varphi(A^m)\) for any \(a \in A \setminus \set{0}\). We may apply proposition 2.4 from the text, which furnishes \(a_1, \hdots a_m\), so that for any \(x \in A^m\)
        \begin{align*}
            \varphi^m(x) + a_1 \varphi^{m-1}(x) + \cdots + a_{m-1}\varphi(x) + a_m x = 0
        \end{align*}
        In particular, this applies for \(e_m\), after applying linearity of \(\varphi\), this implies that
        \begin{align*}
            \varphi(\varphi^{m-1}(e_m) + a_1 \varphi^{m-2}(e_m) + \cdots + a_{m-1}e_m) = -a_m e_m
        \end{align*}
        Hence \(-a_m e_m \in \varphi(A^m)\) implies that \(a_m = 0\). Hence we can conclude that for all \(x \in A^m\),
        \begin{align}
            \varphi(\varphi^{m-1}(x) + a_1 \varphi^{m-2}(x) + \cdots + a_{m-1}x) = 0
        \end{align}
        Since both sides of the equation in (5) are in \(\varphi(A^m)\), injectivity of \(\varphi\) allows us to apply \(\varphi\vert_{\varphi(A^m)}^{-1}\) to either side of the equation, implying that for any \(x \in A^m\)
        \begin{align*}
            \varphi^{m-1}(x) + a_1 \varphi^{m-2}(x) + \cdots + a_{m-1}x = 0
        \end{align*}
        Applying this argument recursively, we can conclude that \(a_{m-1}, \hdots, a_2 = 0\), eventually reaching the desired result
        \begin{align*}
            \varphi(e_m) + a_1e_m = 0 \implies a_1 = 0 \implies \varphi(e_m) = 0
        \end{align*}
        But this implies that \(e_m \in \ker \varphi\), so that \(\varphi\) is not injective, which is a contradiction. \qed
    \end{pb}
\end{document}