\documentclass[10.5pt]{article}
\usepackage{amsmath, amsfonts, amssymb,amsthm}
\usepackage[includeheadfoot]{geometry} % For page dimensions
\usepackage{fancyhdr}
\usepackage{enumerate} % For custom lists

\fancyhf{}
\lhead{Math 425hw1}
\rhead{Tighe McAsey - 37499480}
\pagestyle{fancy}

% Page dimensions
\geometry{a4paper, margin=1in}

\theoremstyle{definition}
\newtheorem{pb}{}

% Commands:

\newcommand{\set}[1]{\{#1\}}
\newcommand{\abs}[1]{\lvert#1\rvert}
\newcommand{\norm}[1]{\lvert\lvert#1\rvert\rvert}
\newcommand{\tand}{\text{ and }}
\newcommand{\tor}{\text{ or }}


\begin{document}
\begin{pb}
    We can consider the parameterized line defined by \(N \tand \begin{pmatrix}
        x&y&z
    \end{pmatrix}\), \( \; l(t) = \begin{pmatrix}
        tx&ty&1 + t(z-1)
    \end{pmatrix}\) Then \(t(z-1) = -1\) when \(t = \frac{1}{1-z}\). Plugging in this value for \(t\) defines \(\varphi\).
    \[\varphi: \begin{pmatrix}
        x&y&z
    \end{pmatrix} \mapsto \begin{pmatrix}
        \frac{x}{1-z}& \frac{y}{1-z}
    \end{pmatrix}\]

    We define \(\tilde{\varphi}\) similarly. In this case we get
    \[\tilde{\varphi} \begin{pmatrix}
        x&y&z
    \end{pmatrix} \mapsto \begin{pmatrix}
        \frac{x}{1+z}& \frac{y}{1+z}
    \end{pmatrix}\]

    Similar to before, we find \(\varphi^{-1}\) by parameterizing the line through \(\begin{pmatrix}
        u,v,0
    \end{pmatrix}\) and \(N\), to compute the inverse we check where \(\ell(t):= \begin{pmatrix} tu, tv, 1 - t\end{pmatrix}\) intersects the sphere.
    \begin{align*}
        (tu)^2 + (tv)^2 + (1-t)^2 = 1 \iff t(t(u^2 + v^2 + 1) - 2) = 0 \iff t = 0 \tor t = \frac{2}{u^2 + v^2 + 1}
    \end{align*}
    We can ignore the case of \(t=0\), since this corresponds to \(N\), plugging in this \(t\) gives the inverse.
    \begin{align*}
        \varphi^{-1}: (u,v) \mapsto \begin{pmatrix}
            \frac{2u}{u^2+v^2+1}&\frac{2v}{u^2+v^2+1}&\frac{u^2+v^2-1}{u^2+v^2+1}
        \end{pmatrix}
    \end{align*}
    Here is the coordinate change:
    \[\tilde{\varphi}\circ \varphi^{-1}\begin{pmatrix}
        u&v
    \end{pmatrix} = \tilde{\varphi}\begin{pmatrix}
        \frac{2u}{u^2+v^2+1}&\frac{2v}{u^2+v^2+1}&\frac{u^2+v^2-1}{u^2+v^2+1}
    \end{pmatrix} = \begin{pmatrix}
       \frac{u}{u^2+v^2}&\frac{v}{u^2+v^2}
    \end{pmatrix}\]
\end{pb}
\begin{pb}
    \begin{align*}
        \varphi_0^{-1}:(x_1,\hdots,x_n) \mapsto (1,x_1,\hdots,x_n)
    \end{align*}
    Then we have:
    \begin{align*}
        &\varphi_0 \circ \varphi_0^{-1}:(x_1,\hdots,x_n)\mapsto \varphi_0 (1,x_1,\hdots,x_n) = (x_1,\hdots,x_n) \\
        &\varphi_0^{-1} \circ \varphi_0:(x_0,x_1,\hdots,x_n) \mapsto \varphi_0^{-1} (\frac{x_1}{x_0},\hdots,\frac{x_n}{x_0}) =
         (1,\frac{x_1}{x_0},\hdots,\frac{x_n}{x_0}) \sim (x_0,x_1,\hdots,x_n)
    \end{align*}
    Here is the change of coordinates:
    \begin{align*}
        \varphi_1\circ\varphi_0^{-1}: (x_i)_{i=1}^n \mapsto (x_1^{-1},x_2/x_1, x_3/x_1, \hdots, x_n/x_1)
    \end{align*}
\end{pb}
\begin{pb}
    \textbf{(a)}
    Suppose for the sake of contradiction \(R_1\), \(R_2\) are products of elementary row operations such that
    \begin{align*}
        R_1 A = \begin{bmatrix}
            1& 0& x_1& x_3 \\
            0& 1& x_2& x_4
        \end{bmatrix} = \begin{bmatrix} \mathbf{v}_1 \\ \mathbf{v}_2 \end{bmatrix} \tand R_2 A = \begin{bmatrix}
            1& 0& y_1& y_3 \\
            0& 1& y_2& y_4
        \end{bmatrix} = \begin{bmatrix} \mathbf{u}_1 \\ \mathbf{u}_2 \end{bmatrix}
    \end{align*}
    Where \((x_1,x_2,x_3,x_4) \neq (y_1,y_2,y_3,y_4)\) (Note that these two row spaces are subsets of \(\text{rowsp} A\), but since they have rank 2 they are equal to eachother and \(\text{rowsp} A\)). 
    It follows that one of \(\mathbf{v}_1 - \mathbf{u}_1 \tor \mathbf{v}_2 - \mathbf{u}_2\) 
    is non-zero (WLOG \(\mathbf{w} := \mathbf{v}_1 - \mathbf{u}_1\)). But then since \(\mathbb{R}\mathbf{v}_1 + \mathbb{R}\mathbf{v}_2 = \mathbb{R}\mathbf{u}_1 + \mathbb{R}\mathbf{u}_2\) 
    it follows that \(\mathbf{w} = a\mathbf{v}_1 + b\mathbf{v}_2, \; a,b \in \mathbb{R}\). Since \(\mathbf{w}_i = 0, \; i \in \set{1,2}\), and 
    \((a\mathbf{v}_1 + b\mathbf{v}_2)_1 = a \tand (a\mathbf{v}_1 + b\mathbf{v}_2)_2 = b\) it follows that \(a = b = 0\) and hence \(\mathbf{w} = \mathbf{0}\), a contradiction. \newline
    
    \textbf{(b)} 
    I will denote \(A = (a_{i,j})\) for \(i \leq 2, j \leq 4\). Note that the determinant is a continuous function on \((a_{i,j}), \; i,j \leq 2\) 
    (it is a polynomial in these 4 variables and the sum and product of continuous functions is continuous).
    Furthermore, row operations on \(A\) are just row operations on \((a_{i,j}), \; i,j \leq 2\). It follows that if \(\det (a_{i,j}), \; i,j \leq 2\) is non-zero, continuity furnishes some \(\delta > 0\), such that
    if \(\abs{b_i - a_i} < \delta\) for each \(i\), then \(\abs{\det (b_{i,j}) - \det (a_{i,j})} < \abs{\det (a_{i,j})}\) for \(i,j \leq 2\), we are done once we apply the reverse triangle inequality
    \[\abs{\det (a_{i,j})} - \abs{\det (b_{i,j})} \leq \abs{\det (b_{i,j}) - \det (a_{i,j})} < \abs{\det (a_{i,j})} \implies 0 < \abs{\det (b_{i,j})}\]
    Now let \(B = \begin{bmatrix}
        \mathbf{b}_1 \\
        \mathbf{b}_2
    \end{bmatrix}\) be a matrix who's entries are within \(\delta\) of \(A\)'s entries. It follows that since \((b_{i,j}), \; i,j \leq 2\) is full rank
    we get that \(\begin{pmatrix}
        1 & 0
    \end{pmatrix} \tand \begin{pmatrix}
        0 & 1
    \end{pmatrix}\) are in the span of \(\mathbb{R} \mathbf{b}_1' + \mathbb{R} \mathbf{b}_2'\) where \(\mathbf{b}_i'\) is \(\mathbf{b}_i\) restricted to its first two coordinates.
    It follows that when we dont restrict to the first two coordinates,
    \(\begin{pmatrix}
        1 & 0 &y_1&y_2
    \end{pmatrix} \tand \begin{pmatrix}
        0 & 1 &y_1&y_2
    \end{pmatrix}\) for some \(y_i\) are in the rowspace of \(B\).

    \textbf{(c)} If \(A\) has full rank, then it has a \(2 \times 2\) submatrix of full rank, hence by the same procedure as part (b) can be brought
    to one of these forms by row operations since all \(2 \times 2\) submatrices are listed.

    \textbf{(d)} We can do the computation by using row operations to convert a matrix of form \(U_2\) to \(U_1\).
    \begin{align*}
        \varphi_1 \circ \varphi_2^{-1}:
            \begin{pmatrix}
                y_1 & y_2 & y_3 & y_4
            \end{pmatrix}
            \mapsto \begin{pmatrix}
                -y_1/y_2 & 1/y_2 & y_3 - \frac{y_1y_4}{y_2} & y_4/y_2
            \end{pmatrix}
    \end{align*}
\end{pb}
\end{document}