\documentclass[10.5pt]{article}
\usepackage{amsmath, amsfonts, amssymb,amsthm}
\usepackage{graphicx}
\usepackage[includeheadfoot]{geometry} % For page dimensions
\usepackage{fancyhdr}
\usepackage{enumerate} % For custom lists
\usepackage{xcolor}

\fancyhf{}
\lhead{Math 425hw9}
\rhead{Tighe McAsey - 37499480}
\pagestyle{fancy}

% Page dimensions
\geometry{a4paper, margin=1in}

\theoremstyle{definition}
\newtheorem{pb}{}

% Commands:

\newcommand{\set}[1]{\{#1\}}
\newcommand{\abs}[1]{\lvert#1\rvert}
\newcommand{\norm}[1]{\lvert\lvert#1\rvert\rvert}
\newcommand{\gen}[1]{\langle #1 \rangle}
\newcommand{\tand}{\text{ and }}
\newcommand{\tor}{\text{ or }}
\newcommand{\vp}{\varphi}
\newcommand{\R}{\text{Re}}
\newcommand{\I}{\text{Im}}
\newcommand{\parx}{\frac{\partial}{\partial x}}
\newcommand{\pary}{\frac{\partial}{\partial y}}
\newcommand{\parz}{\frac{\partial}{\partial z}}
\newcommand{\paru}{\frac{\partial}{\partial u}}
\newcommand{\parv}{\frac{\partial}{\partial v}}
\newcommand{\tth}{\tilde{\theta}}
\newcommand{\z}{\imath}
\newcommand{\hook}{\lrcorner\;}

\begin{document}
    \begin{pb}
        First let \(n\) denote the dimension of \(M\), and let \((U_\alpha,x_\alpha^i)_\alpha\) be a smooth atlas for \(M\), I will denote the coordinate maps as \(\phi_\alpha\) respectively. This induces a smooth atlas \((\pi^{-1}(U_\alpha),(x_\alpha^i,v_\alpha^i))\) for \(TM\), I will denote the coordinate maps as \(\varphi_\alpha\) respectively. Now take charts \((\pi^{-1}(U_\alpha),(x_\alpha^i,v_\alpha^i)), (\pi^{-1}(U_\beta),(x_\beta^i,v_\beta^i))\), (assume that they have non-empty intersection else there is nothing to check) the change of coordinates is given by
        \begin{align*}
            (x_\beta^i,v_\beta^i) = \vp_\beta \vp_\alpha^{-1}(x_\alpha^i,v_\alpha^i) = (\phi_\alpha \phi_\beta(x_\alpha^i), \frac{\partial x_\beta^i}{\partial x_\alpha^j} v_\alpha^j)
        \end{align*}
        We want to show that the determinant of the following block diagonal matrix (\(A\)) is positive:
        \begin{align*}
            A := \begin{bmatrix} 
                \left(\frac{\partial x_\beta^i}{\partial x_\alpha^j} \right)_{1 \leq i,j \leq n} &
                \left(\frac{\partial x_\beta^i}{\partial v_\alpha^j} \right)_{1 \leq i,j \leq n} \\
                \left(\frac{\partial v_\beta^i}{\partial x_\alpha^j} \right)_{1 \leq i,j \leq n} &
                \left(\frac{\partial v_\beta^i}{\partial v_\alpha^j} \right)_{1 \leq i,j \leq n}
            \end{bmatrix}
        \end{align*}
        By definition of the coordinate change on \(TM\), it is apparent that
        \[\det \left(\frac{\partial x_\beta^i}{\partial v_\alpha^j} \right)_{1 \leq i,j \leq n} = 0\]
        since \(\phi_\alpha \phi_\beta(x_\alpha^i)\) is independent of each \(v_\alpha^i\).
        It follows that
        \begin{align*}
            \det A = \det \left(\frac{\partial x_\beta^i}{\partial x_\alpha^j} \right)_{1 \leq i,j \leq n} \det \left(\frac{\partial v_\beta^i}{\partial v_\alpha^j} \right)_{1 \leq i,j \leq n}
        \end{align*}
        But then we can read off of the change of coordinates formula for \(TM\) that
        \(\frac{\partial v_\beta^i}{\partial v_\alpha^j} = \frac{\partial x_\beta^i}{\partial x_\alpha^j}\), so that
        \begin{align*}
            \det A = \left(\det \left(\frac{\partial x_\beta^i}{\partial x_\alpha^j} \right)_{1 \leq i,j \leq n}\right)^2 > 0
        \end{align*}
        We can say \(> 0\) here, since the change of coordinates is smooth and invertible with smooth inverse the inverse function theorem tells us its Jacobian is invertible.
    \end{pb}
    \begin{pb}
        First we give an orientation on \(M\), consider the coordinate charts
        \begin{align*}
            U_1 \times U_2, \quad \tilde{U_1} \times U_2, \quad U_1 \times \tilde U_2, \quad
            \tilde{U_1} \times \tilde{U_2}
        \end{align*}
        Given by the product of the standard \(\theta,\tilde{\theta}\) coordinates on \(S^1\). Here since \(w^2 + x^2 = 1 = y^2 + z^2\), we can associate to any point
        \[(w,x,y,z) = (\cos\theta_1,\sin\theta_1,\cos\theta_2\sin\theta_2) \longleftrightarrow (e^{i\theta_1},e^{i\theta_2})\]
        Allowing us to define the charts in the standard way on \(S^1\). Hence if we can show that the coordinate change between \(U\) and \(\tilde{U}\) on \(S^1\) has jacobian with positive determinant, we will be done since each of the coordinate changes are products of the determinant of the identity and the determinant of this Jacobian. This is straightforward, since \(\tilde{\theta} = \theta - 2\pi\), we have \(\frac{\partial \tilde{\theta}}{\partial \theta} = 1\) (the reverse coordinate change is the same). Thus this is an orientation.

        Let \(U\) denote that chart
        \((0,2\pi)\times(0,2\pi)\), with coordinates \((\theta_1,\theta_2)\).
        It follows that in \(U\), we have 
        \begin{align*}
            &dw = \frac{\partial w}{\partial \theta_1} d\theta_1 + \frac{\partial w}{\partial \theta_2} d\theta_2 = -\sin\theta_1 d\theta_1\\
            &dy = \frac{\partial y}{\partial \theta_1}d\theta_1 + \frac{\partial y}{\partial \theta_2}d\theta_2 = -\sin\theta_2 d\theta_2
        \end{align*}
        Using this, we compute
        \(\omega = \sin^2\theta_1\sin^2\theta_2 d\theta_1 \wedge \theta_2\), so that
        \begin{align*}
            \int_{T^2} \omega &= \int_{T^2 \setminus S^1 \cup S^1} \omega = \int_U \omega
            = \int_0^{2\pi}\int_0^{2\pi}\sin^2\theta_1\sin^2\theta_2 d\theta_1d\theta_2 \\
            &= \int_0^{2\pi}\pi\sin^2\theta_2d\theta_2 = \pi^2
        \end{align*}
    \end{pb}
    \begin{pb}
        We first compute \(d\text{Vol}_g\). Equip \(S^2\) with the polar coordinate chart given by
        \begin{align*}
            U = ((0,2\pi) \times (0,\pi),\theta,\phi)
        \end{align*}
        We only need to check for equality of \(g\) and \(\omega\) on this chart, since both are smooth, equality on \(U\) also implies equality on \(\overline{U} = S^2\).
        On \(U\) we have \((x,y,z) \longleftrightarrow (\cos\theta\sin\phi,\sin\theta\sin\phi,\cos\phi)\), and \newline
        \(\z: (\theta,\phi) \mapsto (\cos\theta\sin\phi,\sin\theta\sin\phi,\cos\phi)\), so that
        \begin{align*}
            g = \z^*g_{\text{EUC}} = AA^T = \begin{bmatrix} \sin^2\phi & 0 \\ 0 & 1 \end{bmatrix}
        \end{align*}
        Where in this case we have
        \begin{align*}
            A =
            \begin{bmatrix}  
                \frac{\partial \z^x}{\partial \theta} & \frac{\partial \z^y}{\partial \theta} & \frac{\partial \z^z}{\partial \theta} \\
                \frac{\partial \z^x}{\partial \phi} & \frac{\partial \z^y}{\partial \phi} & \frac{\partial \z^z}{\partial \phi}
            \end{bmatrix}
            =
             \begin{bmatrix}
                -\sin\theta\sin\phi & \cos\theta\sin\phi & 0\\
                \cos\theta\cos\phi & \sin\theta\cos\phi & -\sin\phi
            \end{bmatrix}
        \end{align*}
        This allows us to compute
        \begin{align*}
            d\text{Vol}_g \overset{\text{def.}}{=} \sqrt{\det \begin{bmatrix} \sin^2\phi & 0 \\ 0 & 1 \end{bmatrix}}d\theta \wedge d\phi = \sin\phi d\theta \wedge d\phi
        \end{align*}
        Now we compute \(\omega\) on \(U\) to check for equality, to simplify the computations, we first compute
        \begin{align*}
            &\z^* dx = d(x \circ \z) = d\cos\theta\sin\phi = -\sin\theta\sin\phi d\theta + \cos\theta \cos\phi d\phi\\
            &\z^* dy = d(y \circ \z) = d\sin\theta\sin\phi = \cos\theta\sin\phi d\theta + \cos\theta \cos\phi d\phi \\
            &\z^* dz = d(z \circ \z) = d\cos\phi = -\sin\phi d\phi
        \end{align*}
        So that,
        \begin{align*}
            \omega &= \z^*(xdy\wedge dz + ydz\wedge dx + zdx \wedge dy)
            = x\z^*dy\wedge \z^*dz + y\z^*dz\wedge \z^*dx + z\z^*dx \wedge \z^*dy \\
            &= \cos\theta\sin\phi(\cos\theta\sin\phi d\theta + \cos\theta \cos\phi d\phi) \wedge (-\sin\phi d\phi) \\
            &+ \sin\theta\sin\phi (-\sin\phi d\phi) \wedge (-\sin\theta\sin\phi d\theta + \cos\theta \cos\phi d\phi) \\
            &+ \cos\phi (-\sin\theta\sin\phi d\theta + \cos\theta \cos\phi d\phi) \wedge (\cos\theta\sin\phi d\theta + \cos\theta \cos\phi d\phi) \\
            &= -\cos^2\theta\sin^3\phi d\theta \wedge d\phi
            - \sin^2\theta\sin^3\phi d\theta \wedge d\phi - \cos^2\phi\sin^2\theta\sin\phi d\theta \wedge d\phi - \cos^2\phi \cos^2\theta \sin\phi d\theta \wedge d\phi \\
            &= (-\sin^3\phi - \cos^2\phi\sin\phi)d\theta \wedge d\phi = -\sin\phi d\theta \wedge d\phi
        \end{align*}

        \textcolor{red}{\textbf{ASK ABOUT ORIENTATION}}

        \textbf{(b)} We may take \(\z: B^3 \to \mathbb{R}^3, \; (r,\theta,\phi) \mapsto (r\cos\theta\sin\phi,r\sin\theta\sin\phi,r\cos\phi)\), then on \(U \subset S^2 = \partial B^3\) we have that \(\iota\) restricts to the same map as above.
        To simplify the computations, we first compute
        \begin{align*}
            &d(x \circ \z) = \cos\theta\sin\phi dr - r\sin\theta\sin\phi d\theta + r\cos\theta\cos\phi d\phi\\
            &d(y \circ \z) = \sin\theta\sin\phi dr + r\cos\theta\sin\phi d\theta + r\sin\theta \cos\phi d\phi\\
            &d(z \circ \z) = \cos\phi dr - r\sin\phi d\phi\\
            \implies &d(x \circ \z) \wedge d(y \circ \z) \wedge d(z \circ \z) = 
            -r^2(\cos^2\theta \sin^3\phi + \sin^2\theta\sin^3\phi + \sin^2\theta\cos^2\phi\sin\phi + \cos^2\theta\cos^2\phi\sin\phi) dr \wedge d\theta \wedge d\phi \\
            &\quad \quad \quad \quad \quad \quad \quad \quad \quad \quad \quad \;\; =
            -r^2\sin\phi dr \wedge d\theta \wedge d\phi
        \end{align*}

        \begin{align*}
            \int_{S^2} \omega &= \int_{\partial B^3} \z^* \tilde{\omega} \overset{\text{Stoke's Thm}}{=} \int_{B^3} d\z^*{\tilde{\omega}} = \int_{B^3} \z^* d \tilde{\omega} = \int_{B^3} \z^*d(x dy \wedge dz + y dz \wedge dx + z dx \wedge dy) \\
            &= \int_{B^3} \z^* (dx \wedge dy \wedge dz + dy \wedge dz \wedge dx + dz \wedge dx \wedge dy) = \int_{B^3} \z^*(3dx \wedge dy \wedge dz) \\
            &= \int_{B^3} 3 \z^*dx \wedge \z^*dy \wedge \z^* dz 
            = \int_{B^3} d(x \circ \z) \wedge d(y \circ \z) \wedge d(z \circ \z) \\
            &= \int_{B^3} -3r^2\sin\phi dr \wedge d\theta \wedge d\phi = \int_0^{\pi}\int_0^{2\pi}\int_0^1 -3r^2\sin\phi dr d\theta d\phi \\
            &= \int_0^{\pi}\int_0^{2\pi}-\sin\phi d\theta d\phi = \int_0^\pi-2\pi \sin\phi
            = -2\pi \left(\cos\pi - \cos0\right) = 4\pi
        \end{align*}

        \textbf{(c)} Assume that such an \(\alpha\) exists, then
        \begin{align*}
            4\pi = \int_{S^2} \omega = \int_{S^2}d \alpha = \int_{\partial B^3}d \alpha \overset{\text{Stoke's Thm}}{=} \int_{B^3} d^2 \alpha = \int_{B^3} 0 = 0
        \end{align*}
        Contradiction.
    \end{pb}
\end{document}