\documentclass[10.5pt]{article}
\usepackage{amsmath, amsfonts, amssymb,amsthm}
\usepackage{graphicx}
\usepackage[includeheadfoot]{geometry} % For page dimensions
\usepackage{fancyhdr}
\usepackage{enumerate} % For custom lists
\usepackage{xcolor}

\fancyhf{}
\lhead{Math 425hw9}
\rhead{Tighe McAsey - 37499480}
\pagestyle{fancy}

% Page dimensions
\geometry{a4paper, margin=1in}

\theoremstyle{definition}
\newtheorem{pb}{}

% Commands:

\newcommand{\set}[1]{\{#1\}}
\newcommand{\abs}[1]{\lvert#1\rvert}
\newcommand{\norm}[1]{\lvert\lvert#1\rvert\rvert}
\newcommand{\gen}[1]{\langle #1 \rangle}
\newcommand{\tand}{\text{ and }}
\newcommand{\tor}{\text{ or }}
\newcommand{\vp}{\varphi}
\newcommand{\R}{\text{Re}}
\newcommand{\I}{\text{Im}}
\newcommand{\parx}{\frac{\partial}{\partial x}}
\newcommand{\pary}{\frac{\partial}{\partial y}}
\newcommand{\parz}{\frac{\partial}{\partial z}}
\newcommand{\paru}{\frac{\partial}{\partial u}}
\newcommand{\parv}{\frac{\partial}{\partial v}}
\newcommand{\tth}{\tilde{\theta}}
\newcommand{\z}{\imath}
\newcommand{\hook}{\lrcorner\;}

\begin{document}
    \begin{pb}
        First let \(n\) denote the dimension of \(M\), and let \((U_\alpha,x_\alpha^i)_\alpha\) be a smooth atlas for \(M\), I will denote the coordinate maps as \(\phi_\alpha\) respectively. This induces a smooth atlas \((\pi^{-1}(U_\alpha),(x_\alpha^i,v_\alpha^i))\) for \(TM\), I will denote the coordinate maps as \(\varphi_\alpha\) respectively. Now take charts \((\pi^{-1}(U_\alpha),(x_\alpha^i,v_\alpha^i)), (\pi^{-1}(U_\beta),(x_\beta^i,v_\beta^i))\), (assume that they have non-empty intersection else there is nothing to check) the change of coordinates is given by
        \begin{align*}
            (x_\beta^i,v_\beta^i) = \vp_\beta \vp_\alpha^{-1}(x_\alpha^i,v_\alpha^i) = (\phi_\alpha \phi_\beta(x_\alpha^i), \frac{\partial x_\beta^i}{\partial x_\alpha^j} v_\alpha^j)
        \end{align*}
        We want to show that the determinant of the following block diagonal matrix (\(A\)) is positive:
        \begin{align*}
            A := \begin{bmatrix} 
                \left(\frac{\partial x_\beta^i}{\partial x_\alpha^j} \right)_{1 \leq i,j \leq n} &
                \left(\frac{\partial x_\beta^i}{\partial v_\alpha^j} \right)_{1 \leq i,j \leq n} \\
                \left(\frac{\partial v_\beta^i}{\partial x_\alpha^j} \right)_{1 \leq i,j \leq n} &
                \left(\frac{\partial v_\beta^i}{\partial v_\alpha^j} \right)_{1 \leq i,j \leq n}
            \end{bmatrix}
        \end{align*}
        By definition of the coordinate change on \(TM\), it is apparent that
        \[\det \left(\frac{\partial x_\beta^i}{\partial v_\alpha^j} \right)_{1 \leq i,j \leq n} = 0\]
        since \(\phi_\alpha \phi_\beta(x_\alpha^i)\) is independent of each \(v_\alpha^i\).
        It follows that
        \begin{align*}
            \det A = \det \left(\frac{\partial x_\beta^i}{\partial x_\alpha^j} \right)_{1 \leq i,j \leq n} \det \left(\frac{\partial v_\beta^i}{\partial v_\alpha^j} \right)_{1 \leq i,j \leq n}
        \end{align*}
        But then we can read off of the change of coordinates formula for \(TM\) that
        \(\frac{\partial v_\beta^i}{\partial v_\alpha^j} = \frac{\partial x_\beta^i}{\partial x_\alpha^j}\), so that
        \begin{align*}
            \det A = \left(\det \left(\frac{\partial x_\beta^i}{\partial x_\alpha^j} \right)_{1 \leq i,j \leq n}\right)^2 > 0
        \end{align*}
        We can say \(> 0\) here, since the change of coordinates is smooth and invertible with smooth inverse the inverse function theorem tells us its Jacobian is invertible.
    \end{pb}
    \begin{pb}
        First we give an orientation on \(M\), consider the coordinate charts
        \begin{align*}
            U_1 \times U_2, \quad \tilde{U_1} \times U_2, \quad U_1 \times \tilde U_2, \quad
            \tilde{U_1} \times \tilde{U_2}
        \end{align*}
        Given by the standard \(\theta,\tilde{\theta}\)

        We equip each copy of \(S^1\) with the standard charts, then let \(U\) denote that chart
        \((0,2\pi)\times(0,2\pi)\), with coordinates \((\theta_1,\theta_2)\). This is an 
        It follows that in \(U\), we have 
        \begin{align*}
            &dw = \frac{\partial w}{\partial \theta_1} d\theta_1 + \frac{\partial w}{\partial \theta_2} d\theta_2 = -\sin\theta_1 d\theta_1\\
            &dy = \frac{\partial y}{\partial \theta_1}d\theta_1 + \frac{\partial y}{\partial \theta_2}d\theta_2 = -\sin\theta_2 d\theta_2
        \end{align*}
        Using this, we compute
        \(\omega = \sin^2\theta_1\sin^2\theta_2 d\theta_1 \wedge \theta_2\), so that
        \begin{align*}
            \int_{T^2} \omega &= \int_{T^2 \setminus S^1 \cup S^1} \omega = \int_U \omega
            = \int_0^{2\pi}\int_0^{2\pi}\sin^2\theta_1\sin^2\theta_2 d\theta_1d\theta_2 \\
            &= \int_0^{2\pi}\pi\sin^2\theta_2d\theta_2 = \pi^2
        \end{align*}
        
    \end{pb}
\end{document}