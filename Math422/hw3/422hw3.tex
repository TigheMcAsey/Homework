\documentclass[11pt]{article}
\usepackage{amsmath, amsfonts, amssymb,amsthm}
\usepackage[includeheadfoot]{geometry} % For page dimensions
\usepackage{fancyhdr}
\usepackage{enumerate} % For custom lists

\fancyhf{}
\lhead{Math 422hw3}
\rhead{Tighe McAsey - 37499480}
\pagestyle{fancy}

% Page dimensions
\geometry{a4paper, margin=1in}

\theoremstyle{definition}
\newtheorem{pb}{}

% Commands:

\newcommand{\set}[1]{\{#1\}}
\newcommand{\abs}[1]{\lvert#1\rvert}
\newcommand{\norm}[1]{\lvert\lvert#1\rvert\rvert}
\newcommand{\tand}{\text{ and }}
\newcommand{\tor}{\text{ or }}

\begin{document}
    \begin{pb}
        From the notes if the roots of \(f\) are \(\set{\theta_i}_{i=1}^n\), then \[\text{Disc}(f) = (-1)^{\binom{n}{2}}\prod_1^n f'(\theta_i)\]
        In this case, the roots are \(\set{\zeta_n^i}_{i=1}^n\), so that
        \begin{align*}
            \text{Disc}(x^n-1) &= (-1)^{\binom{n}{2}}\prod_1^n n\zeta^{-i}_n = (-1)^{\binom{n}{2}}n^n\prod_1^n \zeta^{-i}_n \\
            &= (-1)^{\binom{n}{2}}n^n \overline{\prod_1^n \zeta^{i}_n} = (-1)^{\binom{n}{2}}n^n \overline{(-1)^n \prod_1^n -\zeta^{i}_n}
        \end{align*}
        Then we can recognize \(\prod_1^n -\zeta^{i}_n\) as the constant term of \(x^n -1\), since \(x^n - 1 = \prod_1^n(x-\zeta_n^i)\) implies that
        \(\prod_1^n -\zeta_n^i = -1\), hence \[(-1)^n\prod_1^n -\zeta^{i}_n = (-1)^n(-1) = (-1)^{n-1} = \overline{(-1)^{n-1}} = \overline{(-1)^n\prod_1^n -\zeta^{i}_n} = (-1)^n(-1)\]
        Substituting this in to the original expression, we get
        \begin{align*}
            \text{Disc}(x^n-1) = (-1)^{\binom{n}{2}}n^n \overline{(-1)^n \prod_1^n -\zeta^{i}_n} = (-1)^{\binom{n}{2} + n - 1}n^n
        \end{align*}
        As desired.
    \end{pb}
    \begin{pb}
        Denote \(f(t)\) as the polynomial in the question, then notice \(f(t) = (t+1)^3 - 5\). I claim that the splitting field of \(f(t)\) is equal to the splitting field of \(p(t) = t^3 - 5 = f(t-1)\). 
        As proof, if \(\alpha\) is a root of \(p(t)\), then \(1 + \alpha\) is a root of \(f(t)\), and if \(\beta\) is a root of \(f(t)\), then \(1-\beta\) is a root of \(p(t)\), hence any field containing the roots of
        one of the two polynomials contains the roots of the other. So it will suffice to find the splitting field of \(p(t)\), which is 
        \[\mathbb{Q}(5^{1/3},5^{1/3}\zeta_3,5^{1/3}\zeta_3^{-1}) = \mathbb{Q}(5^{1/3},\zeta_3)\]
    \end{pb}
    \begin{pb}
        We note that by Gauss' lemma, we can check for irreducibility in \(\mathbf{F}_p[X,Y][T]\). Assume \(f = gh\), then \(\deg_{X,Y}g + \deg_{X,Y}h = 1\), so we can assume wlog
        that \(\deg_{X,Y}g = 1, \tand \deg_{X,Y}h = 0\), then \(g = Xg_1(T) + Yg_2(T) + g_3(T)\), where \(g_i \in \mathbf{F}_p[T]\), then \(hg_2 = 1\) implies that \(h\) is a unit, so \(f\)
        is irreducible. To see that \([L:k]_s = p\), we write \(f(T) = p(T^p)\), where \(p(T) = T^p + XT + Y\) (seperable, and irreducible by irreducibility of \(f\)), so that each root of \(f\) has
        multiplicity \(p\), and hence \(p\) conjugates. Taking an algebraically closed field \(K \supset L\), and an embedding \(\sigma: \mathbf{F}_p(X,Y) \to K\), the number of ways to
        extend \(\sigma\) to \(k(\alpha)\) is the number of conjuates of \(\alpha\), which is equal to \(p\), hence by definition \([L:k]_s = p\).

        Consider the extension \(k(\alpha^p)\), then we have \(k(\alpha)/k(\alpha^p)/k\), then \(\alpha^p\) is a root of \(p(T)\), hence this is a degree \(p\)
        and therefore proper intermediate extension, since \(\min(\alpha^p,k)\) is seperable, this is also a seperable extension, equal to the seperable closure of \(k\) in \(k(\alpha)\)
        since it has degree equal to \([k(\alpha):k]_s\).
        To show this extension is unique, suppose for contradiction there exists some other intermediate extension \(k'\).
        Note that any proper intermediate extension must be such that \([k(\alpha):k'] = [k':k] = p\),
        but then \([k':k]\) must be purely inseperable, since if it were seperable then it would have to be equal to \(k_{\text{sep}} = k(\alpha^p)\), and \([k':k]_s \vert p\), implies
        that \(k'/k\) is purely inseperable. Consequently \(k(\alpha)/k'\) must be a seperable extension, let \(q(T) := \min(\alpha,k')\), then \(\deg(q) = p\) and \(q \vert f\), since
        \(f(\alpha) = 0\), and \(f(T) \in k'[T]\). Now, since \(f\) has \(p\) unique factors of multiplicity \(p\) and \(q\) has \(p\) distinct linear factors, it must be the case that 
        \(q^p = f\), from the binomial theorem we can see that \(q\) must be \(T^p + T\sqrt[p]{X} + \sqrt[p]{Y}\). This implies that \(\sqrt[p]{X},\sqrt[p]{Y} \in k'\), so that
        \(k' = k(\sqrt[p]{X})(\sqrt[p]{Y})/k(\sqrt[p](X))/k\) is a tower of degree \(p\) extensions, so that \([k':k] = p^2\), a contradiction.

        Then if \(L/E/k\) is a tower, such that \(E/k\) is not seperable, it must be the case that \(E = L\), and in this case, the extension \(L/k\) is not purely inseperable.
    \end{pb}
    \begin{pb}
        \begin{enumerate}
            \item No, although every finite extension is algebraic not every finite extension is seperable. As a counter example consider
            \(\mathbf{F}_p(t)(\alpha)/\mathbf{F}_p(t)\), where \(\alpha\) is a root of the irreducible polynomial (irreducible by Gauss' Lemma, then Eisensteins criterion)
            \(X^p - t\) in \(\mathbf{F}_p(t)[X]\). By construction we have \([\mathbf{F}_p(t)(\alpha):\mathbf{F}_p(t)] = p\) so the extension is finite,
            but \(\min(\alpha,\mathbf{F}_p(t)(\alpha)) =  X^p - t\) has zero derivative so the extension is not seperable.

            \item No, the extension \(\mathbb{Q}(\sqrt[3]{2})\) is a counterexample. As proof, first note that the extension is seperable since \(\mathbb{Q}\) is perfect.
            However, we have the embedding (fixing \(\mathbb{Q}\)) \(\sigma: \mathbb{Q}(\sqrt[3]{2}) \to \mathbb{C}\), where \(\sigma: \sqrt[3]{2} \mapsto \zeta_3\sqrt[3]{2}\), then by the extension theorem this can be
            extended to an automorphism \(\overline{\sigma}:\mathbb{C} \to \mathbb{C}\). \(\mathbb{Q}(\sqrt[3]{2})\) is not fixed under this automorphism implying it is not normal.

            \item No, in fact every purely inseperable extension is normal. Existence of purely inseperable extensions is proven in the first counter example 
            (for a more in depth proof of why this extension is purely inseperable see the lemma in problem 5). To see they are normal, let
            \(E/F\) be a purely inseperable extension, and \(L\) an algebraically closed field containing \(E\). Then for any \(\alpha \in E\)
            \(\alpha\) is the only root of its minimum polynomial over \(F\). Hence if \(\sigma\) is an \(F\)-automorphism of \(L\), then
            we have seen \(\sigma(\alpha)\) is a root of \({}^\sigma f = f\), hence \(\alpha\) is fixed. Since this holds for all elements of \(E\), we have that
            \(\sigma\vert_E = \mathbf{1}_E \in \text{Aut}(E)\) is an automorphism of \(E\) so that \(E\) is normal.
        \end{enumerate}
    \end{pb}
    \begin{pb}
        We have that \(a \in E\) is purely inseperable if and only if \(a^{p^n} \in F\) for some \(n \geq 0\), denote this value of \(n\) for an element \(\alpha\) as \(n_\alpha\).
        We also have that each element of \(F\) is purely inseperable. Now suppose that \(a,b \in E\) are purely inseperable, we have that
        \begin{align*}
            (a+b)^{p^{\max\set{n_a,n_b}}} = a^{p^{\max\set{n_a,n_b}}} + b^{p^{\max\set{n_a,n_b}}} \in F \\
            (ab)^{p^{\max\set{n_a,n_b}}} = a^{p^{\max\set{n_a,n_b}}}b^{p^{\max\set{n_a,n_b}}} \in F \\
            (a^{-1})^{n_a} = (a^{n_a})^{-1} \in F
        \end{align*}

        Since purely inseperable extensions are normal and \([P:F] \leq [E:F] < \infty\), we have that \(P\) is the splitting field of a polynomial \(f\) over \(F\).
        Each of the irreducible factors \(\set{f_i}_1^n\) of \(f\) must be purely inseperable, hence over \(P\) we have
        \begin{align*}
            f = \prod_1^n (x-\alpha_i)^{p^{k_i}}
        \end{align*}
        So we can take the tower of purely inseperable simple extensions (it is easy to see the extensions are purely inseperable since \([P:F]_s = 1\))
        \begin{align*}
            P = F(\alpha_1,\hdots,\alpha_n)/\cdots/F(\alpha_1)/F
        \end{align*}
        Where each simple extension has \(p\)-power order by inseperability of the elements, and hence \(P/F\) has \(p\)-power order by multiplicativity of degree.
    \end{pb}
\end{document}