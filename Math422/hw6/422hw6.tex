\documentclass[11pt]{article}
\usepackage{amsmath, amsfonts, amssymb,amsthm}
\usepackage[includeheadfoot]{geometry} % For page dimensions
\usepackage{fancyhdr}
\usepackage{enumerate} % For custom lists

\fancyhf{}
\lhead{Math 422hw3}
\rhead{Tighe McAsey - 37499480}
\pagestyle{fancy}

% Page dimensions
\geometry{a4paper, margin=1in}

\theoremstyle{definition}
\newtheorem{pb}{}

% Commands:

\newcommand{\set}[1]{\{#1\}}
\newcommand{\abs}[1]{\lvert#1\rvert}
\newcommand{\norm}[1]{\lvert\lvert#1\rvert\rvert}
\newcommand{\gen}[1]{\left\langle #1 \right\rangle}
\newcommand{\tand}{\text{ and }}
\newcommand{\tor}{\text{ or }}
\newcommand{\falg}{F^{\text{alg}}}
\newcommand{\gal}{\text{Gal}}

\begin{document}
    \begin{pb}
        First note that \([k(x,y):k(x^p,y^p)] = p^2\). It is obvious that \([k(x,y^p):k(x^p,y^p)] = p\), then I claim that \(\min(y;k(x,y^p)) = f(T) := T^p - y^p\) in
        \(k(x,y^p)[T]\). Proof being, firstly that \(f(y) = 0\), and appealing to Gauss' lemma, and eisenstein's Criterion (for \(y^p\) prime) \(f\) is irreducible.

        Define \(u(n)\) as \(y+x^{np + 1}\), then the following extensions satisfy the criteria.
        \begin{align*}
            k(x^p,y^p) \subsetneq k(x^p,y^p,u(n)) \subsetneq k(x,y), \quad \forall n \in \mathbb{N}
        \end{align*}
        Furthermore, if \(n \neq m\), then \(k(x^p,y^p,u(n)) \neq k(x^p,y^p,u(m))\).
        The first inequality is obvious, since if \(f \in k[x^p,y^p]\), then \(p \vert \deg_yf\). As for the second inequality,
        \([k(x^p,y^p,u(n)):k(x^p,y^p)] = p\), since \(p \vert [k(x^p,y^p,u(n)):k(x^p,y^p)]\), and \(u(n)\) satisfies the polynomial
        \(T^p - y^p - x^{p^{np+1}}\).

        Now suppose \(n \neq m\), then
        \begin{align*}
            k(x^p,y^p,u(n),u(m)) &= k(x^p,y^p,u(n) - u(m),u(n)) \\
            &= k(x^p,y^p, x(x^{np} - x^{mp}), u(n)) \\
            &= k(x,y^p,u(n)) = k(x,y)
        \end{align*}
        And hence \(k(x^p,y^p,u(n))k(x^p,y^p,u(m)) \supsetneq k(x^p,y^p,u(n)) \tand k(x^p,y^p,u(m))\), implying that the extensions are not equal.
    \end{pb}
    \begin{pb}
        We first show that \(\min(a^{1/p}:\mathbb{Q}) = X^p - a\), proof being we can factor \(X^p - a = \prod_{k=1}^p (X - a^{1/p}\zeta_p^k)\) in \(\mathbb{C}[X]\). If this
        polynomial were reducible in \(\mathbb{Q}\), then if \(g\) were a factor, the last coefficient of \(g\) must be of the form \(\pm a^{k/p}\). This is impossible since
        \(a^{k/p} \in \mathbb{Q}, \; k < p\), then by bezouts identity, there exist \(u,v\) such that \(uk + vp = 1\), implying that \(a^{1/p} = a^{uk/p}a^{vp/p} \in \mathbb{Q}\).

        This gives the desired result for both \(L\) and \(F\) extensions, since by multiplicativity of degree,
        \begin{align*}
            &p \vert [F(a^{1/p}):\mathbb{Q}] \tand [F: \mathbb{Q}] \leq p-1 \\
            &p \vert [L(a^{1/p}):\mathbb{Q}] \tand [L: \mathbb{Q}] \leq p-1
        \end{align*}
        implying that \(p \vert [F(a^{1/p}):F], [L(a^{1/p}):L]\). Then since \(F \supset \cos(2\pi/p), -\sin^2(2\pi/p)\) (proven below), we get that \(L = F(\sqrt{-\sin^2(2\pi/p)})\), i.e. \([L:F] = 2\), also note that
        \(N = L(a^{1/p})\). So that \([L(a^{1/p}):F] = [L(a^{1/p}):L][L:F] = 2p\), implies \(\# \gal(N/F) = 2p\). Finally, we have that
        \(\gal(N/F) \supset \gen{\tau,\sigma} \simeq D_p\), where \(\tau\) is complex conjugation and \(\sigma\) is a generator of the cyclic group \(\gal(\mathbb{Q}(\zeta_p):\mathbb{Q})\). The
        isomorphism follows from \(\sigma\tau = \tau\sigma^{-1}, \tau^2 = 1, \sigma^p = 1\), meaning the multiplication rules of \(D_p\) are satisfied. Then since 
        \(\# \gal(N/F) = 2p = \# \gen{\tau,\sigma}\) we have equality.

        To show that \(F \supset \cos(2\pi/p), \sin^2(2\pi/p)\), we have \(\zeta_p, \zeta_p^{-1} \in F\), hence we have \(\frac12(\zeta_p + \zeta_p^{-1}) = \cos(2\pi/p) \in F\). This implies we also have
        \((\zeta_p - \cos(2\pi/p))^2 = -\sin^2(2\pi/p)\).

    \end{pb}
    \begin{pb}
        \begin{align*}
            [F:\mathbb{Q}] = 2^9
        \end{align*}
        First note that \(F = \mathbb{Q}(\sqrt{p}\vert p \text{ prime and }p \leq 28)\), since the other radicals are simply products of these radicals, furthermore there are \(9\) primes less than or equal to \(28\).

        First we prove a lemma, namely: if 
        \(K\) has characteristic 0, \(a,b \in K\) then \([K(\sqrt{a},\sqrt{b}):K] = 4\) when
        \(\sqrt{a},\sqrt{b},\sqrt{ab} \not \in K\). Proof being:
        since \(\sqrt{a} \not \in K\) we have \([K(\sqrt{a}):K] = 2\), so we need to show that \(\sqrt{b} \not \in K(\sqrt{a})\), so that 
        \([K(\sqrt{a},\sqrt{b}):K(\sqrt{a})] = 2\), allowing us to conclude by multiplicativity of degree. So suppose for contradiction that
        \(\sqrt{b} = s\sqrt{a} + t\) for \(s,t \in K\). This implies that:
        \[b = as^2 + 2ts\sqrt{a} + t^2\]
        it follows that one of \(t\) or \(s\) must be zero (if both are zero we get \(b=0\) an immediate contradiction), else this contradicts
        \(\sqrt{a} \not \in K\). Suppose first \(s = 0\), then \(b = t^2 \implies t = \sqrt{b} \in K\) a contradiction. Then it must be the case that \(t = 0\), implying that \(b = as^2\), so that \(\sqrt{ab} = (\sqrt{a})(\sqrt{a}s) = as \in K\) also a contradiction, hence proving the lemma.

        Now we finish the proof using the lemma, we have \([\mathbb{Q}(\sqrt{p_1}):\mathbb{Q}] = 2\) by irrationality. Now assume that \([\mathbb{Q}(P):\mathbb{Q}] = 2^{\# P}\), for \(P\) a collection of at most \(n\) square roots of elements of \(\mathbb{Q}\), such that none of the \(2^n\) products of elements of the collection lie in \(\mathbb{Q}\), define
        \(K = \mathbb{Q}(\sqrt{p_1},\sqrt{p_2}, \hdots \sqrt{p_{n-1}})\), then by induction we have 
        \[[K(\sqrt{p_n}):K] = [K(\sqrt{p_{n+1}}):K] = [K(\sqrt{p_np_{n+1}}):K] = 2\]
        So that none of these elements lie in \(K\). We may apply the lemma that
        \[[K(\sqrt{p_n},\sqrt{p_{n+1}}):K] = 4 \implies [\mathbb{Q}(\sqrt{p_1}, \hdots, \sqrt{p_{n+1}}):\mathbb{Q}] = 2^{n+1}\]
        The result is proven, given that
        \[F = \mathbb{Q}(\sqrt{2},\sqrt{3},\sqrt{5},\sqrt{7},\sqrt{11},\sqrt{13},\sqrt{17},\sqrt{19},\sqrt{23})\]
        where clearly none of the products of square roots adjoined lie in \(\mathbb{Q}\).
    \end{pb}
    \begin{pb}
        Suppose that \(\# F = p^m = q\), and that \(\# K = q^n\), note that
        \(N_{K/F}: K \to F\), as it sends any element to the constant term of its minimum polynomial raised to some exponent. We know that \(\gal(K/F)\) is cyclic, with
        generator \(\Phi:a \mapsto a^q\). Let \(a \in K\), then
        \begin{align*}
            N_{K/F}(a) = \prod_{\sigma \in \gal(K/F)}\sigma(a)
            = \prod_{k=0}^{n-1} \Phi^k(a) = \prod_{k=0}^{n-1} a^{kq} = a^{\sum_{k=0}^{n-1} kq}
            = a^{\frac{q^n-1}{q-1}}
        \end{align*}
        It is immediate that \(N_{K/F}(0) = 0\), since \(K\) is a field, and an element cannot be conjugate to \(0\) any other element must be sent to \(F^*\) having order \(q-1\). Now since \(K\) is finite, we have shown \(K^*\) is cyclic, hence it has a generator \(\alpha\) with order \(q^{n}-1\), this implies that each of \(N(\alpha^i)\) are distinct for \(i \in \set{1,\hdots,q-1}\) by the formula above and hence \(\#\set{N_{K/F}(\alpha^i)}_{i=1}^{q-1} = q-1 = \#F^*\), so that \(N\) maps onto both \(0\) and \(F^*\).
    \end{pb}
    \begin{pb}
        We can define the map \(\varphi: \mathbb{Z}/2 \mathbb{Z} \overset{\varphi}{\to} (\mathbb{Z}/4 \mathbb{Z})\) as \(\varphi(1): x \mapsto -x\), this is a well defined
        automorphism, since \(\varphi(1)^2 = \mathbf{1}_{\mathbb{Z}/ 4 \mathbb{Z}} = \varphi(0) = \varphi(1 + 1)\). Any element \(x \in D_4\) can be written in the form of
        \(\sigma^i\tau^j\) using the relation \(\sigma\tau = \tau\sigma^{-1}\). So define the map
        \begin{align*}
            \psi: D_4 &\to \mathbb{Z}/4 \mathbb{Z} \underset{\varphi}{\rtimes} \mathbb{Z}/2 \mathbb{Z} \\
            \sigma^i\tau^j &\mapsto (i,j)
        \end{align*}
        is an isomorphism. \(\mathbf{1} \mapsto (0,0)\) is immediate. And (here I deal with both possible cases \(j = 1, 0\) seperately)
        \begin{align*}
            &\psi(\sigma^i\tau \sigma^k \tau^\ell) = \psi(\sigma^{i-k}\tau^{1+\ell}) = (i-k,1+\ell) = (i + \varphi(1)(k), 1 + \ell) = (i,1)(k,\ell) = \psi(\sigma^i\tau) \psi(\sigma^k \tau^\ell)\\
            &\psi(\sigma^i\tau^{0} \sigma^k \tau^\ell) = \psi(\sigma^{i+k} \tau^\ell) = (i + k, \ell) = (i + \varphi(0)(k), 0 + \ell) = (i,0)(k,\ell) = \psi(\sigma^i\tau^{0}) \psi(\sigma^k \tau^\ell)
        \end{align*}
        This proves that \(\psi\) is a homomorphism, and \[\psi(\sigma^i\tau^j) = (0,0) \iff i \equiv 0 \text{mod}4 \tand j \equiv 0 \text{mod}2 \iff \sigma^i\tau^j = \mathbf{1}\]
        proving that \(\ker \psi = \mathbf{1}\). Then since \(\# D_4 = \# \mathbb{Z}/4 \mathbb{Z} \underset{\varphi}{\rtimes} \mathbb{Z}/2 \mathbb{Z}\) and the map is injective, it must also be surjective.
    \end{pb}
    \begin{pb}
        \textbf{(a)}
            It is immediate that \(\mathbb{Q}\) satisfies the conditions of containing \(\pm 1\). The degree being at most \(2^r\) is immediate since \(K\) is a tower of \(r\) extensions of degree at most \(2\). An example of when the degree is equal to \(2^r\) is when each of the \(a_i\) are primes, as shown in the solution to exercise 3. An example of the degree less than \(2^r\) is when \(a_r = a_1a_2\), since this is contained in the previous extension having degree at most \(2^{r-1}\). Explicit examples would be \(\mathbb{Q}(\sqrt{2},\sqrt{3})\) having degree 4, and \(\mathbb{Q}(\sqrt{2},\sqrt{3},\sqrt{6})\) having degree \(4 < 8\). It remains to show that \(K/\mathbb{Q}\) is a \(2\)-Kummer extension, the extension is clearly normal and seperable hence Galois, since it is the splitting field of a family of degree 2 polynomials algebraic over a characteristic 0 field. Suppose that \(K/\mathbb{Q} = 2^r\), else we can simply remove dependent \(\sqrt{a_i}\) until it does. Then each of \(\sigma_1,\hdots,\sigma_r\) are in \(\gal(K/\mathbb{Q})\)
            where \(\sigma_i\vert_{\mathbb{Q}(\sqrt{a_1},\hdots,\sqrt{a_{i-1}},\sqrt{a_{i+1}}\hdots,\sqrt{a_r})} = 1, \sigma(\sqrt{a_i} = -a_i)\). It follows that each of the \(2^r\) combinations of these permutations are unique, hence \(\gal(K/\mathbb{Q}) = \gen{\sigma_1,\hdots,\sigma_r}\). Since the group is generated by order 2 elements, all of its elements have order 2 and groups with exponent \(2\) are abelian, hence \(K/\mathbb{Q}\) is 2-Kummer

            \textbf{Proof That Groups of Exponent 2 Are Abelian:} \(a,b \in G\), then
            \(ab = (ab)^{-1} = b^{-1}a^{-1} = ba\)


        \textbf{(b)} It is immediate that both are less than or equal to \(n\).
        
        First suppose that \(a^k \in K^{*^n}\), then \(a^{k/n} \in K^*\), so that \(\min(a^{1/n};K) \vert x^k -a^{k/n}\), implying that
        \([K(\sqrt[n]{a}):K] \leq k\), so that \([K(\sqrt[n]{a}):K] \leq o(aK^{*^n})\)

        Conversely, supppose that \([K(\sqrt[n]{a}):K] = k\), then \(a^{1/n}\) has minimum polynomial \(g\) of degree \(k\), furthermore 
        \(g \vert x^n - a = \prod_0^{n-1} (x - a^{1/n}\zeta_n^j)\), so that the constant term of \(g\) must be \(a^{k/n}\zeta_n^r\) for some \(r\), then since \(\zeta_n \in K\), this implies that \(a^{k/n} \in K^*\), so that
        \(a^{k} \in K^{*^n}\) this implies that \([K(\sqrt[n]{a}):K] \geq o(aK^{*^n})\). Both inequalities taken together implies equality.
    \end{pb}
\end{document}