\documentclass[11pt]{article}
\usepackage{amsmath, amsfonts, amssymb,amsthm}
\usepackage[includeheadfoot]{geometry} % For page dimensions
\usepackage{fancyhdr}
\usepackage{enumerate} % For custom lists

\fancyhf{}
\lhead{Math 422hw1}
\rhead{Tighe McAsey - 37499480}
\pagestyle{fancy}

% Page dimensions
\geometry{a4paper, margin=1in}

\theoremstyle{definition}
\newtheorem{pb}{}

% Commands:

\newcommand{\set}[1]{\{#1\}}
\newcommand{\abs}[1]{\lvert#1\rvert}
\newcommand{\norm}[1]{\lvert\lvert#1\rvert\rvert}
\newcommand{\tand}{\text{ and }}
\newcommand{\tor}{\text{ or }}

\begin{document}
    \begin{pb}
        for \(p=2\) we have splitting field \(\mathbb{Q}(\sqrt2)\) which is a degree \(2\) extension. Now let \(p > 2\), then the splitting field is
        \(k := \mathbb{Q}(2^{1/p},\zeta_p2^{1/p},\hdots,\zeta_p^{p-1}2^{1/p})\), then \(\frac{\zeta_p2^{1/p}}{2^{1/p}} = \zeta_p \in k\), hence
        \(k \subset \mathbb{Q}(\zeta_p,2^{1/p})\), and the reverse inclusion is obvious so they are equal. It follows that the degree of this extension is
        \(p(p-1)\), since \(\zeta_p\) has minimum polynomial \(x^{p-1} + x^{p-2} + \cdots + 1\) of degree \(p-1\) in \(\mathbb{Q}[x]\). And \(2^{1/p}\) has minimum polynomial
        \(x^p - 2\) (Gauss' lemma to reduce to \(\mathbb{Z}\), then irreducible from Eisenstein) of degree \(p\). 
        Implying that \(p, p-1 \vert [k:\mathbb{Q}]\), or equivalently \(p(p-1) \vert [k:\mathbb{Q}]\). For equality, note that
        \([\mathbb{Q}(\zeta_p,2^{1/p}):\mathbb{Q}(\zeta_p)] \leq p\), since \(2^{1/p}\) still satisfies \(x^p-2\), so that
        \begin{align*}
            p(p-1) \leq [k:\mathbb{Q}] = [\mathbb{Q}(\zeta_p,2^{1/p}):\mathbb{Q}(\zeta_p)][\mathbb{Q}(\zeta_p):\mathbb{Q}] \leq p(p-1)
        \end{align*}

        Proof that \(x^{p-1} + x^{p-2} + \cdots + 1\) is irreducible: We first note that \(f(x)\) is irreducible in \(\mathbb{Q}\) when it is irreducible in \(\mathbb{Z}\)
        by Gauss' lemma. Then \(f(x)\) is irreducible if and only if \(f(x+1)\) is irreducible, 
        one way to see this is \(F: \mathbb{Z}[x] \to \mathbb{Z}[x], x \mapsto x+1\) is a \(\mathbb{Z}\)-module automorphism.
        Hence \[f(x) = g(x)h(x) \iff f(F(x)) = F(f(x)) = F(g(x))F(h(x)) = g(F(x))h(F(x))\]
        Then irreducibility follows from Eisensteins criterion after the following computation;
        \begin{align*}
            \sum_{k=0}^{p-1} (x+1)^k &= \frac{(x+1)^p - 1}{(x+1)-1} = x^{-1}\sum_{k=1}^p \binom{p}{k}x^k = \sum_1^p  \binom{p}{k}x^{k-1} \\
            &= x^{p-1} + \sum_{k=1}^{p-2}a_kx^k + p, \text{ where } p \vert a_k
        \end{align*}
    \end{pb}
    \begin{pb}
        I claim that \(z := \zeta_3 + 2^{1/3}\) is such a number, it will suffice to show that \(\deg(\min(z,\mathbb{Q})) = 6 = [\mathbb{Q}(\zeta_3,2^{1/3}):\mathbb{Q}]\).
        But then since \(\mathbb{Q}(z)\) is a subextension it has degree 2,3 or 6 over \(\mathbb{Q}\), so it is sufficient to show that \(\deg \min(z,\mathbb{Q}) > 3\).
        Then we can the take basis for \(\mathbb{Q}(\zeta_3,2^{1/3})/\mathbb{Q}\) to be \(\set{2^{1/3}\zeta_3,2^{2/3}\zeta_3,2^{1/3}\zeta_3^{-1},2^{2/3}\zeta_3^{-1},\zeta_3,\zeta_3^{-1}}\).
        We can see this is a basis, since it contains six elements, and \(1 = 2(\zeta_3 + \zeta_3^{-1})\) is sufficient to check that it spans \(\mathbb{Q}(\zeta_3,2^{1/3})\)
        since this allows us to write the rest of the extension in terms of the basis. First we compute the powers of \(z\) up to \(3\).
        \begin{align*}
            &z^2 = \zeta_3^{-1} + 2 \zeta_3 2^{1/3} + 2^{2/3} \quad \quad z^3 = 3(\zeta_3^{-1}2^{1/3} + \zeta_3 2^{2/3} + 2(\zeta_3 + \zeta_3^{-1})) \\
            &1 = \zeta_3 + \zeta_3^{-1} \quad \quad \quad \quad \quad \quad \quad \;\; z = \zeta_3 + 2^{1/3}\zeta_3 + 2^{1/3}\zeta_3^{-1}
        \end{align*}
        Now consider any degree \(\leq 3\) polynomial evaluated at \(z\),
        \begin{align*}
            p(z) &= az^3 + bz^2 + cz + d \\
            &= (2b + c) (2^{1/3}\zeta_3) + 3a 2^{2/3}\zeta_3 + (3a + c) (2^{1/3}\zeta_3^{-1}) + (6a + c + d)(\zeta_3) + (6a + b + d)(\zeta_3^{-1})
        \end{align*}
        Then by linear independence of basis elements, \(p(z)\) is equal to zero if and only if
        \begin{align*}
            2b + c &=0 \\
            3a + c &=0 \\
            6a+c+d &=0 \\
            6a+b+d &=0
        \end{align*}
        then from the second 2 equations we get \(c = -b\), implying from the first equation that \(c=b = 0\) which implies in the second equation
        \(a = 0\), so that \(a=b=c=0\), which means \(d\) must be zero as well. So that \(\deg \min(z,\mathbb{Q}) > 3\), which implies it must be 6 and we are done.
    \end{pb}
    \begin{pb}
        \([K:\mathbb{Q}] \in \set{6,3,2,1}\), since \([K:\mathbb{Q}] \vert 6\).
        For a more detailed explanation, the largest irreducible factor of \(f(x)\) may have degree 1,2 or 3. The first case is the trivial case where \(f\) splits over \(\mathbb{Q}\),
        so that \(K = Q\) is an extension of degree 1. In the second case \([K:\mathbb{Q}]\) has degree \(2\), since adjoining a root \(\alpha\) of a quadratic polynomial gives a field extension of degree 2
        containing the other root. Finally if \(f\) itself is irreducible, then \([K:\mathbb{Q}]\) may have degree 3 in the case where the extension
        is simple, or degree 6 otherwise no other values are possible, since when a root of degree 3 is adjoined, then the polynomial splits in the new field, or
        is a quadratic, so that the roots are contained in a degree 2 extension of the degree 3 extension, which has degree 6 over the original field.

        Examples:
        \begin{align*}
            &[K:\mathbb{Q}] = 1 \quad f(x) = (x-1)^3 \quad K = \mathbb{Q} \\
            &[K:\mathbb{Q}] = 2 \quad f(x) = (x-1)(x^2 - 2) \quad K = \mathbb{Q}(\sqrt2) \\
            &[K:\mathbb{Q}] = 3 \quad f(x) = (x - (\zeta_7 + \zeta_7^{-1}))(x - (\zeta_7^3 + \zeta_7^{4}))(x - (\zeta_7^2 + \zeta_7^{5})) = x^3 + x^2 - 2x - 1 \quad K = \mathbb{Q}(\zeta_7 + \zeta_7^{-1}) \\
            &[K:\mathbb{Q}] = 6 \quad f(x) = x^3 - 2 \quad K = \mathbb{Q}(2^{1/3},\zeta_3)
        \end{align*}
    \end{pb}
    \begin{pb}
        Let \(L\) be the algebraic closure containing \(N\) and \(L'\) be an algebraic closure of \(N'\), then we have the embedding by the identity \(E \to L'\). 
        By the extension theorem, there exists an extension \(\sigma: N \to L'\)
        which is identity on \(E\) and thus \(F\). We first check that \(^\sigma N\) is normal, if \(N\) is the splitting field of polynomials \(\set{f_i}_i\), then \(^\sigma N\)
        is the splitting field of \(\set{^\sigma f_i}_i\), since if \(f_i\) has roots \(\set{\alpha_j}_j\), then \(\set{^\sigma \alpha_j}_j\) are the roots of \(^\sigma f_i\). 
        If \(^\sigma N\) werent the normal closure of \(E\) in \(L'\), then there would exist \(E \subset N'' \subsetneq {}^\sigma N\) normal. Then since \(\sigma\) is invertible on its image,
        we could take \(E \subset {}^{\sigma^{-1}} N'' \subsetneq N\), where once again the image would be normal, but this contradicts \(N\) being the normal closure. Hence \(^\sigma N\) is the
        normal closure of \(E\) in \(L'\), this implies that \(E \subset {}^\sigma N \cap N' = {}^\sigma N\) since the intersection is normal. Hence \(\sigma: N \to N'\) is an \(F\) homomorphism.

        If \(\sigma: N \to N'\) is an \(F\) homomorphism, then it must be injective. From the construction above we have identity on \(E\), so that \(E \subset {}^\sigma N \subset N'\), but then
        by hypothesis, there are no subextensions implying that \(\sigma(N) = N'\) is an \(F\)-isomorphism.
    \end{pb}
    \begin{pb}
        \(E\) is contained in some algebraicly closed field \(L\), then \(\sigma\) can be seen as an embedding from \(K\) into \(L\), so that there exists an embedding \(\tau:E \to L\) extending
        \(\sigma\) by the extension theorem. Since \(E\) is normal this embedding is an automorphism on \(E\) (one of the three equivalent conditions for normal extensions - NOR 1 in Lang).
    \end{pb}
\end{document}
