\documentclass[11pt]{article}
\usepackage{amsmath, amsfonts, amssymb,amsthm}
\usepackage[includeheadfoot]{geometry} % For page dimensions
\usepackage{fancyhdr}
\usepackage{enumerate} % For custom lists

\fancyhf{}
\lhead{Math 422hw7}
\rhead{Tighe McAsey - 37499480}
\pagestyle{fancy}

% Page dimensions
\geometry{a4paper, margin=1in}

\theoremstyle{definition}
\newtheorem{pb}{}

% Commands:

\newcommand{\set}[1]{\{#1\}}
\newcommand{\abs}[1]{\lvert#1\rvert}
\newcommand{\norm}[1]{\lvert\lvert#1\rvert\rvert}
\newcommand{\gen}[1]{\left\langle #1 \right\rangle}
\newcommand{\tand}{\text{ and }}
\newcommand{\tor}{\text{ or }}
\newcommand{\falg}{F^{\text{alg}}}
\newcommand{\gal}{\text{Gal}}
\newcommand{\floor}[1]{\left\lfloor #1 \right\rfloor}

\begin{document}
    \begin{pb}
        We have that \(\mathbf{Q}(\zeta_7)/\mathbf{Q} \simeq (\mathbf{Z}/n\mathbf{Z})^\times \simeq \mathbf{Z}/(2)\oplus\mathbf{Z}/(3)\). Since this is an abelian group, all of its
        subgroups are normal. So that \(\varphi:\sigma \mapsto \sigma \vert_{\mathbf{Q}(a)}\) maps surjectively onto \(\gal(\mathbf{Q}(a)/\mathbf{Q})\) by the galois correspondence.
        It is immediate that \(\gal(\mathbf{Q}(\zeta_7)/\mathbf{Q}) = \gen{\tau}\oplus\gen{\sigma}\), where \(\tau\) represents complex conjugation and \(\sigma: \zeta_7 \mapsto \zeta_7^2\). \(\sigma(a) \neq a \tand \tau(a) = a\) verifies that \(\ker \varphi = \tau\), i.e.
        \begin{align*}
            \gal(\min(a;\mathbf{Q})) = \gal (\mathbf{Q}(a)/\mathbf{Q}) \simeq \gal(\mathbf{Q}(\zeta_7)/\mathbf{Q})/\gen{\tau}
            \simeq \gen{\sigma} \simeq \mathbf{Z}/(3)
        \end{align*}
        This also implies that 
        \begin{align*}
            \min(a;\mathbf{Q}) = (x - a)(x - \sigma(a))(x - \sigma^2(a))
            = (x - (\zeta_7^3 + \zeta_7^4))(x - (\zeta_7^6 + \zeta_7))(x - (\zeta_7^5 + \zeta_7^2))
        \end{align*}
    \end{pb}
    \begin{pb}
        Denote \(K := \mathbf{Q}(\sqrt[3]{2},\zeta_3,\sqrt{3})\). Then \(K\) is the splitting field of polynomials \(x^3 - 2\), and \(x^2 - 3\) over \(\mathbf{Q}\), hence is galois. To see that \([K:\mathbf{Q}] = 12\), denote \(F = \mathbf{Q}(2^{1/3},\sqrt{3})\). Then
        \[3 = [\mathbf{Q}(2^{1/3}):\mathbf{Q}]\vert[F:\mathbf{Q}] \tand 2 = [\mathbf{Q}(\sqrt{3}):\mathbf{Q}]\vert[F:\mathbf{Q}]\]
        Similarly it is immediate that \([F:\mathbf{Q}] \leq 6\), so that it is in fact equal to 6.
        Now since \(F \subset \mathbf{R}\), we have \([F(i):F] = 2\), but note that by the expanded form for \(\zeta_3\) we have \(K = F(i)\), so that by multiplicativity of degree \([K:\mathbf{Q}] = 12\). We can equivelantly write \(K = \mathbf{Q}(\sqrt[6]{108},i)\) then we have
        \[\gal(K/\mathbf{Q}) = \gen{\sigma,\tau}\]
        where \(\sigma: \sqrt[6]{108} \mapsto \sqrt[6]{108}\zeta_6\), and \(\tau\) represents complex conjugation. Furthermore we have \(\sigma\tau \neq \tau\sigma\), as they do not agree on \(\sqrt[6]{108}\). But since \([G:\gen{\sigma}] = 2\), \(\gen{\sigma}\) must be normal and hence \(\tau\gen{\sigma}\tau = \gen{\sigma}\), these two conditions together imply that \(\tau\sigma\tau\) is a generator of \(\gen{\sigma}\) not equal to \(\sigma\), hence we must have \(\tau\sigma\tau = \sigma^{-1}\) implying that \(\tau\sigma = \sigma^{-1}\tau\), i.e. that \(\gal(K/\mathbf{Q})\) satisfies the relations of the dihedral group;
        \[\gal(K/\mathbf{Q}) \simeq D_{12}\]
        
        % We have
        % each of \(\sigma,\tau,\rho \in \gal{K/\mathbf{Q}}\) where 
        % \[\sigma: 2^{1/3}\mapsto \zeta_32^{1/3} \quad \tau: \sqrt{3} \mapsto -\sqrt{3} \quad \rho: i \mapsto -i\]
        % It is immediate that \(\sigma\) is distinct from the other two (and their products) since it has order 3. To see that
        % \(\tau \tand \rho\) are distinct, note that \(\tau \vert_{\mathbf{R}} = 1 \neq \rho \vert_{\mathbf{R}}\). This implies that 
        % \[12 = \# \gen{\tau,\sigma,\rho} \leq [K:\mathbf{Q}] = [K:\mathbf{Q}(\sqrt[3]{2},\zeta_3)][\mathbf{Q}(\sqrt[3]{2},\zeta_3):\mathbf{Q}] \leq 12\]
        % Finally, we have that 
        % \begin{align*}
        %     &\sigma\rho(2^{1/3}) = \zeta_32^{1/3} \neq \rho\sigma(2^{1/3}) = \zeta_3^{-1}2^{1/3} \\
        %     &\sigma\tau(2^{1/3}) = \zeta_32^{1/3} \neq \tau\sigma(2^{1/3}) = \zeta_3^{-1}2^{1/3}
        % \end{align*}
        % So that \(\gen{\sigma,\rho} \tand \gen{\sigma,\tau}\) are noncommutative groups of order 6 (thus can be identified with \(S_3\)), and \(\gen{\rho,\tau}\) has exponent 2 hence is abelian. Together this gives a description for \(\gal(K/\mathbf{Q})\), namely
        % \begin{align*}
        %     \gal(K/\mathbf{Q}) 
        % \end{align*}
    \end{pb}
    \begin{pb}
        Let \(f\) be a monic polynomial, such that \[f \vert \Phi_n(x) \tand f \vert \Phi_m(x)\] in \(\mathbf{F_p}(x)\). It follows that \(f^2 \vert \Phi_n(x)\Phi_m(x) \vert x^{nm} - 1\), but
        the derivative of \(x^{nm} - 1\) is \(nmx^{nm-1} \neq 0\) implies that \(x^{nm} - 1\) is seperable, so that \(f\) must have degree \(0\) (any root of \(f\) is a multiple root of \(x^{nm}-1\)).
    \end{pb}
    \begin{pb}
        We use the recursion formula, \(\Phi_8(x)\Phi_4(x)\Phi_2(x)\Phi_1(x) = x^8 - 1\), then
        \(\Phi_2(x) = x+1\), applying the recursion formula to \(\Phi_4(x)\), we find that
        \[\Phi_4(x) = \frac{x^4-1}{\Phi_2(x)\Phi_1(x)} = \frac{x^4-1}{x^2 - 1} = x^2 + 1\]
        Now finally applying it to \(\Phi_8\), we find
        \[\Phi_8(x) = \frac{x^8-1}{\Phi_4(x)\Phi_2(x)\Phi_1(x)} 
        =\frac{x^8-1}{x^4-1} = x^4 + 1\]

        If \(p = 2\), then \(x^4 + 1 = (x+1)^4\) so the result is trivial. So assume \(p \neq 2\)

        First if there exists \(a \in \mathbf{F_p}\), such that \(a^2 = -1\), then \([\mathbf{F_p}(\sqrt{a}):\mathbf{F_p}] \leq 2\), and \(\sqrt{a}^4 + 1 = 0\), so that \(x^4 + 1\) is not irreducible. Assume that \(-1 \not \in \mathbf{F_p}^2\), then for any \(\alpha \in \mathbf{F_p}^2\) we have \(-\alpha \not \in \mathbf{F_p}^2\) (otherwise we have \(-\alpha/\alpha = -1 \in \mathbf{F_p}^2\)), from the previous homework we showed that \(\# \mathbf{F_p} = \frac{p+1}{2}\), we may write \(\mathbf{F_p} = \set{0,1,\hdots,\frac{p-1}{2}, -\frac{p-1}{2},\hdots,-1}\) so that by the pigeonhole principle \(\mathbf{F_p}\) has exactly one of \(k,-k\) for each \(0 \leq k \leq \frac{p-1}{2}\). This implies that one of \(2,-2 \in \mathbf{F_p}^2\). Let \(\alpha\) be in \(\mathbf{F_p}\), such that \(\alpha^2 = \pm 2\). Then we have
        \begin{align*}
            \Phi_8(x) = x^4 + 1 = (x^2 + \alpha x \pm 1)(x^2 - \alpha x \pm 1)
        \end{align*}
    \end{pb}
    \begin{pb}
        We first show that \(\phi(p^r) = p^{r-1}(p-1)\), we can show this by showing \(\Phi_{p^r}(x)= \Phi_p(x^{r-1})\) this is true for \(r = 1\), now assume it for \(k < r\), we have that
        \begin{align*}
            \Phi_{p^r}(x) = \frac{x^{p^r}-1}{\prod_{0 \leq k<r}\Phi_{p^k}(x)} 
            = \frac{x^{p^r}-1}{(x-1)\prod_{1 \leq k<r}\Phi_{p}(x^{p^{k-1}})}
            = \frac{x^{p^r}-1}{(x^p-1)\prod_{2 \leq k<r}\Phi_{p}(x^{p^{k-1}})}
        \end{align*}
        Continuing this process recursively we get
        \begin{align*}
            \Phi_{p^r}(x) = \frac{x^{p^r}-1}{(x^{p^{r-1}}-1)}
        \end{align*}
        Substituting \(u\) in as \(x^{p^{r-1}}\) on the right hand side yields the familiar formula
        \(\frac{u^p - 1}{u - 1} = \Phi_p(u)\), which proves the lemma by substituting back \(u \mapsto x^{p^{r-1}}\).

        Suppose that \(\zeta\) is an \(nk\)-th root of unity, then \(\zeta\) satisfies the polynomial \(x^{nk}-1=0\), this of course implies that \(\zeta^k\) satisfies a polynomial of degree \(n\), namely \(x^n-1\). Since \(\Phi_{nk}\) is the minimal polynomial of \(\zeta\), it will suffice to show that \(\phi(nk) = \deg\Phi_{nk} = \deg\Phi_n(x^k) = k\deg\Phi_n\).
        Suppose that \(\prod_i p_i^{e_i}\) is the prime factorization of \(nk\), we may write \(n = kr\) so that \(nk = k^2r\), then the prime factorizations of \(k \tand r\) may be written as
        \begin{align*}
            k = \prod p_i^{k_i} \quad \tand \quad r = \prod p_i^{r_i} \quad \text{ Such that } \quad 2k_i + r_i = e_i
        \end{align*}
        It follows from this factorization and the above lemma that
        \begin{align*}
            &\phi(nk) = \prod_i \phi(p_i^{e_i}) = \prod p_i^{e_i-1}(p_i-1) = \prod_i p_i^{2k_i+r_i-1}(p_i-1)\\
            &k\phi(n) = \left(\prod_i p_i^{k_i}\right)\left(\prod_i \phi(p_i^{k_i+r_i})\right)
            = \left(\prod_i p_i^{k_i}\right)\left(\prod_i p_i^{k_i+r_i-1}(p_i-1)\right) = \prod_i p_i^{2k_i+r_i-1}(p_i-1)
        \end{align*}
        As desired.
    \end{pb}
    \begin{pb}
        Let \(k\) denote the order of \(a\) in the multiplicative group of \(\mathbf{F}\). First assume that \(k \vert n\), then \(n = k\ell\) for some \(\ell\), so that \(a^n = a^{k\ell} = 1^\ell = 1\). Conversely, suppose for contradiction that \(a^n = 1\) and \(k \nmid n\). Then by long division we may write \(n = mk + r\) for some \(0 < r < k\). It follows that
        \begin{align*}
            1 = 1^{-m} = a^{-km} = a^{-km}1 = a^{-km}a^n = a^r
        \end{align*}
        But \(0 < r < k\) contradicts \(k = o_{\mathbf{F}}(a)\).
    \end{pb}
\end{document}