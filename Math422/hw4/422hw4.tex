\documentclass[11pt]{article}
\usepackage{amsmath, amsfonts, amssymb,amsthm}
\usepackage[includeheadfoot]{geometry} % For page dimensions
\usepackage{fancyhdr}
\usepackage{enumerate} % For custom lists

\fancyhf{}
\lhead{Math 422hw3}
\rhead{Tighe McAsey - 37499480}
\pagestyle{fancy}

% Page dimensions
\geometry{a4paper, margin=1in}

\theoremstyle{definition}
\newtheorem{pb}{}

% Commands:

\newcommand{\set}[1]{\{#1\}}
\newcommand{\abs}[1]{\lvert#1\rvert}
\newcommand{\norm}[1]{\lvert\lvert#1\rvert\rvert}
\newcommand{\tand}{\text{ and }}
\newcommand{\tor}{\text{ or }}
\newcommand{\falg}{F^{\text{alg}}}

\begin{document}
    \begin{pb}
        Let \(\falg\) be an algebraic closure of \(F\) containing both \(K \tand L\). Now let \(\sigma\) an extension of an embedding \(F \to \falg\) to
        \(KL\). Then for any \(x \in KL, x = \frac{\sum_1^n k_i\ell_i}{\sum_1^m k_j' \ell_j'}\). For \(k_i,k_j' \in K \tand \ell_i,\ell_j' \in L\). Then by the homomorphism property,
        \[\sigma(x) = \sigma(\frac{\sum_1^n k_i\ell_i}{\sum_1^m k_j' \ell_j'}) = \frac{\sum_1^n \sigma(k_i)\sigma(\ell_i)}{\sum_1^m \sigma(k_j')\sigma(\ell_j')} \in \sigma(K)\sigma(L)\]
        Furthermore, any \(y \in \sigma(K)\sigma(L)\) is of the form \(\frac{\sum_1^n \sigma(k_i)\sigma(\ell_i)}{\sum_1^m \sigma(k_j')\sigma(\ell_j')}\), so the \(\sigma\) is onto with range \(\sigma(K)\sigma(L)\).
        In other words we have proven \(\sigma(KL) = \sigma(K)\sigma(L)\). But since \(K,L\) are normal we have \(K = \sigma(K) \tand L = \sigma(L)\).

        The converse is \textbf{not} true, since taking \(F = \mathbb{Q}, K = \mathbb{Q}(\sqrt[3]{2}), L = \mathbb{Q}(\zeta_3)\) we have
        \(KL = \mathbb{Q}(\sqrt[3]{2}, \zeta)\) is normal over \(\mathbb{Q}\) (it is the splitting field of \(x^3 -2\) as shown in the previosu homework),
         whereas \(K\) is not normal over \(\mathbb{Q}\) (Take the extension of the identity on 
        \(\mathbb{Q}\) mapping \(\sqrt[3]{2} \mapsto \zeta_3\sqrt[3]{2}\) which is not in \(\text{Aut} K\)).
    \end{pb}
    \begin{pb}
        Consider the polynomial \(f:= x^k - n\) in \(\mathbf{F_p}\). Since \(k = 2 \tor 3\), \(f\) either has a factor of degree one, hence a root or \(f\) is irreducible over \(\mathbf{F_p}\).
        If \(f\) has a root we are done, so assume not. Then we can consider the extension \(\mathbf{F_p}(\alpha)\), where \(\alpha\) is a root of \(f\). This extension has degree \(k\), hence
        \(\mathbf{F_p}(\alpha)\) is a finite field of cardinality \(p^k\). Since all finite fields of equal cardinality \(p^k\) are \(\mathbf{F_p}\) isomorphic, we have an isomorphism
        \(\sigma: \mathbf{F_p} \to F\), so that \(0 = \sigma(0) = \sigma(f(\alpha)) = f({}^\sigma\alpha)\) so that \(^\sigma \alpha \in F\) is a root of \(x^k - n\), hence 
        \(^\sigma \alpha\) satisfies the equation \({}^\sigma \alpha^k = n\).
    \end{pb}
    \begin{pb}
        \begin{enumerate}
            \item \textbf{True} Firstly, \(F\) is finite, hence perfect and since \(K\) is a finite field, \(K/F\) is a finite extension so it is algebraic over a perfect field, hence seperable.
            Secondly, since \(K,F\) are finite, they must have characteristic \(p\). It follows that \(\# K = p^n, \# F = p^m, n \geq m\). Then we have that \(K\) is the splitting field of
            \(x^{p^n} - x\), so that \(K/\mathbf{F_p}\) is normal. Then any extension of a map from \(\mathbf{F_p}\) to an algebraic closure extends to an automorphism of \(K\), and since any extension of a map \(\sigma\)
            from \(F\) into an algebraic closure  to \(K\) is just an extension of \(\sigma\vert_{\mathbf{F_p}}\) to \(K\) this extension must also be in \(\text{Aut} K\), proving that \(K/F\) is normal. This proves the extension is Galois.
            \item \textbf{True} First let \(u\) denote \(t^4 + t^{-4}\), then \(K\) is the splitting field of the polynomial \(f = X^8 - uX^4 + 1\) in \(F[X]\). This is easily seen, since
            \(t \in \set{t\zeta_4^n,t^{-1}\zeta_4^n}_{n=1}^4 \subset \mathbb{C}(t)\) are the roots of \(f\).
            \item \textbf{False} Since \(K/F\) is finite, it is algebraic. \(K/F\) admits a normal closure \(M/K/F\), so that \(M/F\) is normal.
            \item \textbf{False} Consider \(K = \mathbf{F_3}\), then \(f(K) = \set{0,1} \neq K\)
            \item \textbf{True} For every element \(x\) of \(K\) not equal to zero, \(p \neq 2 \implies 2x \neq 0 \implies x \neq -x\), so that \(f(x) = f(-x)\). Now we need only show that
            \(y \not \in \set{x,-x}\) implies that \(f(y) \neq f(x)\). As proof note that the polynomial \(T^2 - x^2 \in K[T]\) can have at most two roots in \(K\) by the factor theorem.
            Hence \(f\) is 1-1 on 0 and 2-1 on each other element. This implies that \(\# f(K) = 1 + \frac{\# K - 1}{2} = \frac{p^n + 1}{2}\).
            \item \textbf{True} \(\mathbb{Q}(S)\) is the splitting field of the collection of polynomials \(\set{x^2 - p}_{p \; \text{prime}}\), so \(K/\mathbb{Q}\) is normal.
            Furthermore, \(\mathbb{Q}\) has characteristic zero, so is perfect hence \(K/\mathbb{Q}\) is seperable making it Galois.
        \end{enumerate}
    \end{pb}
\end{document}