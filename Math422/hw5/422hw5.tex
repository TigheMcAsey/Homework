\documentclass[11pt]{article}
\usepackage{amsmath, amsfonts, amssymb,amsthm}
\usepackage[includeheadfoot]{geometry} % For page dimensions
\usepackage{fancyhdr}
\usepackage{enumerate} % For custom lists

\fancyhf{}
\lhead{Math 422hw3}
\rhead{Tighe McAsey - 37499480}
\pagestyle{fancy}

% Page dimensions
\geometry{a4paper, margin=1in}

\theoremstyle{definition}
\newtheorem{pb}{}

% Commands:

\newcommand{\set}[1]{\{#1\}}
\newcommand{\abs}[1]{\lvert#1\rvert}
\newcommand{\norm}[1]{\lvert\lvert#1\rvert\rvert}
\newcommand{\gen}[1]{\left\langle #1 \right\rangle}
\newcommand{\tand}{\text{ and }}
\newcommand{\tor}{\text{ or }}
\newcommand{\falg}{F^{\text{alg}}}
\newcommand{\gal}{\text{Gal}}

\begin{document}
    \begin{pb}
        Each \(F\)-Automorphism of \(K\) is an extension of the embedding \(F \to \falg\) which is identity on \(F\) to \(K\), and hence
        \(n = \# G \leq [K:F]_\text{sep} \leq [K:F]\). To prove the opposite inequality, consider \(\alpha_1, \hdots, \alpha_m \in K\) such that \(m > n\).
        Then the system of equations
        \begin{align*}
            &a_1\tau_1(\alpha_1) + a_2 \tau_1(\alpha_2) + \cdots + a_m \tau_1(\alpha_m) = 0 \\
            &\vdots \\
            &a_1\tau_n(\alpha_1) + a_2 \tau_n(\alpha_2) + \cdots + a_m \tau_n(\alpha_m) = 0
        \end{align*}
        With \(n\) equations, and \(m\) unknowns. Then the equation must have a (non-zero) solution, we wish to show that it has a solution lying in \(F\),
        which would prove that we cannot have more than \(n\) \(F\)-linearly independent elements of \(K\). Suppose that \((b_1,\hdots,b_m)\)
        is a solution with the most non-zero terms, WLOG we can take \(b_1 \neq 0\), and further take \(b_1 = 1\) by dividing the each \(b_i\) by \(b_1\).
        Then for each \(\tau_i\), since \(\tau_i(0) = 0\), and \(\tau_i\) permutes the other \(\tau_j\) since \(G\) is a group, we get that
        applying any \(\tau_i\) to the system of equations yields another solution, with the same zero terms. But then 
        \((1 - \tau_i(1),\hdots,b_m - \tau_i(b_m))\) is another solution, with the same zero coordinates as \(b\), along with the first coordinate being 0
        which by our minimality assumption implies that this is the zero vector. Since this holds for each \(\tau_i\), it follows that \(\tau_i(b_j) = b_j\) for each
        \(i,j\), so that each \(b_j\) is fixed by \(G\) and thus lies in \(F\), hence the first equation gives us that \(\alpha_1,\hdots,\alpha_m\) are \(F\)-linearly dependent,
        so no \(F\)-linearly independent set of \(K\) may have cardinality greater than \(n\), i.e. \([K:F] \leq n\). Since we have proven both inequalities, \([K:F] = n\).
    \end{pb}
    \begin{pb}
        Since \(K/F\) is finite, we may write it as \(K = F(\alpha_1, \hdots, \alpha_n)\). It is immediate that since \(\set{\alpha_1,\hdots,\alpha_n} \subset K \subset KL\), that we can write
        \(KL = L(\alpha_1,\hdots,\alpha_n)\), since this field contains \(F\) and each \(\alpha_i\) it must contain \(K\). Since each \(\alpha_i\) satisfies a polynomial with coefficients in
        \(F \subset L\), we know that \(KL/L\) is algebraic. To show that its seperable, note that \(\min(\alpha_i;L) \vert \min(\alpha_i;F)\), where \(\min(\alpha_i;F)\) contains no repeated roots, proving that
        each \(\alpha_i\) is seperable over \(L\). Finally, since \(K/F\) is normal of finite degree, we know that \(K\) is the splitting field of some polynomial \(f \in F[x]\), it is immediate that
        \(L(\alpha_1,\hdots,\alpha_n)\) is the splitting field of \(f\) over \(L\), so that \(KL/L\) is normal and hence Galois.

        Consider the map \(\pi: \gal(KL/L) \to \gal(K/(K\cap L))\), defined by \(\sigma \mapsto \sigma \vert_K\). It is clear that this map is well defined, and satisfies the homomorphism properties.
        To check this is an isomorphism, suppose that \(\sigma \in \ker \pi\), then \(\sigma \vert_K = 1\), hence \(\sigma(\alpha_i) = \alpha_i\) for each \(i\), furthermore \(\sigma\) fixes \(L\) by definition.
        It follows that for any \(x \in KL\), we have \(x = \sum_i \left(\ell_i \prod_{j}\alpha_j\right)\), so that 
        \[\sigma(x) = \sigma\left(\sum_i \left(\ell_i \prod_{j}\alpha_j\right)\right) = \sum_i \left(\sigma(\ell_i) \prod_{j}\sigma(\alpha_j)\right) = x\]
        implying that \(\sigma = 1\), so that this map is injective. To show surjectivity,
        let \(\tau \in \gal(K/(K\cap L))\), define \(\sigma(\alpha_i) = \tau(\alpha_i)\), this is a well defined extension of the identity map on \(L\), 
        since if \(\alpha_i, \alpha_j\) are conjugate over \(K \cap L\), then they are conjugate over \(L\). This can be seen since
        \[\min(\alpha;L) \vert \min(\alpha;K\cap L) \tand \min(\alpha;L) = (x - \alpha)(x-\beta_1)\cdots(x-\beta_k)\]
        for \(\beta_i \in K\), then the coefficients of \(\min(\alpha;L)\) are the symmetric polynomials in \(\alpha,\beta_1,\hdots,\beta_k\) so that they also lie in \(K\), so that in particular
        the coefficients lie in \(K \cap L\), this implies that \(\min(\alpha;L)\) is a polynomial with coefficients in \(K\cap L\) which is satisfied by \(\alpha\),
        implying that \(\min(\alpha;K\cap L) \vert \min(\alpha;L)\), so that in particular they are equal.
        Then by construction, we get \(\sigma\vert_K = \tau\) proving surjectivity. Since this is an ismomorphism between the two Galois groups it is also a bijection, so in particular
        \begin{align*}
            [KL:L] = \# \gal(KL/L) = \# \gal(K/K\cap L) = [K:K\cap L]
        \end{align*}
    \end{pb}
    \begin{pb}
        First note that
        \[\bigcap_{\sigma \in G}\sigma H \sigma^{-1} = \cap_{\sigma \in G}\gal{K/\sigma(L)}\]
        This follows since \(\tau \mapsto \sigma\tau\sigma^{-1}\) is a bijection from \(H\) to \(\gal{K/\sigma(L)}\).

        We first show that \(\gal(K/N) \subset \bigcap_{\sigma \in G}\gal(K/\sigma(L))\). Proof being, since \(N \supset L\) is normal, for any \(\sigma \in G\), we have 
        \(\sigma(L) \subset \sigma(N) = N\). Thus if \(\tau\vert_N = 1\), then \(\tau\vert_{\sigma(L)} = 1\) for any \(\sigma \in G\). This suffices to show that
        \(\tau \in \gal(K/N)\) implies \(\tau \in \bigcap_{\sigma \in G}\gal(K/\sigma(L))\).

        Now we show the other inclusion, namely \(\gal(K/N) \supset \bigcap_{\sigma \in G}\gal(K/\sigma(L))\). Since \(N\) is the compositum \(\gen{\sigma(L)}_{\sigma\in G}\)
        (proven below), we have that \(\tau \in \cap_{\sigma \in G}\gal{K/\sigma(L)}\), then \(\tau\vert_{N} = 1\), since for any \(\alpha\) in \(N\), we can write
        \(\alpha\) as a finite sum/product/quotient of elements in \(\bigcup_{\sigma\in G} \sigma(L)\), each of which is fixed by \(\tau\), so that \(\tau\) fixes \(\alpha\) by the
        homomorphism property. It follows that \(\tau \in \gal(K/N)\) proving that indeed \(N = \cap_{\sigma \in G}\gal{K/\sigma(L)} = \bigcap_{\sigma \in G}\sigma H \sigma^{-1}\).

        Proof of \(N = \gen{\sigma(L)}_{\sigma\in G}\): the right to left inclusion is obvious by normality, then by definition of normal closure, it will suffice to show that
        \(\gen{\sigma(L)}_{\sigma\in G}\) is normal. Let \(\tau \in G\), then we have that 
        \[\tau(\gen{\sigma(L)}_{\sigma\in G}) = \gen{\tau\sigma(L)}_{\sigma\in G} = \gen{\sigma(L)}_{\sigma\in G}\]
        The second equality follows since \(\tau: G \to G\) bijectively. The first equality follows from if \(\alpha\) in \(\gen{\sigma(L)}_{\sigma\in G}\), we can write
        \(\alpha\) as a finite sum/product/quotient of elements in \(\bigcup_{\sigma\in G} \sigma(L)\), by the homomorphism property, we apply \(\tau\) to each element in the
        sum/product/quotient so that \(\alpha\) is a finite sum/product/quotient of elements in \(\bigcup_{\sigma\in G} \tau\sigma(L)\)
    \end{pb}
    \begin{pb}
        Since \(K/F\) is Galois, and \(K \supset L_0 \supset F\), we have \(K/L_0\) is galois, with galois group \(N(H)\). Then since \(H\) is normal in \(N(H)\), we have
        \(L/L_0\) is galois. Furthermore, suppose that \(L \supset M \supset F\), with \(L/M\) Galois implying that \(H\) is normal in \(\gal(K/M)\). Then since the normalizer is the largest subgroup \(R\)
        of \(G\), such that \(H \subset R\) is normal, we get that \(\gal(K/M) \subset N(H)\), implying that \(M \supset L_0\) as desired.
    \end{pb}
    \begin{pb}
        We can define the map \(\varphi: \mathbb{Z}/2 \mathbb{Z} \overset{\varphi}{\to} (\mathbb{Z}/4 \mathbb{Z})\) as \(\varphi(1): x \mapsto -x\), this is a well defined
        automorphism, since \(\varphi(1)^2 = \mathbf{1}_{\mathbb{Z}/ 4 \mathbb{Z}} = \varphi(0) = \varphi(1 + 1)\). Any element \(x \in D_4\) can be written in the form of
        \(\sigma^i\tau^j\) using the relation \(\sigma\tau = \tau\sigma^{-1}\). So define the map
        \begin{align*}
            \psi: D_4 &\to \mathbb{Z}/4 \mathbb{Z} \underset{\varphi}{\rtimes} \mathbb{Z}/2 \mathbb{Z} \\
            \sigma^i\tau^j &\mapsto (i,j)
        \end{align*}
        is an isomorphism. \(\mathbf{1} \mapsto (0,0)\) is immediate. And (here I deal with both possible cases \(j = 1, 0\) seperately)
        \begin{align*}
            &\psi(\sigma^i\tau \sigma^k \tau^\ell) = \psi(\sigma^{i-k}\tau^{1+\ell}) = (i-k,1+\ell) = (i + \varphi(1)(k), 1 + \ell) = (i,1)(k,\ell) = \psi(\sigma^i\tau) \psi(\sigma^k \tau^\ell)\\
            &\psi(\sigma^i\tau^{0} \sigma^k \tau^\ell) = \psi(\sigma^{i+k} \tau^\ell) = (i + k, \ell) = (i + \varphi(0)(k), 0 + \ell) = (i,0)(k,\ell) = \psi(\sigma^i\tau^{0}) \psi(\sigma^k \tau^\ell)
        \end{align*}
        This proves that \(\psi\) is a homomorphism, and \[\psi(\sigma^i\tau^j) = (0,0) \iff i \equiv 0 \text{mod}4 \tand j \equiv 0 \text{mod}2 \iff \sigma^i\tau^j = \mathbf{1}\]
        proving that \(\ker \psi = \mathbf{1}\). Then since \(\# D_4 = \# \mathbb{Z}/4 \mathbb{Z} \underset{\varphi}{\rtimes} \mathbb{Z}/2 \mathbb{Z}\) and the map is injective, it must also be surjective.
    \end{pb}
\end{document}