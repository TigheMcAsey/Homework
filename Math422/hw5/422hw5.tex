\documentclass[11pt]{article}
\usepackage{amsmath, amsfonts, amssymb,amsthm}
\usepackage[includeheadfoot]{geometry} % For page dimensions
\usepackage{fancyhdr}
\usepackage{enumerate} % For custom lists

\fancyhf{}
\lhead{Math 422hw3}
\rhead{Tighe McAsey - 37499480}
\pagestyle{fancy}

% Page dimensions
\geometry{a4paper, margin=1in}

\theoremstyle{definition}
\newtheorem{pb}{}

% Commands:

\newcommand{\set}[1]{\{#1\}}
\newcommand{\abs}[1]{\lvert#1\rvert}
\newcommand{\norm}[1]{\lvert\lvert#1\rvert\rvert}
\newcommand{\gen}[1]{\left\langle #1 \right\rangle}
\newcommand{\tand}{\text{ and }}
\newcommand{\tor}{\text{ or }}
\newcommand{\falg}{F^{\text{alg}}}
\newcommand{\gal}{\text{Gal}}

\begin{document}
    \begin{pb}
        Each \(F\)-Automorphism of \(K\) is an extension of the embedding \(F \to \falg\) which is identity on \(F\) to \(K\), and hence
        \(n = \# G \leq [K:F]_\text{sep} \leq [K:F]\). To prove the opposite inequality, consider \(\alpha_1, \hdots, \alpha_m \in K\) such that \(m > n\).
        Then the system of equations
        \begin{align*}
            &a_1\tau_1(\alpha_1) + a_2 \tau_1(\alpha_2) + \cdots + a_m \tau_1(\alpha_m) = 0 \\
            &\vdots \\
            &a_1\tau_n(\alpha_1) + a_2 \tau_n(\alpha_2) + \cdots + a_m \tau_n(\alpha_m) = 0
        \end{align*}
        With \(n\) equations, and \(m\) unknowns. Then the equation must have a (non-zero) solution, we wish to show that it has a solution lying in \(F\),
        which would prove that we cannot have more than \(n\) \(F\)-linearly independent elements of \(K\). Suppose that \((b_1,\hdots,b_m)\)
        is a solution with the most non-zero terms, WLOG we can take \(b_1 \neq 0\), and further take \(b_1 = 1\) by dividing the each \(b_i\) by \(b_1\).
        Then for each \(\tau_i\), since \(\tau_i(0) = 0\), and \(\tau_i\) permutes the other \(\tau_j\) since \(G\) is a group, we get that
        applying any \(\tau_i\) to the system of equations yields another solution, with the same zero terms. But then 
        \((1 - \tau_i(1),\hdots,b_m - \tau_i(b_m))\) is another solution, with the same zero coordinates as \(b\), along with the first coordinate being 0
        which by our minimality assumption implies that this is the zero vector. Since this holds for each \(\tau_i\), it follows that \(\tau_i(b_j) = b_j\) for each
        \(i,j\), so that each \(b_j\) is fixed by \(G\) and thus lies in \(F\), hence the first equation gives us that \(\alpha_1,\hdots,\alpha_m\) are \(F\)-linearly dependent,
        so no \(F\)-linearly independent set of \(K\) may have cardinality greater than \(n\), i.e. \([K:F] \leq n\). Since we have proven both inequalities, \([K:F] = n\).
    \end{pb}
    \begin{pb}
        Since \(K/F\) is finite, we may write it as \(K = F(\alpha_1, \hdots, \alpha_n)\). It is immediate that since \(\set{\alpha_1,\hdots,\alpha_n} \subset K \subset KL\), that we can write
        \(KL = L(\alpha_1,\hdots,\alpha_n)\), since this field contains \(F\) and each \(\alpha_i\) it must contain \(K\). Since each \(\alpha_i\) satisfies a polynomial with coefficients in
        \(F \subset L\), we know that \(KL/L\) is algebraic. To show that its seperable, note that \(\min(\alpha_i;L) \vert \min(\alpha_i;F)\), where \(\min(\alpha_i;F)\) contains no repeated roots, proving that
        each \(\alpha_i\) is seperable over \(L\). Finally, since \(K/F\) is normal of finite degree, we know that \(K\) is the splitting field of some polynomial \(f \in F[x]\), it is immediate that
        \(L(\alpha_1,\hdots,\alpha_n)\) is the splitting field of \(f\) over \(L\), so that \(KL/L\) is normal and hence Galois.

        Consider the map \(\pi: \gal(KL/L) \to \gal(K/(K\cap L))\), defined by \(\sigma \mapsto \sigma \vert_K\). It is clear that this map is well defined, and satisfies the homomorphism properties.
        To check this is an isomorphism, suppose that \(\sigma \in \ker \pi\), then \(\sigma \vert_K = 1\), hence \(\sigma(\alpha_i) = \alpha_i\) for each \(i\), furthermore \(\sigma\) fixes \(L\) by definition.
        It follows that for any \(x \in KL\), we have \(x = \sum_i \left(\ell_i \prod_{j}\alpha_j\right)\), so that 
        \[\sigma(x) = \sigma\left(\sum_i \left(\ell_i \prod_{j}\alpha_j\right)\right) = \sum_i \left(\sigma(\ell_i) \prod_{j}\sigma(\alpha_j)\right) = x\]
        implying that \(\sigma = 1\), so that this map is injective. To show surjectivity,
        let \(\tau \in \gal(K/(K\cap L))\), define \(\sigma(\alpha_i) = \tau(\alpha_i)\), this is a well defined extension of the identity map on \(L\), 
        since if \(\alpha_i, \alpha_j\) are conjugate over \(K \cap L\), then they are conjugate over \(L\). This can be seen since
        \[\min(\alpha;L) \vert \min(\alpha;K\cap L) \tand \min(\alpha;L) = (x - \alpha)(x-\beta_1)\cdots(x-\beta_k)\]
        for \(\beta_i \in K\), then the coefficients of \(\min(\alpha;L)\) are the symmetric polynomials in \(\alpha,\beta_1,\hdots,\beta_k\) so that they also lie in \(K\), so that in particular
        the coefficients lie in \(K \cap L\), this implies that \(\min(\alpha;L)\) is a polynomial with coefficients in \(K\cap L\) which is satisfied by \(\alpha\),
        implying that \(\min(\alpha;K\cap L) \vert \min(\alpha;L)\), so that in particular they are equal.
        Then by construction, we get \(\sigma\vert_K = \tau\) proving surjectivity. Since this is an ismomorphism between the two Galois groups it is also a bijection, so in particular
        \begin{align*}
            [KL:L] = \# \gal(KL/L) = \# \gal(K/K\cap L) = [K:K\cap L]
        \end{align*}
    \end{pb}
    \begin{pb}
        First denote \(M := \text{Gal}(K/N)\). Since \(N\) is the smallest normal field extension of \(F\) containing \(L\), it must be the case that
        \(M\) is the largest subgroup of \(H\) which is normal in \(G\). As proof, assume there exists some normal subgroup \(R\), such that \(M \subsetneq R \subset H\).
        Then by the galois correspondence, \(N = K^M \supsetneq K^R \supset K^H = L\), where \(K^R/F\) is normal. But this contradicts \(N\) being the normal closure of
        \(L/F\). Now all that remains to show is that \(\bigcap_{\sigma \in G} \sigma H \sigma^{-1}\) is the largest subgroup of \(H\) which is normal in \(G\)
        it is a subgroup since it is the intersection of subgroups.
        To see that it is normal, for any \(\tau \in G\) 
        \begin{align*}
            \tau \left(\bigcap_{\sigma \in G} \sigma H \sigma^{-1}\right) \tau^{-1} = \bigcap_{\sigma \in G} \tau \sigma H \sigma^{-1} \tau^{-1}
            = \bigcap_{\sigma \in G} (\tau \sigma) H (\tau \sigma)^{-1} = \bigcap_{\sigma \in G} \sigma H \sigma^{-1}
        \end{align*}
        The last equality follows from \(\tau\) acting as a permutation on \(G\). To see its the largest, suppose that \(S \subset H\) is normal in \(G\). Then 
        \[S = \bigcap_{\sigma \in G} \sigma S \sigma^{-1} \subset \bigcap_{\sigma \in G} \sigma H \sigma^{-1}\]
    \end{pb}
    \begin{pb}
        Since \(K/F\) is Galois, and \(K \supset L_0 \supset F\), we have \(K/L_0\) is galois, with galois group \(N(H)\). Then since \(H\) is normal in \(N(H)\), we have
        \(L/L_0\) is galois. Furthermore, suppose that \(L \supset M \supset F\), with \(L/M\) Galois implying that \(H\) is normal in \(\gal(K/M)\). Then since the normalizer is the largest subgroup \(R\)
        of \(G\), such that \(H \subset R\) is normal, we get that \(\gal(K/M) \subset N(H)\), implying that \(M \supset L_0\) as desired.
    \end{pb}
    \begin{pb}
        We can define the map \(\varphi: \mathbb{Z}/2 \mathbb{Z} \overset{\varphi}{\to} (\mathbb{Z}/4 \mathbb{Z})\) as \(\varphi(1): x \mapsto -x\), this is a well defined
        automorphism, since \(\varphi(1)^2 = \mathbf{1}_{\mathbb{Z}/ 4 \mathbb{Z}} = \varphi(0) = \varphi(1 + 1)\). Any element \(x \in D_4\) can be written in the form of
        \(\sigma^i\tau^j\) using the relation \(\sigma\tau = \tau\sigma^{-1}\). So define the map
        \begin{align*}
            \psi: D_4 &\to \mathbb{Z}/4 \mathbb{Z} \underset{\varphi}{\rtimes} \mathbb{Z}/2 \mathbb{Z} \\
            \sigma^i\tau^j &\mapsto (i,j)
        \end{align*}
        is an isomorphism. \(\mathbf{1} \mapsto (0,0)\) is immediate. And (here I deal with both possible cases \(j = 1, 0\) seperately)
        \begin{align*}
            &\psi(\sigma^i\tau \sigma^k \tau^\ell) = \psi(\sigma^{i-k}\tau^{1+\ell}) = (i-k,1+\ell) = (i + \varphi(1)(k), 1 + \ell) = (i,1)(k,\ell) = \psi(\sigma^i\tau) \psi(\sigma^k \tau^\ell)\\
            &\psi(\sigma^i\tau^{0} \sigma^k \tau^\ell) = \psi(\sigma^{i+k} \tau^\ell) = (i + k, \ell) = (i + \varphi(0)(k), 0 + \ell) = (i,0)(k,\ell) = \psi(\sigma^i\tau^{0}) \psi(\sigma^k \tau^\ell)
        \end{align*}
        This proves that \(\psi\) is a homomorphism, and \[\psi(\sigma^i\tau^j) = (0,0) \iff i \equiv 0 \text{mod}4 \tand j \equiv 0 \text{mod}2 \iff \sigma^i\tau^j = \mathbf{1}\]
        proving that \(\ker \psi = \mathbf{1}\). Then since \(\# D_4 = \# \mathbb{Z}/4 \mathbb{Z} \underset{\varphi}{\rtimes} \mathbb{Z}/2 \mathbb{Z}\) and the map is injective, it must also be surjective.
    \end{pb}
\end{document}