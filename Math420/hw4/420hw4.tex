\documentclass[10.5pt]{article}
\usepackage{amsmath, amsfonts, amssymb,amsthm}
\usepackage[includeheadfoot]{geometry} % For page dimensions
\usepackage{fancyhdr}
\usepackage{enumerate} % For custom lists

\fancyhf{}
\lhead{Math 420hw3}
\pagestyle{fancy}

% Page dimensions
\geometry{a4paper, margin=1in}

\theoremstyle{definition}
\newtheorem{pb}{}

% Commands:

\newcommand{\set}[1]{\{#1\}}
\newcommand{\abs}[1]{\left\vert#1\right\vert}
\newcommand{\norm}[1]{\lvert\lvert#1\rvert\rvert}
\newcommand{\tand}{\text{ and }}
\newcommand{\tor}{\text{ or }}

\begin{document}
    \begin{pb}
        Let \(I = (a,b)\) be an open interval, then we have from equivalence to Riemann integral and FTC, we have
        \begin{align*}
            \abs{\int \chi_I e^{inx} dm} &= \abs{\int \chi_{\overline{I}} e^{inx} dm} = \abs{\int_a^b e^{inx} dx} = \abs{\frac{1}{in}(e^{inb} - e^{ina})} \leq \frac{2}{n}
        \end{align*}
        And hence taking the limit of both sides, we see that \(\lim_{n\to\infty}\int_\mathbb{R} \chi_I e^{inx} dm = 0\). 
        
        Let \(\epsilon > 0\) by the \(L^1\) approximation theorem (Folland 2.26), there exists
        a simple function \(\phi\), which as in 2.26 we take the indicators to be on open intervals, such that \(\int \abs{f - \phi}dm < \epsilon\). Then
        \begin{align*}
            \abs{\int f e^{inx}dm} \leq \abs{\int (f - \phi) e^{inx}dm} + \abs{\int \phi e^{inx}dm} \leq \int \abs{f - \phi} dm + \abs{\int \phi e^{inx}dm} < \epsilon + \abs{\int \phi e^{inx}dm}
        \end{align*}
        Unwrapping the definition of \(\phi\), and taking the limit on both sides gives the desired result.
        \begin{align*}
            \lim_{n\to\infty} \abs{\int f e^{inx}dm} < \epsilon + \lim_{n\to\infty}\abs{\int \sum_1^N a_j \chi_{I_j} e^{inx}dm} 
            \leq \epsilon + \sum_1^N a_j \lim_{n\to\infty}\abs{\int \chi_{I_j}e^{inx}dm} = \epsilon
        \end{align*}   
        And since \(\epsilon\) was arbitrary, it must be the case that \(\lim_{n\to\infty} \int f e^{inx}dm = 0\).
    \end{pb}
    \begin{pb}
        \textbf{(a)} For each \(k\), let \(E_k\) be a set such that \(f_n \rightrightarrows f\) on \(E_k^c\), and \(\mu(E_k) < \frac{1}{k}\), then it is immediate that
        \(f_n \to f\) on \(\left(\bigcap_1^\infty E_k \right)^c\), where \(E_1\) finally, monotonicity implies that \(\mu(\bigcap_1^\infty E_k) = 0\).
        Here is some elaboration, let \(x \in \left(\bigcap_1^\infty E_k \right)^c\), then \(x \in E_k^c\) for some \(k\), since \(f_n \to f\) on \(E_k^c\), it follows that \(f_n(x) \to f(x)\).

        \textbf{(b)} Let \(\hat{\epsilon}, \epsilon > 0\), then let \(N_k\) be the index given by uniform convergence \(f_n \rightrightarrows f\) on \(E_k^c\), such that \(n \geq N_k\) implies that
        \(\norm{f - f_n}_\infty < \epsilon\) on \(E_k^c\). It follows that for some natural number \(k\), \(\frac{1}{k} < \hat{\epsilon}\). Then for any \(n > N_k\), we have
        \(\norm{f - f_n}_\infty < \epsilon\) on \(E_k^c\), so that \(\set{x \vert \lim_{n\to\infty} \abs{f(x) - f_n(x)} \geq \epsilon} \subset E_k^c\), this implies by monotonicity that
        \(\mu\set{x \vert \lim_{n\to\infty} \abs{f(x) - f_n(x)} \geq \epsilon} < \hat{\epsilon}\). But since \(\hat{\epsilon}\) was arbitrary, 
        \(\mu\set{x \vert \lim_{n\to\infty} \abs{f(x) - f_n(x)} \geq \epsilon} = 0\) as desired. The proof is complete since \(\epsilon\) was arbitrary.
    \end{pb}
    \begin{pb}
        By Folland 2.30, some subsequence \(\set{f_{n_k}}_k \subset \set{f_n}_n\) converges pointwise to \(f\) almost everywhere. Then since
        \(\lim_{k\to \infty}\abs{f_{n_k}}  \leq g\), we have \(\abs{f} \leq g\) almost everywhere so that there is some \(h \in L^1\), such that \(h = g\) a.e. and \(\abs{f} \leq h\).
        It follows that for any \(n\), we have \(\abs{f - f_n} \leq \abs{f} + \abs{f_n} \leq g + h\).

        Let \(\epsilon > 0\), now since \(g, h \in L^1\), there exists some set \(K\) of finite measure, such that 
        \[\int h - h\chi_K d\mu < \epsilon/4 \tand \int g - g\chi_K d\mu < \epsilon/4\]
        (this is an immediate consequence of taking maximum support of either of the simple function approximations of \(g \tand h\)).
        It follows that
        \begin{align*}
            \int\abs{f - f_n}d\mu &\leq \int \abs{f-f_n}\chi_Kd\mu + \int \abs{f - f_n}\chi_{K^c}d\mu \\
            &\leq \int \abs{f-f_n}\chi_Kd\mu + \int \abs{f}\chi_{K^c}d\mu + \int\abs{f_n}\chi_{K^c}d\mu \\
            &\leq \int \abs{f-f_n}\chi_Kd\mu + \int h\chi_{K^c}d\mu + \int g\chi_{K^c}d\mu \\
            &< \int \abs{f-f_n}\chi_Kd\mu + \frac{1}{2}\epsilon
        \end{align*}
        Since \(g + h \in L^1\), we can apply problem 6 of last homework, which tells us that there is some \(\delta > 0\), such that \(\mu(E) < \delta\) implies that
        \(\int_E g+h d\mu < \frac{1}{4}\epsilon\). By convergence in measure, there exists some \(N\), such that for any \(n \geq N\) we have 
        \(\mu\set{x \vert \abs{f(x) - f_n(x)} \geq \frac{1}{4 \mu(K)}\epsilon} < \delta\). 
        Let \[E := \set{x \in K \vert \abs{f(x) - f_n(x)} \geq \frac{1}{4 \mu(K)}\epsilon}\]
        Then for any \(n \geq N\) we have (note that we take \(E^c = K \setminus E\))
        \begin{align*}
            \int \abs{f-f_n}d\mu &< \int \abs{f - f_n}\chi_K d\mu + \frac{1}{2}\epsilon \\
            &= \int \abs{f - f_n}\chi_E d\mu + \int \abs{f - f_n}\chi_{E^c} d\mu + \frac{1}{2}\epsilon \\
            &\leq \int (g + h)\chi_E d\mu + \int \abs{f - f_n}\chi_{E^c} d\mu + \frac{1}{2}\epsilon \\
            &< \int \abs{f - f_n}\chi_{E^c} d\mu + \frac{3}{4}\epsilon \\
            &\leq \int \frac{1}{4 \mu(K)}\epsilon \chi_{E^c}d\mu + \frac{3}{4}\epsilon \\
            &= \frac{\mu(E^c)\epsilon}{4 \mu(K)} + \frac{3}{4}\epsilon \leq \frac{\mu(K)\epsilon}{4 \mu(K)} + \frac{3}{4}\epsilon = \epsilon
        \end{align*}
        So that by definition \(\lim_{n\to\infty}\int \abs{f - f_n}d\mu = 0\)
    \end{pb}
    \begin{pb}
        We first show that \(D\) is in the product algebra. Define \(E_n := \bigcup_{\mathbb{Q}^2 \cap D}N_{1/n}(q)\)
        It is immediate that each \(E_n \in \mathcal{B}_{[0,1]}\cap \mathcal{P}([0,1]) = \mathcal{M}\). Then since \(\mathcal{M}\) is a sigma algebra, we have
        \(\cap_1^\infty E_n \in \mathcal{M}\). To see that \(\cap_1^\infty E_n = D\), first note that for any \(x \in D\), we have \(x = (\alpha,\alpha)\), then for any \(n\), by density
        of rationals, we have some \(q\), such that \(\abs{\alpha - q} < \frac{1}{2n}\), then it follows that \(d(x,(q,q)) = \sqrt{2\frac{1}{4n^2}} = \frac{1}{n\sqrt{2}} < 1/n\). And since this holds for
        each \(n\), we have \(x \in D\). Secondly since \(D\) is compact, any point in \(D^c\) has some positive distance away from \(D\), i.e. if \(x \in D^c\), then there exists some \(n\), such that
        \(d(x,D) \geq \frac{1}{n}\) implying that \(x \not \in E_n\), so that \(x \in \left(\bigcap_1^\infty E_n \right)^c\). This suffices to show \(D = \bigcap_1^\infty E_n\).

        For any \(x \in D\), we have by definition that \(\# D_x = 1\), so that we can write
        \begin{align*}
            \int \# D_x dm(x) = \int 1 dm = 1
        \end{align*}
        Whereas, for any \(y \in [0,1]\), we have that \(m(D^y) = 0\), so that
        \[\int m(D^y)d\#(y) = \int 0 d \#(y) = 0\]
        Finally, we compute \(m \times \# (D)\) from its definition as an outer measure. If \(\set{I_i}_1^\infty\) is any collection of rectangles covering \(D\). Define the map
        \(\pi: \begin{cases}
            (x,y) \mapsto 0 & x\neq y \\
            (x,x) \mapsto x
        \end{cases}\) then \(m\left(\pi\left(\bigcup_1^\infty I_i\right)\right) = 1 \leq \sum_1^\infty m(\pi(I_i))\), hence we may choose \(\ell\), such that \(m(\pi(I_\ell)) > 0\). This implies that
        \(I_\ell \cap D\) must be uncountable, so that \(m(I_\ell^x) > 0 \tand \# I_\ell^y = \infty\). 
        It follows that \[\sum_1^\infty m \times D (I_i) \geq m \times D (I_\ell) = m(I_\ell^x) \# I_\ell^y = \infty\]
        Now since \(\set{I_i}_1^\infty\) was an arbitrary collection, this proves the measure must be infinity by definition of the outer measure.
        
        These results are not inconsistent with Tolleni's theorem, since the conditions of the theorem are not met. Namely, \(([0,1],\#)\) is not sigma finite, since the only
        sets of finite measure are finite sets, but a countable union of at most countable sets is countable, so that no countable union of finite sets may cover \([0,1]\) which is uncountable.
    \end{pb}
    \begin{pb}
        \textbf{(a)}
        since \([0,1]\) is compact by the Heine Borel theorem, and \(F'\) is continuous on a compact set, we have some real number \(M\), such that
        \(\abs{F'} \leq M\) on  \([0,1]\). Then by the mean value theorem for vector valued functions, we have for any \(0 \leq x \leq y \leq 1\), that
        \(\abs{F(y) - F(x)} = \abs{(y-x)F'(c)}\), for \(c \in (x,y)\). It follows that \(\abs{F(y) - F(x)} \leq M\abs{y-x}\). Then for any \(N \in \mathbb{N}\),
        we may cover \(F[0,1]\) with
        \[R_N := \bigcup_{k=1}^N \prod_{i=1}^n [F(\frac{1}{k})_{x_i} - \frac{M}{N}, F(\frac{1}{k})_{x_i} + \frac{M}{N}]\]
        Where \([F(\frac{1}{k})_{x_i} - \frac{M}{N}, F(\frac{1}{k})_{x_i} + \frac{M}{N}]\) is a closed interval about the i-th coordinate of \(F(\frac{1}{k})\).
        Since this holds for any \(N\), in particular we have by monotonicity
        \begin{align*}
            m^n(F[0,1]) \leq m^n(R_N) = N \left(\frac{2M}{N}\right)^n = \left(\frac{2M}{N^{\frac{n-1}{n}}}\right)^n \underset{N \to \infty}{\longrightarrow} 0
        \end{align*}
        And since the inequality holds for all \(N\), it also holds for the limit

        \textbf{(b)}
        Let \(n \geq d > 1\) and \(\delta > 0\). Choose \(N > \frac{M2\sqrt{n}}{\delta}\) (where \(M\) is as in part (a)). Then as in part (a), we have that th we can cover \(\mathcal{C}\) with
        \(N\) rectangles \(\set{R_N^i}_1^N\), which are cartesian products of closed intervals length \(\frac{2M}{n}\) in each coordinate. Then we have
        \begin{align*}
            \text{Diam}R_N^i \leq \sqrt{\sum_{j=1}^n (\sup\set{\abs{z - y} \vert z,y \in (R_N^i)_{x_j}})^2} = \sqrt{\sum_1^n \left(\frac{2M}{N}\right)^2} = \frac{2M\sqrt{n}}{N} < \delta
        \end{align*}
        This implies that since \(H_{d,\delta}(\mathcal{C})\) comes from the infimum over such covers, we have for any \(N > \frac{M2\sqrt{n}}{\delta}\)
        \begin{align*}
            H_{d,\delta}(\mathcal{C}) \leq \sum_1^N \left(\frac{2M\sqrt{n}}{N}\right) = (2M\sqrt{n})^d N^{1-d}
        \end{align*}
        since this holds for all such \(N\), it also holds for the limit as \(N \to \infty\) so that (since \(d > 1\))
        \begin{align*}
            H_{d,\delta}(\mathcal{C}) \leq \lim_{N\to\infty} (2M\sqrt{n})^d N^{1-d} = 0
        \end{align*}
        And since this holds for any \(\delta > 0\), the right limit must be equal to zero. I.e. \(H_d(\mathcal{C}) = 0\).

        % \textbf{(a)} Denote \(\frac{d}{dx}F(x) = (f_1(x),\hdots,f_n(x))\), where it is clear by definition that each \(f_i\) is continuous. We know that \(F[0,1]\) is measurable, since it is the continuous image of
        % a compact set, hence compact, so that it is closed therefore in the Borel sets.
        % Furthermore, since \(f_1\) is continuous, \(E_0 := \set{x \in [0,1] \vert f_1(x) = 0}\) is closed and clearly bounded hence compact, 
        % so by the same logic, \(F(E_0)\) is Borel, then finally \(E_1 := \set{x \in [0,1] \vert f_1(x) \neq 0} = F(E) \cap F(E_0)^c\) is the intersection of two Borel sets (closure under compliments) and hence Borel.
        % It follows that \(m^n(F(E)) = m^n(F(E_0)) + m^n(F(E_1))\). I claim that \(F(E_1)_x \tand F(E_0)_x\) are countable unions of connected components. We first prove it for \(F_1\), where for any 
        % \(y \in F_1(E_1)\), we have \(x = F^{-1}(y)\), so that \(f_1(x) > 0\) (WLOG), then since \(f_1\) is continuous, there is some neighborhood of \(y\), such that \(f_1 > 0\).

        % Now since \(\mathbb{R} \tand \mathbb{R}^{n-1}\) are both \(\sigma\)-finite with respect to the Lebesgue measure, we have by the Tolleni theorem, that
        % \begin{align*}
        %     m^n(F[0,1]) = m^n(F(E_0)) + m^n(F(E_1)) = \int m(F(E_0)^x) dm^{n-1}(x) + \int m^{n-1}(F(E_1)^x) dm(x)
        % \end{align*}
    \end{pb}
    \begin{pb}
        We first verify that \(R\) is measurable for simple \(f\). Since \(f\) is simple, we may write \(f = \sum_1^N c_i \chi_{E_i}\) for disjoint \(E_i\). It is immediate that
        \(R = \bigcup_1^N E_i \times [0,c_i)\) which is a finite union of rectangles. We move on to the proof for general \(f\).

        Since \(f\) is measurable and positive, Folland 2.10 furnishes a sequence of simple functions \(\phi_n \nearrow f\). Defining \(R_n\) for each \(\phi_n\), then \(\bigcup_1^\infty R_n = R\)
        is a countable union of rectangles, thus is in the product sigma algebra. 
        
        It is easy to see that \(\bigcup_1^\infty R_n = R\), since for any \(x \in X\), 
        \(f(x) > \sup\set{y \vert (x,y) \in \bigcup_1^\infty R_n} \geq \phi_i(x), \forall i\) where \(\phi\nearrow f\). Furthermore, for any \((x,y) \in \bigcup_1^\infty R_n\), we have \((x,y) \in R_n\) for some \(n\), then since
        \(R_n\) was defined from a simple function, we have for any \(0 \leq z < y\), \((x,z) \in R_n \subset \bigcup_1^\infty R_n\).

        Finally, verifying the value of the integral is relatively easy. Since \((X,\mathcal{M},\mu), (\mathbb{R},\mathcal{B}_\mathbb{R},m)\) are \(\sigma\)-finite, we can apply Folland theorem 2.36
        \[\mu \times m(R) = \int_X m(R_x)d\mu\]
        But then, since for any \(x\), \(m(R_x) = [0,f(x))\) has lebesgue measure \(f(x)\),
        \[\mu \times m(R) = \int_X m(R_x)d\mu = \int_X f(x)d\mu\]
    \end{pb}
\end{document}