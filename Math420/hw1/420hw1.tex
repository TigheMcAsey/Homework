\documentclass[10.5pt]{article}
\usepackage{amsmath, amsfonts, amssymb,amsthm}
\usepackage[includeheadfoot]{geometry} % For page dimensions
\usepackage{fancyhdr}
\usepackage{enumerate} % For custom lists

\fancyhf{}
\lhead{Math 420hw1}
\rhead{Tighe McAsey - 37499480}
\pagestyle{fancy}

% Page dimensions
\geometry{a4paper, margin=1in}

\theoremstyle{definition}
\newtheorem{pb}{}

% Commands:

\newcommand{\set}[1]{\{#1\}}
\newcommand{\abs}[1]{\lvert#1\rvert}
\newcommand{\norm}[1]{\lvert\lvert#1\rvert\rvert}
\newcommand{\tand}{\text{ and }}
\newcommand{\tor}{\text{ or }}



\begin{document}
	\begin{pb}
		Let \((a,b)\) be an arbitrary open interval, then \((-\infty,b) \in \mathcal{B}\), furthermore
		\begin{align*}
			\set{(-\infty, a - 1/n)}_{n \in \mathbb{N}} \subset \mathcal{B} &\implies \set{[a - 1/n, \infty)}_{n \in \mathbb{N}} \subset \mathcal{B} \\
			&\implies (a,\infty) = \bigcup_{\mathbb{N}} [a-1/n,\infty) \in \mathcal{B} \\
			&\implies (a,b) = (a,\infty) \cap (-\infty, b) \in \mathcal{B}
		\end{align*}

		Now since each open interval is an open set we have that \(\mathcal{B} \subset \mathcal{B}_\mathbb{R}\). But since each open set is a countable union
		of open intervals it follows that each open set is in \(\mathcal{B}\), and hence by closure properties we have that
		the sigma algebra they generate must also be contained in \(\mathcal{B}\), so that \(\mathcal{B}_\mathbb{R} \subset \mathcal{B}\).
	\end{pb}

	\begin{pb}
		Countable sets. As proof, assume \(X\) is countable, then \(\set{\set{x}: x \in X} \subset \mathcal{P}(X)\), and each \(\set{x}\) has counting measure \(1\).
		It follows by assumption that \(\bigcup_{x\in X}\set{x} = X\) is a countable union of sets of finite measure, i.e. \(\sigma\)-finite.
		Conversely, if \(X\) is uncountable then it is not a countable union of countable sets, hence any countable collection of sets
		\(\set{X_i}_{i \in I}\), such that \(\bigcup_I X_i = X\) must have atleast one uncountable \(X_i\) (so that \(X_i\) has infinite counting measure).
	\end{pb}

	\begin{pb}
		\textbf{(a)} Suppose that \(E \in f_* \mathcal{M}\), then since \(f^{-1}(A^c) = f^{-1}(A)^c\),
		\begin{align*}
			f^{-1}(E) \in \mathcal{M} \implies (f^{-1}(E))^c = f^{-1}(E^c) \in \mathcal{M} \implies E^c \in f_* \mathcal{M}
		\end{align*}
		Now suppose that \(\set{E_i}_{i \in \mathbb{N}} \subset f_* \mathcal{M}\), then since \(\bigcup_\mathbb{N} f^{-1}(E_i) = f^{-1}(\bigcup_\mathbb{N} E_i)\),
		\begin{align*}
			\bigcup_\mathbb{N} f^{-1}(E_i) \in \mathcal{M} \implies f^{-1}(\bigcup_\mathbb{N} E_i) \in \mathcal{M} \implies \bigcup_\mathbb{N} E_i \in \mathcal{M}
		\end{align*}

		\textbf{(b)} We need only check that \(f_*\mu\) is a measure. Since the image of \(f_*\mu\) is a subset of the image of \(\mu\) it is clear 
		that for each \(E\), \(0 \leq f_*(\mu) \leq \infty\). It is also immediate that \(f_*\mu(\emptyset) = \mu(f^{-1}(\emptyset)) = \mu(\emptyset) = 0\).
        To check additivity, consider \(\set{E_i}_\mathbb{N}\subset f_*\mathcal{M}\), where the \(E_i\) 
		are disjoint (note this implies that each \(f^{-1}(E_i)\) is disjoint). It follows that
		\begin{align*}
			f_*\mu(\bigcup_\mathbb{N} E_i) = \mu(f^{-1}(\bigcup_\mathbb{N} E_i)) = \mu(\bigcup_\mathbb{N} f^{-1}(E_i)) = \sum_\mathbb{N} \mu(f^{-1}(E_i)) = \sum_\mathbb{N} f_*\mu(E_i)
		\end{align*}

		\textbf{(c)} The point mass at \(y_0\) let \(E \in f_*\mathcal{M}\), then
		\begin{align*}
			f_*\mu(E) = \begin{cases}
				\mu(\emptyset) = 0 & y_0 \not \in E \\
				\mu(f^{-1}(y_0)) = \mu(X) & y_0 \in E
			\end{cases}
		\end{align*}

		\textbf{(d)} This measure counts the number of perfect squares in a set \(E\). If \(E \subset \mathbb{N}\), then
		\(f_*\mu(E) = \# f^{-1}(E) = \# \set{n \in \mathbb{N}: n^2 \in E}\)
	\end{pb}
	\begin{pb}
		\(\mu\) is a measure for \(j \geq 0\).

		Suppose that \(j \leq -1\), then consider the sets \(E_n := \set{n^2}\).
        \begin{align*}\sum_\mathbb{N} \mu(E_n) = \sum_\mathbb{N} n^2 < \infty = \mu\left(\bigcup_\mathbb{N} E_n\right)\end{align*}

        Conversely, suppose that \(j \geq 0\), \(\mu(\emptyset) = 0\) by definition. If \(\set{E_i}_\mathbb{N}\) is a countable disjoint family,
        then we are done immediately if any \(E_i\) is infinite, since then \(\mu\left(\bigcup_\mathbb{N}E_i\right) = \infty = \mu(E_i) \leq \sum_\mathbb{N} \mu(E_i)\).
        Similarly, if infinitely many \(E_i\) are non-empty, then
        \begin{align*}
           \mu\left(\bigcup_\mathbb{N}E_i\right) = \infty = \sum_\mathbb{N} 1 = \sum_{\mathbb{N}} \sum_{n \in E_i} n^0 \leq \sum_{\mathbb{N}} \sum_{n \in E_i} n^j
        \end{align*}
        Finally, in the case where each \(E_i \) is finite, and only finitely many \(E_i \neq \emptyset\), we have for some \(N\), \(\set{E_i}_\mathbb{N} = \set{E_i}_{i=1}^N\), then
        \begin{align*}
            \mu\left(\bigcup_1^N E_i\right) = \sum_{n \in \bigcup_1^N E_i} n^j = \sum_{i =1}^N \sum_{n\in E_i} n^j = \sum_{i=1}^N \mu(E_i)
        \end{align*}
        where the second equality follows from the \(\set{E_i}_1^n\) being disjoint.
	\end{pb}
	\begin{pb}
		Note, for notational convenience I will use \(\ell:(a,b) \mapsto b-a\).

		\textbf{(a)} Let \(\epsilon > 0\), and consider the finite collection \(\set{E_i}_{i=1}^n \subset \mathcal{P}(\mathbb{R})\) and intervals \(\set{I_i^j}_{i,j=1}^{n,m}\), so that
		(note we may pick \(m\) independent of \(n\) by just choosing \(m\) to be the maximum number of intervals associated to any \(i \in \set{1,...,n}\))
		\begin{align*}
			\bigcup_{j=1}^m I_i^j \supset E_i, \tand \sum_{j=1}^m \ell(I_i^j) < J^*(E_i) + \frac{\epsilon}{n}, \; \; i \in \set{1,..,n}
		\end{align*}
		This gives us the desired result,
		\begin{align*}
			J^*(\bigcup_{i=1}^n E_i) \leq \sum_{i,j=1}^{n,m} \ell(I_i^j) = \sum_{i=1}^n\sum_{j=1}^m \ell(I_i^j) < \sum_{i=1}^n J^*(E_i) + \frac{\epsilon}{n} = \left(\sum_{i=1}^n J^*(E_i)\right) + \epsilon
		\end{align*}
		And since \(\epsilon\) was arbitrary, this proves finite subadditivity. To show that \(J^*\) is not countably subbadditive, notice that
		for any rational number \(q, \; J^*(q) = 0\), since \((q-\epsilon/2,q+\epsilon/2)\) covers \(q\) for any \(\epsilon > 0\). We may enumerate
		\(\mathbb{Q}\cap[0,1] = \set{q_1,q_2,...}\), so that applying part (b),
		\begin{align*}
			J^*(\mathbb{Q}\cap[0,1]) = 1 > 0 = \sum_{i=1}^\infty 0 = \sum_{i=1}^\infty J^*(q_i)
		\end{align*}

		\textbf{(b)} Assume for the sake of contradiction that \(J^*(\mathbb{Q}\cap[0,1]) < 1\), then there exists some collection 
		\(\set{I_i}_{i=1}^n, \; I_i = (a_i,b_i)\) covering \(\mathbf{Q} := \mathbb{Q}\cap[0,1]\), such that \(\sum_{i=1}^n \ell(I_i) < 1\). We may reindex this collection, so that
		\(a_i \leq a_{i+1}\) (note that this implies \(a_1 < 0\)), and assume WLOG that \(b_i < b_{i+1}\), otherwise \(\bigcup_{1 \leq j \leq n, \; j \neq i} I_j\) still covers \(\mathbf{Q}\), and
		\(\sum_{1 \leq j \leq n, \; j \neq i} I_j \leq \sum_{1 \leq j \leq n} I_j\), 
		we may also assume \(b_1 > 0 \tand a_n < 1\), otherwise \(\mathbf{Q} \subset \bigcup_{i=2}^n I_i \tor \mathbf{Q}\subset\bigcup_{i=1}^{n-1} I_i\).
		Finally, we have \(a_i \leq b_{i-1}\), otherwise \(\emptyset \neq (b_{i-1},a_i) \subset [0,1]\), and since \(\mathbb{Q}\) is dense in \(\mathbb{R}\),
		\begin{align*}
			\emptyset \neq \mathbb{Q}\cap(b_{i-1},a_i) = \mathbf{Q}\cap(b_{i-1},a_i) \subset (\bigcup_{i=1}^n I_i)^c \implies \mathbf{Q} \not \subset \bigcup_{i=1}^n I_i
		\end{align*}
		Using the above results, we get
		\begin{align*}
			&1 > \sum_{i=1}^n b_i - a_i \geq \sum_{i=1}^n b_i - b_{i-1} = b_n - b_1 \geq b_n - a_1 > b_n
		\end{align*}
		And hence \(1 \in (\bigcup_{i=1}^n I_i)^c\), a contradiction. \newline
		
		\textbf{(c)} Consider any interval \(I\), if \(I \not \subset [0,1]\), then \(I \not \subset \mathbf{Q} := \mathbb{Q}\cap[0,1]\). If
		\(I \subset [0,1]\), then since the irrationals are dense in \(\mathbb{R}\), there exists some \(\alpha \in I \cap \mathbb{Q}^c \supset \mathbf{Q}^c\).
		This proves that there are no non-empty open intervals which are subsets of \(\mathbf{Q}\), hence if \(\set{I_i}_{i=1}^n\) is a collection of open
		intervals, such that \(\bigcup_1^n I_i \subset \mathbf{Q}\), then each \(I_i\) must be empty, implying that \(\sum_1^n \ell(I_i) = 0\), so that
		\(J_*(\mathbf{Q}) = 0 \neq 1 = J^*(\mathbf{Q})\), i.e. \(\mathbf{Q}\) is not Jordan measureable.
	\end{pb}
	\begin{pb}
		First note that from monotonicity, \(\mu^*(E) \leq \mu^*(\tilde{E}) = \mu(\tilde{E})\) for any \(\tilde{E}\supset E\). Hence we need only show existence of some \(\tilde{E} \supset E\), such that
		\(\mu(\tilde{E}) \leq \mu^*(E)\). If \(\mu^*(E) = \infty\) we are done trivially with \(\tilde{E} = X\), so assume not. Define \(I_n := \bigcup_{i=1}^\infty A_i \supset E\), 
		such that for each \(i, \; A_i \in \mathcal{A}\)
		and \(\sum_{i=1}^\infty \mu_0(A_i) = \sum_{i=1}^\infty \mu(A_i) < \mu^*(E) + \frac{1}{n}\); existence of such \(A_i\) is guarunteed by the definition of \(\mu^*\).
		It follows that since \(\mathcal{A} \subset \mathcal{M}\) (Folland 1.13), each \(I_n \in \mathcal{M}\) by the property of \(\mathcal{M}\) being an algebra. Then we have 
		\begin{align*}
			&E \subset \tilde{E} := \cap_{n=1}^\infty I_n \in \mathcal{M}\\
			\implies &\mu^*(E) \leq \mu^*(\tilde{E}) = \mu(\tilde{E}) \leq \mu(I_n), \; \forall n \\
			\implies &\mu^*(E) \leq \mu(\tilde{E}) \leq \mu^*(E) + \frac{1}{n}, \; \forall n \\
			\implies &\mu(\tilde{E}) = \mu^*(E)
		\end{align*}
	\end{pb}
\end{document}