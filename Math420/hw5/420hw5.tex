\documentclass[10.5pt]{article}
\usepackage{amsmath, amsfonts, amssymb,amsthm}
\usepackage[includeheadfoot]{geometry} % For page dimensions
\usepackage{fancyhdr}
\usepackage{enumerate} % For custom lists

\fancyhf{}
\lhead{Math 420hw3}
\pagestyle{fancy}

% Page dimensions
\geometry{a4paper, margin=1in}

\theoremstyle{definition}
\newtheorem{pb}{}

% Commands:

\newcommand{\set}[1]{\{#1\}}
\newcommand{\abs}[1]{\left\vert#1\right\vert}
\newcommand{\norm}[1]{\lvert\lvert#1\rvert\rvert}
\newcommand{\tand}{\text{ and }}
\newcommand{\tor}{\text{ or }}
\newcommand{\floor}[1]{\left\lfloor #1 \right\rfloor}
\newcommand{\ceil}[1]{\left\lceil #1 \right\rceil}

\begin{document}
    \begin{pb}
        \textbf{(a)} Suppose that \(E\) is not \(\nu\)-null, then there exists some \(A \subset E\), such that \(\nu^+(A) \neq 0\) (choosing + WLOG), so that
        \[\abs{\nu}(E) \geq \nu^+(E) \geq \nu^+(A) > 0\]
        Conversely, suppose that \(E\) is \(\nu\)-null, then for any \(A \subset E\) measurable, \(\nu(A) = 0 \implies \nu^+(A) = \nu^-(A) = 0\), hence by the Hahn decomp, we have
        \(\nu(P) = \nu(N) = 0, \; E = P \cup N\), which gives us \(\abs{\nu}(E) = \nu^+(E) + \nu^-(E)= \nu(P) + \nu(N) = 0\).

        \textbf{(b)} By definition of mutually singular, let \(E,F\) be sets such that \(E \cup F = X\), and \(E\) is \(\nu\)-null,
        \(F\) \(\mu\)-null.

        [\(\nu \perp \mu \implies \abs{\nu}\perp\mu\)]: by part (a), \(\abs{\nu}(E) = 0\), and since \(\abs{\nu}\) is monotonic, \(E\) is \(\abs{\nu}\)-null.

        [\(\abs{\nu} \perp \mu \implies \nu^+ \perp \mu \tand \nu^- \perp \mu\)]: (Redefine \(E,F\) for \(\abs{\nu},\mu\)) Then for any measurable \(A \subset E\), we have
        \(\abs{\nu}(A) = 0 \implies \nu^+(A) = \nu^-(A) = 0\), implying that \(E\) is a null set for both \(\nu^+ \tand \nu^-\).

        [\(\nu^+ \perp \mu \tand \nu^- \perp \mu \implies \nu \perp \mu\)] Let \(E^+,F^+ \tand E^-,F^-\) be the sets furnsihed by mutual singularity for \(\nu^+,\nu^-\) with \(\mu\) respectively.
        Then \(X = (E^+ \cap E^-) \cup (F^- \cup F^+)\), for \(F^- \cup F^+\) \(\mu\)-null, and \(E^+ \cap E^-\) \(\nu^+ \tand \nu^-\) null. Then for any \(A \subset E^+ \cap E^-\), we have
        \(\nu(A) = \nu^+(A) - \nu^-(A) = 0\), hence the sets \(F^- \cup F^+\) and \(E^+ \cap E^-\) witness the mutual singularity of \(\nu \tand \mu\).

        \textbf{(c)} Take \(\nu(E) := \int_E \frac{1}{\abs{x}}d\mu\), then for any \(\delta > 0\), we have \(m(0,\delta) \leq \delta\), but \(\nu(0,\delta) = \infty\). This can be seen since
        we have \(f \geq \sum_1^\infty \chi_{0,1/N}\), so that for any natural \(K\) such that \(K < \delta\)
        \begin{align*}
            \int_{(0,\delta)} f dm &\geq \int_{(0,\frac{1}{K}]} f dm \geq \int_{(0,\frac{1}{K}]} \sum_1^\infty \chi_{(0,1/N)} dm \\
            &\overset{\text{MCT}}{=} \lim_{M\to\infty}\sum_1^M \int_{(0,\frac{1}{K}]} \chi_{(0,1/N)} dm = \lim_{M\to\infty} \sum_K^M \frac{1}{N} = \infty
        \end{align*}
    \end{pb}
    \begin{pb}
        \textbf{(a)} For \(x \geq 0\), \(F\) is the product of two increasing functions hence increasing. And \(F(x) < 0\) iff \(x < 0\), so we only need to check increasing on \((-\infty,0)\).
        Now suppose \(y < x < 0\), then \(\floor{y} < \floor{x} < 0 < \abs{x} < \abs{y}\) implying that \(\floor{y}\abs{y} < \floor{x}\abs{x}\). For right continuity, we have shown in previous homeworks
        that \(\floor{x}\) is right continuous, and since \(\abs{x}\) is continuous it is also right continuous, so right continuity follows by the product of right continuous functions being right continuous.

        \textbf{(b)} Take \(d\rho = \abs{\floor{x}} dm\) and \(\lambda\) as the interal of \(\abs{x}\) with respect to the counting measure on \(\mathbb{Z}\).
        It is immediate that these are measures (integral of a measurable function defines a measure as in hw3 p1), it is also immediate that both are \(\sigma\)-finite (we already know \(m\) is \(\sigma\)-finite), 
        using \(\mathbb{R} = \bigcup_1^\infty [-N,-N+1]\cup[N-1,N]\).  Furthermore we have \(m(\mathbb{Z}) = 0\) so that \(\lambda \perp m\). 
        Since the borel sets are generated by h-intervals, we need only show equivalence of \(m^F\) with \(\rho + \lambda\) on h-intervals to show that the measures are equal, by the extension theorem.
        First let \(N \in \mathbb{Z}\), then
        \begin{align*}
            \rho+\lambda(N,N+1] &= \int_{(N,N+1]} \abs{\floor{x}} dm +\int_{(N,N+1]} \abs{x} d\#_\mathbb{Z} = \int \abs{N} \chi_{[N-1,N]} dm + \abs{N+1} = \abs{N} + \abs{N+1} \\
            &= N(\abs{N+1}-\abs{N}) + \abs{N+1} = \abs{N+1}(N+1) - \abs{N}(N) = F(N+1) - F(N)
        \end{align*}
        Now let \(a<b \in \mathbb{R}\), then (here we use the idenitity for integers proved above to telescope), then
        \begin{align*}
            (\rho+\lambda)(a,b] &= (\rho+\lambda)(a,\ceil{a}] + \sum_{k \in (a,b-1]\cap \mathbb{Z}} (\rho+\lambda)(k,k+1] + (\rho+\lambda)(\floor{b},b] \\
            &= (\rho+\lambda)(a,\ceil{a}] + F(\floor{b}) - F(\ceil{a}) + (\rho+\lambda)(\floor{b},b] \\
            &= \int_{(a,\floor{a}+1]}\abs{\floor{x}}dm + \abs{\floor{a} + 1} + F(\floor{b}) - F(\ceil{a}) + \int_{\floor{b},b} \abs{\floor{x}} dm\\
            &= \int \chi_{(a,\floor{a}+1]}\abs{\floor{a}}dm + \abs{\floor{a} + 1} + F(\floor{b}) - F(\ceil{a}) + \int \chi_{(\floor{b},b]} \abs{\floor{b}} dm\\
            &= \abs{\floor{a}}(\floor{a} + 1 - a) + \abs{\floor{a} + 1} + F(\floor{b}) - F(\ceil{a}) + \abs{\floor{b}}(b - \floor{b})\\
            &= \abs{\floor{a}+1}(\floor{a}+1) - \abs{a}\floor{a} + F(\floor{b}) - F(\ceil{a}) + \floor{b}(\abs{b} - \abs{\floor{b}})\\
            &= F(\floor{a}+1) - F(a) + F(\floor{b}) - F(\ceil{a}) + F(b) - F(\floor{b})\\
            &= F(b) - F(a)
        \end{align*}
    \end{pb}
    \begin{pb}
        \textbf{(a)}
        \[\mu(E) = 0 \implies E = \emptyset \implies m(E) = 0\]
        And for any \(f\), we have
        \[\int_{\set{0}}f dm \leq 0 \cdot \infty = 0 < \mu(\set{0})\]

        \textbf{(b)} Suppose for contradiction that \(\rho, \lambda\) are such that \(\mu = \lambda+\rho \tand \rho << m, \lambda \perp m\). Then \(d\rho = f dm\) for some \(f\).
        It follows that if \(x \in [0,1]\), then \(1 = \mu\set{x} = (\rho + \lambda)\set{x} = 0 + \lambda(\set{x})\) hence \(\lambda(x) = 1\), for each \(x \in [0,1]\). 
        This contradicts our assumption of \(\lambda \perp m\), since the only null set for \(\lambda\) is \(\emptyset\), but \(m(\emptyset^c) = m([0,1]) = 1\), 
        meaning the decomposition into mutually singular sets is not possible.

        \textbf{(c)} \(\mu\) is not \(\sigma\)-finite, proof being that \(\mu\) is finite only on finite sets, but a countable union of finite sets is at most countable.
        But \([0,1]\) is uncountable, hence \([0,1]\) is not a countable union of \(\mu\)-finite sets.
    \end{pb}
    \begin{pb}
        Since \(f\) is not almost everywhere 0, we can apply continuity from below so that 
        \[0 < C = \abs{f} dm^n (\mathbb{R}^n) = \lim_{N\to\infty}\abs{f} dm^n (B_N(0))\]
        I.e. we may choose \(N\) large enough so that \(\abs{f} dm^n (B_N(0)) \geq C/2\).
        Now choose \(R = N\), so that for any \(x\) such that \(\abs{x} \geq R\), we have \(B_R(0) \subset B_{2\abs{x}}(x)\). This gives us the following inequality (denote \(s = m^n(B_1)\))
        \begin{align*}
            Hf(x) \geq \frac{1}{m^n(B_{2\abs{x}}(x))}\int_{B_{2\abs{x}}(x)} \abs{f} dm^n \geq \frac{1}{m^n(B_{2\abs{x}}(x))} \int_{B_R{0}} \abs{f} dm^n
            \geq \frac{2^{-n}\abs{x}^{-n}}{s} C/2
        \end{align*}
        so that choosing \(c = \frac{C}{2^{n+1}}\) gives the desired result.

        \textbf{(a)} Use the above to choose \(R,c\) based on \(f\) (still name \(s = m(B_1)\)). Then, rename \(R\) as the smallest natural number larger than \(R\).
        \begin{align*}
            \int Hf(x) dm^n &\geq \int_{B_R(0)^c} Hf(x) dm^n \geq \int_{B_R(0)^c} c\abs{x}^{-n} dm^n \geq c\int_{B_R(0)^c} dm^n \\
            &= \geq c\int \sum_{N=R}^\infty (N+1)^{-n}\chi_{(N,N+1]} dm^n \overset{\text{MCT}}{=} c\sum_{N=R}^\infty \int (N+1)^{-n}\chi_{(N,N+1]} dm^n \\
            &= c\sum_{N=R}^\infty \left(\int_{B_{N+1}} (N+1)^{-n} dm^n - \int_{B_{N}} (N+1)^{-n} dm^n \right) = c\sum_{N=R}^\infty s - s\left(\frac{N}{N+1}\right)^n \\
            &\geq cs\sum_{N=R}^\infty \frac{N^n + (N+1)^{n-1} - N^n}{(N+1)^n} = cs\sum_{N=R+1}^\infty \frac{1}{N} = \infty
        \end{align*}

        \textbf{(b)} Take \(c,R\) as in the problem statement and \(s\) as above, then for any \(x\) such that \(R < \abs{x} \leq kR\) (\(k \geq 2\)) we have \[Hf(x) \geq c(kR)^{-n} > c(2kR)^{-n}\], it follows that
        \begin{align*}
            m\set{x\vert Hf(x) > \frac{c}{(2kR)^n}} \geq m(B_{kR}) - m(B_R) = (k^n-1)R^ns
            \geq \frac{(kR)^ns}{2} = \frac{(2kR)^n}{c}\left(\frac{s}{c2^{n+1}}\right)
        \end{align*}
        Taking \(\alpha = \frac{c}{(2kR)^n}, \; c' = \left(\frac{s}{c2^{n+1}}\right)\) we get for any
        \(\alpha < \frac{c}{(4R)^n}\) that
        \[m\set{x\vert Hf(x) > \alpha} \geq \frac{c'}{\alpha}\]
        since we proved the inequality for each \(k \geq 2\).
    \end{pb}
    \begin{pb}
        \textbf{(a)} Let \(R > 0\), and \(\epsilon > 0\), then we have in general (wlog \(x < y\))
        \begin{align*}
            \abs{F(y) - F(x)} = \int_x^y f(t)dm
        \end{align*}
        Since \(f \in L^1_{\text{loc}}\), for any \(E \subset [-R,R]\), we have from hw3 p6 some \(\delta > 0\), such
        that \(m(E) < \delta\) implies that \(\int{E}\abs{f} dm < \epsilon\). Now let \(x \in [-R,R]\), then for any
        \(y \in N_\delta(x)\cap[-R,R]\), we have (wlog \(x < y\), this only affects how we write the integral)
        \begin{align*}
            \abs{F(y) - F(x)} = \abs{\int_x^y f(t)dm} \leq \int_x^y \abs{f(t)}dm < \epsilon
        \end{align*}

        \textbf{(b)}
        \begin{align*}
            f := \chi_{(0,\infty)} + \frac12 \chi_{\set{0}}
        \end{align*}
        Then for any \(r > 0\),
        \begin{align*}
            A_rf(0) = \frac{1}{2r}\int_{[-r,r]} \chi_{(0,\infty)} + \frac12 \chi_{\set{0}} dm
            = \frac{1}{2r}\int \chi_{(0,r]} = \frac{r}{2r} = \frac12 = f(0)
        \end{align*}
        Since this holds for all \(r\), it also holds for the limit.
        Then for \(x > 0\), we have 
        \[F(x) = \int_{[0,x]}f(t)dm = \int \chi_{[0,x]}dm = x\]
        and \(x < 0\) we have
        \[F(x) = -\int_{[0,x]}f(t)dm = \int 0 dm = 0\]
        Hence \(F\) is not differentiable at \(0\), since its left derivative at \(0\) is \(0\), and right derivative at \(0\) is 1.
        
        \textbf{(c)} First note I will drop the \(dm \; / \; dx\) in the integrals for brevity. I will use the fact that a function is differentiable at a point if and only if its left and right derivatives exist and are equal. Given this it is only necessary to construct \(f\), such that
        \(F\) is right differentiable (at 0) with derivative \(0\), it will then suffice to take
        \(f(-x) = f(x)\) for \(x > 0\). Now suppose that \(f\) is an \(L^1_{\text{loc}}\) function, symmetric about the origin, such that
        \(F'(0)\) exists and is equal to \(0\) (here the right derivative is sufficient, since by symmetry the left derivative will also be \(0\), and we use the fact that left and right derivative agreeing implies existence of derivative)  and \(\lim_{r\to0+}\frac{1}{r}\int_0^r\abs{f} = c > 0\), then we have
        \begin{align*}
            \lim_{r\to0+}A_r\abs{f-F'(0)}(0) &= \lim_{r\to0+}A_r\abs{f}(0) 
            = \lim_{r\to0+}\frac{1}{2r}\int_0^r \abs{f} + \lim_{r\to 0 +}\frac{1}{2r} \int_{-r}^0 \abs{f} \\
            &= \frac22 \lim_{r\to0+} \int_0^r \abs{f} = c
        \end{align*}
        Thus reducing the problem to finding an \(L^1_\text{loc}\) function, on \([0,\infty)\), such that the right derivative
        \[F'(0) = \lim_{r\to0+}\frac{1}{r}\int_0^r f(x) = 0\]
        and
        \[\lim_{r\to0+}\frac{1}{r}\int_0^r\abs{f} = c > 0\]
        Before providing the function, I will prove a lemma which will be used a number of times

        \textbf{Lemma} \(\frac{1}{N} \leq \sum_N^\infty \frac{1}{k^2} < \frac{1}{N+1}\). Proof being,
        \begin{align*}
            \frac{1}{k} - \frac{1}{k+1} = \frac{1}{k(k+1)} \leq \frac{1}{k^2} \leq \frac{1}{k(k-1)} = \frac{1}{k-1} - \frac{1}{k}
        \end{align*}
        The sum of the left and right telescope to \(1/N\) and \(1/(N-1)\) respectively, when we sum over all three terms.

        Now define the following,
        \begin{align*}
            &f = \sum_1^\infty a_{2n}\chi_{(\frac{1}{2n},\frac{1}{2n+1}]} + a_{2n+1}\chi_{(\frac{1}{2n+1},\frac{1}{2n}]} \\
            &a_{2n} = \frac{2n(2n-1)}{n^2}, \quad a_{2n+1} = -\frac{2n(2n-1)}{n^2} \\
        \end{align*}
        also note that 
        \[\int_{(\frac{1}{2n},\frac{1}{2n+1}]} f = \int_{(\frac{1}{2n},\frac{1}{2n+1}]} \abs{f} = \frac{1}{n^2}\]
        \[\int_{(\frac{1}{2n+1},\frac{1}{2n}]} f = -\frac{1}{n^2} = - \int_{(\frac{1}{2n+1},\frac{1}{2n}]}\abs{f}\]

        Now all that remains to prove is that \(f\) actually satisfies the properties from above.
        Firstly, we show that \(f\) is \(L^1\) (by DCT this means we also need not worry about switching integrals and summations), this is straightforward since \(\abs{f} \leq 4\chi_{[0,1]}\).        
        Suppose that
        \(0 < r < \frac12\), then there exists some \(N \in \mathbb{N}\), such that
        \[\frac{1}{2N} \geq r \geq \frac{1}{2N+2} \implies 2N \leq \frac{1}{r} \leq 2N+2\]
        note that taking \(r \to 0+\) is taking \(N \to \infty\). First we show the right derivative is \(0\):
        \begin{align*}
            \abs{\frac{1}{r}\int_0^r f} &\leq (2N+2)\abs{\int_0^r f} \leq (2N+2)\left(\abs{\int_0^{\frac{1}{2N+2}}f}+\abs{\int_{\frac{1}{2N+2}}^r f}\right) \\
            &\leq (2N+2)\left(\abs{\int_0^{\frac{1}{2N+2}}f}+\int_{\frac{1}{2N+2}}^r \abs{f}\right)
            \leq (2N+2)\left(\abs{\int_0^{\frac{1}{2N+2}}f}+\int_{\frac{1}{2N+2}}^{\frac{1}{2N}} \abs{f}\right) \\
            &\leq (2N+2)\left(\abs{-\int a_{2N+1}\chi_{(\frac{1}{2N+2},\frac{1}{2n+1}]} + \sum_{N+1}^\infty \int a_{2n} \chi_{[\frac{1}{2n},\frac{1}{2n-1})}
            + a_{2n+1}\chi_{(\frac{1}{2n+1},\frac{1}{2n}]}} + \int_{\frac{1}{2N+2}}^{\frac{1}{2N}} \abs{f}\right) \\
            &= (2N+2)\left(\abs{-\frac{1}{(N+1)^2} + \sum_{N+1}^\infty\frac{1}{n^2} - \frac{1}{n^2}}+ \int_{\frac{1}{2N+2}}^{\frac{1}{2N}} \abs{f} \right) \leq \frac{2N+2}{N^2} + (2N+2)\int_{\frac{1}{2N+2}}^{\frac{1}{2N}} \abs{f} \\
            &= \frac{2N+2}{N^2} + (2N+2)\int \abs{a_{2N+2}}\chi_{(\frac{1}{2N+2},\frac{1}{2N+1}]}+ \abs{a_{2N+1}}\chi_{(\frac{1}{2N+1},\frac{1}{2N}]} \\
            &= \frac{2N+2}{N^2} + (2N+2)(\frac{1}{(N+1)^2} + \frac{1}{N^2}) \leq \frac{2N+2}{N^2} + \frac{4N+4}{N^2} \leq \frac{12N}{N^2} = \frac{12}{N}
        \end{align*}
        Which goes to zero as \(N \to \infty\), which happens as \(r \to 0\). Now we show that
        \(\lim_{r\to0+}\frac{1}{r}\int_0^r \abs{f} = 4\), using the same setup for \(r,N\) as previously
        \begin{align*}
            \frac{1}{r}\int_0^r\abs{f} &\geq 2N\int_0^{2N+2}\abs{f} \geq 
            2N \sum_{N+2}^\infty \int_{\frac{1}{2n}}^{\frac{1}{2n-1}}\abs{f} + \int_{\frac{1}{2n+1}}^{\frac{1}{2n}} \abs{f}= (2N)\sum_{N+2}^\infty \frac{1}{n^2} + \frac{1}{n^2} \\
            &\geq 2N(\frac{2}{N+2}) = \frac{4N}{N+2}
        \end{align*}
        and similarly,
        \begin{align*}
            \frac{1}{r}\int_0^r\abs{f} &\leq (2N+2)\int_0^{\frac{1}{2N}}\abs{f} 
            \leq (2N+2)\sum_N^\infty \int_{\frac{1}{2n}}^{\frac{1}{2n-1}}\abs{f} + \int_{\frac{1}{2n+1}}^{\frac{1}{2n}} \abs{f} \\
            &= 4(N+1)\sum_N^\infty \frac{1}{n^2} \leq \frac{4(N+1)}{N-1}
        \end{align*}
        As \(r \to 0\), \(N \to \infty\) and both of these bounds go to \(4\). The squeeze theorem implies that
        \(\lim_{r\to0+}\frac{1}{r}\int_0^r \abs{f} = 4\) as desired.
    \end{pb}
\end{document}