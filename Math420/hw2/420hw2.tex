\documentclass[10.5pt]{article}
\usepackage{amsmath, amsfonts, amssymb,amsthm}
\usepackage[includeheadfoot]{geometry} % For page dimensions
\usepackage{fancyhdr}
\usepackage{enumerate} % For custom lists

\fancyhf{}
\lhead{Math 420hw2}
\rhead{Tighe McAsey - 37499480}
\pagestyle{fancy}

% Page dimensions
\geometry{a4paper, margin=1in}

\theoremstyle{definition}
\newtheorem{pb}{}

% Commands:

\newcommand{\set}[1]{\{#1\}}
\newcommand{\abs}[1]{\lvert#1\rvert}
\newcommand{\norm}[1]{\lvert\lvert#1\rvert\rvert}
\newcommand{\tand}{\text{ and }}
\newcommand{\tor}{\text{ or }}



\begin{document}
    \begin{pb}
        To show that \(F\) is increasing, first note that \(\mu\) is nonnegative, so that for any \(x>0, \; y<0\) we have
        \begin{align*}
            F(y) = -\mu((y,0]) \leq F(0) = 0 \leq \mu((0,x]) = F(x)
        \end{align*}
        Then it suffices to show \(F\) is increasing on \([-\infty,0) \tand (0,\infty]\), both follow directly from monotonicity of \(\mu\),
        \begin{align*}
            &0 < x \leq y \implies (0,x] \subset (0,y] \implies \mu((0,x]) \leq \mu((0,y]) \implies F(x) < F(y) \\
            &y \leq x < 0 \implies (x,0] \subset (y,0] \implies -\mu((0,y]) \leq -\mu((0,x]) \implies F(y) < F(x)
        \end{align*}
        Let \(x_0 \geq 0\), then \(\lim_{x\searrow x_0}F(x) = \lim_{x\searrow x_0}\mu(0,x]\). It suffices to show for an arbitrary decreasing sequence \(\set{x_i}_1^\infty\), we have
        \(\lim_{i \to \infty}\mu(0,x_i] = \mu(0,x_0] = F(x_0)\) which follows from continuity from above, so the right sided limit exists and is equal to \(F(x_0)\).

        Now let \(x_0 < 0\), then \(\lim_{x\searrow x_0}F(x) = -\lim_{x\searrow x_0}\mu(x,0]\). It suffices to show for an arbitrary decreasing sequence \(\set{x_i}_1^\infty\), we have
        \(\lim_{i \to \infty}-\mu(x_i,0) = -\mu(x_0,0] = F(x_0)\), which follows from continuity from below, so the right hand sided limit exists and is equal to \(F(x_0)\).
    \end{pb}
    \begin{pb}
        \textbf{(a)} \(F\) bounded. Proof being, assume \(\abs{F} < M\), define sets \(\set{E_i}_1^\infty\), where \(E_i = (-i,-i+1] \cup (i-1,i]\)
        then we have
        \begin{align*}
           m^F(\mathbb{R}) = \sum_1^\infty m^F(E_i) &= \lim_{n\to\infty}\sum_1^n m^F(E_i) \\
            &= \lim_{n\to\infty}\sum_1^n F(i) - F(i-1) + F(-i + 1) - F(-i) \\
            &= \lim_{n\to\infty}F(n) - F(-n) \leq \lim_{n\to\infty}\abs{F(n)} + \abs{F(-n)} \leq 2M
        \end{align*}
        Conversely, assume that \(m^F(\mathbb{R}) = M < \infty\) (note that \(M > 0\)), then for any \(x \geq 0\), we have \(F(x) - F(0) \leq m^F(\mathbb{R}) = M\) by monotonicity, and \(F(0) < F(x)\).
        Similarly, if \(x < 0\), we have \(F(x) \leq F(0)\), and by monotonicity \(F(0) - F(x) \leq m^F(\mathbb{R}) = M\). Taken together for any \(x\) we have
        \begin{align*}
            F(0) - M \leq F(x) \leq M + F(0)
        \end{align*}
        implying \(F\) is bounded. 

        \textbf{(b)} \(F\) continuous at \(x_0\). Proof being, assume \(F\) is continuous at \(x_0\), then for some sequence \(\set{\delta_n}_1^\infty\), we have 
        \(\abs{x - x_0} \leq \delta_n \implies \abs{F(x) - F(x_0)} < \frac{1}{n}\), then continuity from above implies 
        \[0 \leq m^F(\set{x_0}) = m^F\left(\bigcap_1^\infty(\delta_n,x_0]\right) = \lim_{n\to\infty}m^F((\delta_n,x_0]) \leq \lim_{n\to\infty}\frac{1}{n} = 0\]
        Conversely we need only show right continuity. Suppose that \(m^F(\set{x_0}) = 0\), and let \(\epsilon > 0\), then continuity from above implies that
        \[\lim_{n\to\infty}m^F(x_0 - \frac{1}{n},x_0] = m^F\left(\bigcap_1^\infty (x_0 - \frac{1}{n},x_0]\right) = m^F(\set{x_0}) = 0\]
        so in particular, there exists \(N\) sufficiently large that \(m^F(x_0 - \frac{1}{N},x_0] = \abs{F(x_0) - F(x_0 - \frac{1}{N})} < \epsilon\), 
        and since \(F\) is increasing and right continuous this proves left continuity and hence continuity.

        \textbf{(c)} \(m^{F,*}\) is the point mass at \(0\). Proof being, let \(0 \in E \subset \mathbb{R}\), then for any collection of half open intervals \(\set{I_i}_1^\infty\), we have 
        \(0 \in I_n\) for some \(n\). Then we can write \(I_n = (a,b]\), for \(a < 0, \; b \geq 0\), then \(m^F_0(I_n) = 1\), so that \(1 \leq \sum_1^\infty m^{F}_0(I_i)\), and since this holds for all such
        covers of \(E\), we have \(1 \leq m^{F,*}(E)\), and for the reverse inequality note that \(E \subset \mathbb{R} \subset \bigcup_1^\infty (-i, -i+1] \cup (i-1, i+1]\), which is a countable union of
        half open intervals, all but \((-1,0]\) having \(m^F_0(I) = 0\), so \(1 \leq m^{F,*}(E) \leq m^{F,*}(\mathbb{R}) \leq 1\).

        Now suppose that \(0 \not \in E\), then \(E \subset (- \infty, 0) \cup (0,\infty)\), but then \((- \infty, 0) \cup (0,\infty) = \bigcup_1^\infty (-n, 1/n) \cup (0,n)\)
        implies that
        \begin{align*}
            0 \leq m^{F,*}(E) \leq m^{F,*}((- \infty, 0) \cup (0,\infty)) 
            \leq \sum_1^\infty m^F_0(-n, 1/n) + m^F_0(0,n) = 0
        \end{align*}
        Finally note that \(m^{F,*}(\emptyset) = 0\) by definition.

        I claim that \(M_F = \mathcal{P}(\mathbb{R})\), let \(A \subset \mathbb{R}\), and \(E \subset \mathbb{R}\). First assume that \(0 \not \in E\), then
        \[m^{F,*}(E) = 0 = m^{F,*}(E \cap A) + m^{F,*}(E \cap A^c)\]
        since neither of the sets measured on the right hand side of the equation contain 0. Now assume that \(0 \in E\), then \(0 \in E \cap A \tor E \cap A^c\) but not both.
        This suffices to show that
        \[m^{F,*}(E) = 1 = m^{F,*}(E \cap A) + m^{F,*}(E \cap A^c)\]
        So each \(A \in M_F\) is measurable.

        \textbf{(d)} \(m^F\) counts the number of integers in a set \(E\). Proof being, denote the floor function as \(F\).
        First apply theorem 1.16 of Folland (since \(F\) is increasing and right continuous), then \(m^F(a,b] = (F(b) - F(a))\) is a measure.
        If \(z\) is an integer, then we can apply continuity from above:
        \[m^F(\set{z}) = m^F \left(\bigcap_1^\infty (z-\frac1n,z]\right) = \lim_{n\to\infty}m^F(z-\frac1n,z] = \lim_{n\to\infty}1 = 1\]
        So each integer singleton is a measurable set of measure \(1\). Now let \(E \subset \mathbb{R}\setminus\mathbb{Z}\), then \newline
        \(E \subset \bigcup_{n \in \mathbb{Z}}\bigcup_{k=1}^\infty(n-1,n-\frac{1}{k}]\) is a countable union of sets with measure \(0\), hence
        \[0 \leq m^F(E) \leq m^F (\bigcup_{n \in \mathbb{Z}}\bigcup_{k=1}^\infty(n-1,n-\frac{1}{k}]) \leq \sum_{n\in \mathbb{Z}}\sum_{k=1}^\infty m^F(n-1,n-\frac{1}{k}] = 0\]
        Note that singletons are borel sets, then if \(E \subset \mathbb{R}\), we can write \(E \cap \mathbb{Z} = \set{z_i}_i\), then if there are infinitely many \(z_i\):
        \begin{align*}
            \infty = m^F\left(\bigcup_i\set{z_i}\right) \leq m^F(E)
        \end{align*}
        and if there are finitely many \(z_i\):
        \begin{align*}
            m^F(E) = m^F\left(E \cap \mathbb{Z}\right) + m^F\left(E \cap \mathbb{Z}^c\right) = m^F\left(\bigcup_{i=1}^n \set{z_i}\right) + 0 = \sum_1^n m^F(\set{z_i}) = n
        \end{align*}
    \end{pb}
    \begin{pb}
        
    \end{pb}
\end{document}