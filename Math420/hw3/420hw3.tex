\documentclass[10.5pt]{article}
\usepackage{amsmath, amsfonts, amssymb,amsthm}
\usepackage[includeheadfoot]{geometry} % For page dimensions
\usepackage{fancyhdr}
\usepackage{enumerate} % For custom lists

\fancyhf{}
\lhead{Math 420hw2}
\pagestyle{fancy}

% Page dimensions
\geometry{a4paper, margin=1in}

\theoremstyle{definition}
\newtheorem{pb}{}

% Commands:

\newcommand{\set}[1]{\{#1\}}
\newcommand{\abs}[1]{\lvert#1\rvert}
\newcommand{\norm}[1]{\lvert\lvert#1\rvert\rvert}
\newcommand{\tand}{\text{ and }}
\newcommand{\tor}{\text{ or }}

\begin{document}
\begin{pb}
    \textbf{(a)} Since \(g \geq 0\), \(0\) is a simple function less than \(g\), hence for any set \(E \in \mathcal{M}\),
    \[\mu_g(E) = \int_E g d\mu \geq int_E 0 d\mu = 0\]
    Proving nonnegativity. Furthermore, we have \[\mu_g(\emptyset) = \int \chi_\emptyset g d\mu = \int 0 d\mu = 0\]
    So it will suffice to sheck for countable additivity. Consider the disjoint collection \(\set{E_i}_1^\infty\), and denote \(E := \bigcup_1^\infty E_i\).
    Then \(\chi_E = \sum_1^\infty \chi_{E_i}\), so that
    \begin{align*}
        \mu_g(E) &= \int_E gd\mu = \int g\chi_E d\mu = \int \sum_1^\infty g \chi_{E_i} d\mu = \int \lim_{n\to\infty} \sum_1^n g \chi_{E_i} d\mu \\
    &\overset{\text{MCT}}{=} \lim_{n\to\infty}\int \sum_1^n g\chi_{E_i}d\mu = \lim_{n\to\infty} \sum_1^n \int g\chi_{E_i}d\mu = \sum_1^\infty \int g\chi_{E_i}d\mu
\end{align*}
    The second last equality follows from linearity.

    \textbf{(b)}
\end{pb}
\end{document}