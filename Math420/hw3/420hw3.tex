\documentclass[10.5pt]{article}
\usepackage{amsmath, amsfonts, amssymb,amsthm}
\usepackage[includeheadfoot]{geometry} % For page dimensions
\usepackage{fancyhdr}
\usepackage{enumerate} % For custom lists

\fancyhf{}
\lhead{Math 420hw3}
\pagestyle{fancy}

% Page dimensions
\geometry{a4paper, margin=1in}

\theoremstyle{definition}
\newtheorem{pb}{}

% Commands:

\newcommand{\set}[1]{\{#1\}}
\newcommand{\abs}[1]{\lvert#1\rvert}
\newcommand{\norm}[1]{\lvert\lvert#1\rvert\rvert}
\newcommand{\tand}{\text{ and }}
\newcommand{\tor}{\text{ or }}

\begin{document}
\begin{pb}
    \textbf{(a)} Since \(g \geq 0\), \(0\) is a simple function less than \(g\), hence for any set \(E \in \mathcal{M}\),
    \[\mu_g(E) = \int_E g d\mu \geq int_E 0 d\mu = 0\]
    Proving nonnegativity. Furthermore, we have \[\mu_g(\emptyset) = \int \chi_\emptyset g d\mu = \int 0 d\mu = 0\]
    So it will suffice to sheck for countable additivity. Consider the disjoint collection \(\set{E_i}_1^\infty\), and denote \(E := \bigcup_1^\infty E_i\).
    Then \(\chi_E = \sum_1^\infty \chi_{E_i}\), so that
    \begin{align*}
        \mu_g(E) &= \int_E gd\mu = \int g\chi_E d\mu = \int \sum_1^\infty g \chi_{E_i} d\mu = \int \lim_{n\to\infty} \sum_1^n g \chi_{E_i} d\mu \\
    &\overset{\text{MCT}}{=} \lim_{n\to\infty}\int \sum_1^n g\chi_{E_i}d\mu = \lim_{n\to\infty} \sum_1^n \int g\chi_{E_i}d\mu = \sum_1^\infty \int g\chi_{E_i}d\mu
\end{align*}
    The second last equality follows from linearity.

    \textbf{(b)}
    \textbf{Lemma.} For any interval \(I = (a,b)\) \(m_{F'}(I) = m^F(I)\). Proof being, we have \(F'\) continuous on \(\overline{I}\), hence we can apply theorem 2.28 of Folland (equivalence to Riemann integral)
    \[m_{F'}(I) = m_{F'}(\overline{I}) = \int_{\overline{I}} F' dm = \int_a^b F' dx \overset{\text{FTC}}{=} F(b) - F(a) = m^F(I)\]
    Now let \(E \in \mathcal{B}_\mathbb{R}\). We can define sets \(E_N := E \cap \set{x \vert N - 1 \leq \abs{F'(x)} < N}\), so that \(E\) is the disjoint union \(\bigcup_1^\infty E_N\). For each \(N\),
    \(\set{x \vert N - 1 \leq \abs{F'(x)} < N}\) is borel, since by the continuity of \(\abs{F'}\) this is the union of the closed level set \(\set{x \vert \abs{F'(x)} = N-1}\) and the open set 
    \(\set{x \vert N-1 < \abs{F'(x)} < N}\).
    
    By monotonicity it is apparent that \(m_{F'}(E) \leq \inf \set{\sum_1^\infty m_{F'}(I_i) \vert E \subset \bigcup_1^\infty I_i, \; I_i \text{ open Intervals}}\), note the reverse inequality is trivial when \(m_{F'}(E) = \infty\),
    so it will suffice to show the finite case.
    Let \(\epsilon > 0\), then since \(m^* = m\), on the borel sets. Then we can choose for each \(E_N\), some collection of open intervals \(\set{I^N_j}_{j=1}^\infty\) covering \(E_N\), such that
    \(\sum_{j=1}^\infty m(I^N_j) \leq m(E_N) + \epsilon\frac{1}{N2^N}\) and define \(J := \bigcup_{i,j=1}^\infty I_j^i\). It follows that
    \begin{align*}
        m_{F'}(J) - m_{F'}(E) &= \sum_{N=1}^\infty \left(\sum_{j=1}^\infty m_{F'}(I_j^N) - m_{F'}(E_N) \right) = 
        \sum_{N=1}^\infty\left(\sum_{j=1}^\infty \int F' \chi_{I_j^N} dm - \int F' \chi_{E_N} dm \right) \\
        &\overset{\text{MCT}}{=} \sum_{N=1}^\infty\left(\int F' \sum_{j=1}^\infty \chi_{I_j^N} dm - \int F' \chi_{E_N} dm \right)
        = \sum_{N=1}^\infty \int F' \left( \sum_{j=1}^\infty \chi_{I_N^j} - \chi_{E_N}\right)dm \\
        &\leq \sum_{N=1}^\infty N \left(\int \sum_{j=1}^\infty \chi_{I_N^j}dm - \int \chi_{E_N}dm\right) 
        \overset{\text{MCT}}{=} \sum_{N=1}^\infty N \left(\sum_{j=1}^\infty \int \chi_{I_N^j}dm - \int \chi_{E_N}dm\right) \\
        &= \sum_{N=1}^\infty N \left(\sum_{j=1}^\infty m(I_N^j) - m(E_N)\right) \leq \sum_{N=1}^\infty N(\frac{\epsilon}{N2^{-N}}) = \epsilon \\
        \implies m_{F'}(E) \geq m_{F'}(J) - &\epsilon
    \end{align*}

    Now finally, let \(E \in \mathcal{B}_\mathbb{R}\), then for any collection of intervals, \(\set{I_i}_1^\infty\), such that \(E \subset \bigcup_1^\infty I_i\) we have
    \(\sum_1^\infty m_{F'}(I_i) = \sum_1^\infty m^F(I_i)\)
    by the lemma, hence
    \begin{align*}
        m_{F'}(E) &= \inf \set{\sum_1^\infty m_{F'}(I_i) \vert E \subset \bigcup_1^\infty I_i, \; I_i \text{ open Intervals}} \\
        &= \inf \set{\sum_1^\infty m^F(I_i) \vert E \subset \bigcup_1^\infty I_i, \; I_i \text{ open Intervals}} = m^{F,*}(E) = m^F(E) 
    \end{align*}
\end{pb}
\begin{pb}
    \textbf{(a)} We first show that \(\set{x \in \mathbb{R} \vert \frac{x}{2\pi} \in \mathbb{Q}}\) has measure zero. This is straghtforward since 
    \[\frac{x}{2\pi} \in \mathbb{Q} \iff \exists q \in \mathbb{Q}, \text{ such that } x = 2q\pi \iff x \in \set{\pi q \vert q \in \mathbb{Q}}\]
    where \(\pi \mathbb{Q}\) has an obvious bijective correspondance with \(q\), namely multiplication by \(\pi\). Hence \(\pi \mathbb{Q}\) is countable
    and therefore has measure lebesgue \(0\).

    \(f \equiv 0\) on \(\mathbb{R}\setminus[0,\pi]\), so that \(\overline{f} = \underline{f} = 0\) on \(\mathbb{R}\setminus[0,\pi]\).

    Let \(q \in (\pi \mathbb{Q})^c \cap[0,\pi]\), then from the hint we have \(\set{k q\frac{1}{2\pi} (\text{mod}1)\vert k \in \mathbb{N}}\) is dense in \([0,1]\), but this is equivalent to
    \(\set{kq (\text{mod}2\pi)\vert k \in \mathbb{N}}\) being dense in \([0,2\pi]\). Hence \(\limsup \sin^2 kq = 1 \tand \liminf \sin^2 kq = 0\). This proves that 
    \(\overline{f} = \chi_{[0,\pi]} \tand \underline{f} = 0\) almost everywhere, since \(\pi \mathbb{Q}\) has lebesgue measure \(0\).

    \textbf{(b)} 
    Since \(sin^2(kx)\) is continuous on \([0,\pi]\), we can apply Folland Theorem 2.28, to get the following
    \begin{align*}
        \int f_k dm = \int sin^2(kx)\chi_[0,\pi]dm = \int_0^\pi \sin^2(kx)dx \overset{\text{FTC}}{=} \left.\frac{x}{2} - \frac{\sin2kx}{4k}\right\vert_0^\pi = \frac{\pi}{2}
    \end{align*}
    Where this is a valid application of FTC, since
    \begin{align*}
        \frac{d}{dx} \frac{x}{2} - \frac{\sin2kx}{4k} = \frac{1 - cos2kx}{2} = \frac{1 - cos^2(kx) + sin^2(kx)}{2} = \sin^2(kx)
    \end{align*}
    And the integral of \(\overline{f}, \underline{f}\) is as follows, since we proved in part (a)  they have equality to \(\chi_[0,\pi] \tand 0\) almost everywhere:
    \begin{align*}
        \int \overline{f} dm &= \int \chi_{[0,\pi]} dm = m([0,\pi]) = \pi \\
        \int \underline{f} dm &= \int 0 dm = m(\emptyset) = 0
    \end{align*}
    This is indeed consistent with Fatous lemma, since
    \begin{align*}
        \int \liminf f_k dm = 0 \leq \liminf \int f_k dm = \liminf \frac{\pi}{2} = \frac{\pi}{2}
    \end{align*}
\end{pb}
\begin{pb}
    \textbf{(a)} Sequences which are absolutely convergent as series. As proof, any function \(f: \mathbb{N} \to \mathbb{R}\) defines a real valued sequence \(\set{f(n)}_{n \in \mathbb{N}}\).
    So any function \(f\) on \(\mathbb{N}\) is a simple function of the form \(\sum_1^\infty f(n)\chi_{\set{n}}\)
    \begin{align*}
        \int \abs{f} d\mu &= \int \sum_1^\infty \abs{f(n)}\chi_{\set{n}} d\mu 
        = \int \lim_{N\to\infty}\sum_1^N \abs{f(n)}\chi_{\set{n}} d\mu \overset{\text{MCT}}{=} \lim_{N\to\infty}\int \sum_1^N \abs{f(n)}\chi_{\set{n}} d\mu \\ 
        &= \lim_{N\to\infty} \sum_1^N \int \abs{f(n)}\chi_{\set{n}} d\mu = \sum_1^\infty \abs{f(n)}
    \end{align*}
    so that \(\int \abs{f} d\mu < \infty\) exactly when \(\sum_1^\infty \abs{f(n)} < \infty\).

    \textbf{(b)} Functions with countable support, which are absolutely convergent as series on their support. As proof, we need only prove the countable support portion of the claim, then 
    defer to the proof of part (a) for the rest. Suppose for contraposition that \(\set{x \vert f(x) \neq 0}\) is uncountable. Then
    \begin{align*}
        \set{x \vert f(x) \neq 0} = \set{x \vert \abs{f(x)} > 0} = \bigcup_{n=1}^\infty \set{x \vert f(x) \geq \frac{1}{n}}
    \end{align*}
    So that for some \(n \in \mathbb{N}\), we have \(\set{x \vert f(x) \geq \frac{1}{n}}\) is uncountable, otherwise \(\set{x \vert f(x) \neq 0}\) would be a countable union of countable sets hence countable.
    This gives the claimed result, namely if \(E \subset \set{x \vert f(x) \geq \frac{1}{n}}\) is countable, then
    \begin{align*}
        \int f d\mu \geq \int_{\set{x \vert f(x) \geq \frac{1}{n}}} f d\mu \geq \int_E \frac{1}{n} d\mu = \sum_1^\infty \frac{1}{n} = \infty
    \end{align*}

    \textbf{(c)} We proved last homework, (2(d)) that \(m^F(E) = \# E \cap \mathbb{Z}\). Given that, we may enumerate the integers as \(\set{z_i}_1^\infty\), then
    the \(L^1\) functions on \(m^F\) are exactly those whose values as a sequence \(\set{f(z_i)}_1^\infty\) are an absolutely convergent series. Proof being that
    \(f = f\chi_\mathbb{Z}\) almost everywhere with respect to \(m^F\), then the proof of part (a).
\end{pb}
\begin{pb}
    \textbf{a} If \(p = 0\), then the pointwise limit \(f^{(0)} = \chi_[0,\infty]\), if \(p > 0\), then the pointwise limit \(f^{(p)} = 0\). Proof being, first
    let \(p = 0\), then for any \(x < 0\), \(f^{(0)}_n(x)\) = 0 for all \(x < 0\). For any \(x > 0\), \(x < N\) for some natural number \(N\), hence
    \(f^{(0)}_n(x) = 1\) for all \(n \geq N\). If \(p > 0\), then \(\lim_{n\to\infty}n^{-p} = 0\), 
    hence \[\lim_{n\to\infty}f^{(p)}_n = \lim_{n\to\infty}n^{-p}\chi_{[0,n]} = \lim_{n\to\infty}n^{-p}\lim_{n\to\infty}\chi_{[0,n]} = 0\]

    \textbf{(b)} For \(p = 0\), they agree, proof being:
    \begin{align*}
        &\int f^{(0)}dm = \int \chi_{[0,\infty]}dm = m([0,\infty]) = \infty \\
        &\lim_{n\to\infty}\int f^{(0)}_n dm = \lim_{n\to\infty}\int \chi_[0,n]dm = \lim_{n\to\infty} m([0,n]) = \lim_{n\to\infty} n = \infty
    \end{align*}

    For \(p > 0\),
    \begin{align*}
        &\int f^{(p)}dm = 0 \\
        &\lim_{n\to\infty}\int f^{(p)}_n dm = \lim_{n\to\infty}\int n^{-p}\chi_[0,n]dm = \lim_{n\to\infty} n^{-p}m([0,n]) = \lim_{n\to\infty} n^{1-p}
    \end{align*}
    So that \(\lim_{n\to\infty}\int f^{(p)}_n dm = 0 = \int f^{(p)}dm\) iff \(p > 1\).

    \textbf{(c)} \(g^{(p)} := \sum_{n=1}^\infty n^{-p}\chi_{[n-1,n]}\), this is the smallest function dominating \(f^{(p)}_n\) for all \(n\). 
    We have convergence in part (b), exactly when both limits are infinite, or the
    conditions for the dominated convergence theorem are met by \(g\) (equivalently when its possible they be met), since 
    \[p > 1 \iff \sum_{n=1}^\infty n^{-p} < \infty \iff g^{(p)} \in L^1\]
    The last if and only if statement can be seen by
    \begin{align*}
        \int \abs{g^{(p)}} dm &= \int g^{(p)}dm = \int \sum_{n=1}^\infty n^{-p}\chi_{[n-1,n]} dm = \int \lim_{N\to\infty}\sum_1^N n^{-p}\chi_{[n-1,n]} dm \\
        &\overset{\text{MCT}}{=} \lim_{N\to\infty} \int \sum_1^N n^{-p}\chi_{[n-1,n]} dm = \lim_{N\to\infty} \sum_1^N \int n^{-p}\chi_{[n-1,n]} dm \\
        &= \sum_1^\infty n^{-p} m([n-1,n]) = \sum_1^\infty n^{-p}
    \end{align*}
\end{pb}
\begin{pb}
    \textbf{(a)} Apply DCT with \(g := \abs{\psi}\sup_\mathbb{R}\abs{\phi}\), so that \(\int g dm = \sup_\mathbb{R}\abs{\phi} \int \abs{\psi} dm\) is \(L^1\).
    This furnishes
    \begin{align*}
        \lim_{n\to\infty}\int\phi(x/n)\psi(x)dm = \int \lim_{n\to\infty}\phi(x/n)\psi(x)dm = \lim_{x\to 0}\phi(x) \int\psi(x)dm
    \end{align*}
    The last equality as a consequence of continuity of \(\phi\).

    \textbf{(b)} \(\abs{\phi} \vert_{\text{supp}\phi}\) is a continuous function on a compact set, hence bounded. Let \(M\) be an upper bound for
    \(\abs{\phi}\) on it's support and hence all of \(\mathbb{R}\). So we may apply \(DCT\) with \(g := M\abs{\psi}\), which is a constant multiple of an \(L^1\)
    function hence \(L^1\). This lets us evaluate
    \begin{align*}
        \lim_{n\to\infty} \int \phi(nx)\psi(x)dm = \int \lim_{n\to\infty} \phi(nx)\psi(x)dm = \int 0 dm = 0
    \end{align*}
    The second to last equality following since \(\phi\) has compact support, so for every non-zero value \(\phi(nx)\) is eventually outside the support of \(\phi\) for
    large enough \(n\), giving us \(\phi(nx)\psi(x) = 0\) almost everywhere.

    \textbf{(c)}
    \begin{align*}
        \lim_{n\to\infty} n^2 \int \frac{\sin(x^2/n^2)}{x^2(1+x^2)}dm = \lim_{n\to\infty}\int\frac{n^2\sin(x^2/n^2)}{x^2(1+x^2)}dm
    \end{align*}
    So define \(\phi(u) = \frac{sin(u^2)}{u^2}\), which has absolute values less than or equal to 1 and continuous, and \(\psi(x) = \frac{1}{1 + x^2} \in L^1\).
    Then we apply (a), so which gives us
    \begin{align*}
        \lim_{n\to\infty} n^2 \int \frac{\sin(x^2/n^2)}{x^2(1+x^2)}dm &= \lim_{x\to 0}\frac{\sin(x^2)}{x^2} \int \frac{1}{1+x^2} dm 
        = \int \lim_{n \to \infty} \chi([-n,n])\frac{1}{1+x^2}dm \\
        &\overset{\text{MCT}}{=} \lim_{n \to \infty}\int \chi([-n,n])\frac{1}{1+x^2}dm = \lim_{n\to\infty}\int_{-n}^n \frac{1}{1+x^2} dx \\
        &= \lim_{n\to \infty} \arctan(n) - \arctan(-n) = \pi
    \end{align*}
    Where we use Folland 2.28 to evaluate the integral as a Riemann integral.

    \textbf{(d)} We have \(0 \leq \abs{\cos(x^2)} \leq 1\), with equality to \(1\) on \([0,\sqrt{\pi}]\) if and only if \(x = 0 \tor \sqrt(\pi)\).
    Hence at all other points the pointwise limit of 
    \(\lim_{n\to\infty}\abs{\cos^n(x^2)}\chi_{[0,\sqrt{\pi}]} = \lim_{n\to\infty}\alpha^n\) for \(0 \leq \alpha < 1\),
    this implies that \(\lim_{n\to\infty}cos^n(x^2)\chi_{[0,\sqrt{\pi}]} = 0\) almost everywhere.

    Additionally, \(cos^{n}(x^2)\chi_{[0,\sqrt{\pi}]}\) is dominated by \(\chi_{[0,\sqrt{\pi}]} \in L^1\). Hence we can apply DCT, so that
    \begin{align*}
        \lim_{n\to\infty} \int cos^n(x^2)\chi_{[0,\sqrt{\pi}]} dm = \int \lim_{n\to\infty} cos^n(x^2)\chi_{[0,\sqrt{\pi}]} dm
        = \int 0 dm = 0
    \end{align*}
\end{pb}
\begin{pb}
    Let \(\epsilon > 0\), then by the definition of \(\int \abs{f}d\mu\), we can choose some simple function \(\varphi \leq \abs{f}\), such that \(\int \abs{f}d\mu - \int \varphi d\mu < \epsilon/2\).
    Then \(\varphi = \sum_1^N c_i \chi_{E_i}\), so in particular \(\sup(\varphi) = \max_ic_i < \infty\). Now choose \(\delta = \frac{\epsilon}{2\sup(\varphi)}\), so that for any \(E\) with
    \(\mu(E) < \delta\), we have
    \begin{align*}
        &\int_E \abs{f}d\mu - \int_E \varphi d\mu \leq \int \abs{f}d\mu - \int \varphi d\mu < \epsilon/2 \\
        \implies &\int_E \abs{f}d\mu < \epsilon/2 + \int_E \varphi d\mu \leq \epsilon/2 + \int_E \sup \varphi d\mu 
        = \epsilon/2 + \mu(E) \sup \varphi < \epsilon
    \end{align*}
\end{pb}
\end{document}