\documentclass[10.5pt]{article}
\usepackage{amsmath, amsfonts, amssymb,amsthm}
\usepackage{graphicx}
\usepackage[includeheadfoot]{geometry} % For page dimensions
\usepackage{fancyhdr}
\usepackage{enumerate} % For custom lists
\usepackage{xcolor}

\fancyhf{}
\lhead{Math 425hw8}
\rhead{Tighe McAsey - 37499480}
\pagestyle{fancy}

% Page dimensions
\geometry{a4paper, margin=1in}

\theoremstyle{definition}
\newtheorem{pb}{}

% Commands:

\newcommand{\set}[1]{\{#1\}}
\newcommand{\abs}[1]{\lvert#1\rvert}
\newcommand{\norm}[1]{\lvert\lvert#1\rvert\rvert}
\newcommand{\gen}[1]{\langle #1 \rangle}
\newcommand{\tand}{\text{ and }}
\newcommand{\tor}{\text{ or }}
\newcommand{\vp}{\varphi}
\newcommand{\R}{\text{Re}}
\newcommand{\I}{\text{Im}}
\newcommand{\parx}{\frac{\partial}{\partial x}}
\newcommand{\pary}{\frac{\partial}{\partial y}}
\newcommand{\parz}{\frac{\partial}{\partial z}}
\newcommand{\paru}{\frac{\partial}{\partial u}}
\newcommand{\parv}{\frac{\partial}{\partial v}}
\newcommand{\tth}{\tilde{\theta}}
\newcommand{\z}{\imath}
\newcommand{\hook}{\lrcorner\;}

\begin{document}
    I worked with Aaron on problem 1.
    \begin{pb}
        \textbf{(a)}
        \begin{align*}
            g = \gen{\frac{\partial \varphi^{-1}}{\partial x},\frac{\partial \varphi^{-1}}{\partial x}} = \gen{(1,f'(x)),(1,f'(x))} = 1 + (f'(x))^2
        \end{align*}
        Thus the formula for arc length in this context is
        \begin{align*}
            \int_a^b \sqrt{1 + (f'(x))^2}dx
        \end{align*}

        \textbf{(b)}
        \begin{align*}
            g = (\varphi^{-1})^*g_{\text{EUC}} &= \begin{bmatrix}
                \frac{\partial (\varphi^{-1})^x}{\partial x} & \frac{\partial (\varphi^{-1})^y}{\partial x} & \frac{\partial (\varphi^{-1})^z}{\partial x} \\
                \frac{\partial (\varphi^{-1})^x}{\partial y} & \frac{\partial (\varphi^{-1})^y}{\partial y} & \frac{\partial (\varphi^{-1})^z}{\partial y}
            \end{bmatrix}
            \begin{bmatrix}
                \frac{\partial (\varphi^{-1})^x}{\partial x} & \frac{\partial (\varphi^{-1})^x}{\partial y} \\
                \frac{\partial (\varphi^{-1})^y}{\partial x} & \frac{\partial (\varphi^{-1})^y}{\partial y} \\
                \frac{\partial (\varphi^{-1})^z}{\partial x} & \frac{\partial (\varphi^{-1})^z}{\partial y} \\
            \end{bmatrix} \\
            &= \begin{bmatrix} 1 & 0 & \frac{\partial f}{\partial x} \\
                0 & 1 & \frac{\partial f}{\partial y} \end{bmatrix} 
                \begin{bmatrix} 1 & 0 \\ 0 & 1 \\ \frac{\partial f}{\partial x} & \frac{\partial f}{\partial y} \end{bmatrix} \\
                &= \begin{bmatrix} \left(\frac{\partial f}{\partial x}\right)^2 + 1 & \frac{\partial f}{\partial x}\frac{\partial f}{\partial y} \\
                \frac{\partial f}{\partial x} \frac{\partial f}{\partial y} &
                \left(\frac{\partial f}{\partial y}\right)^2 + 1
             \end{bmatrix}
        \end{align*}
        So that \(\det(g_{ij}) = 1 + \left(\frac{\partial f}{\partial x}\right)^2 + \left( \frac{\partial f}{\partial y}\right)^2\). It follows that the surface area formula in this context simplifies to
        \begin{align*}
            \int_U \sqrt{1 + \left(\frac{\partial f}{\partial x}\right)^2 + \left( \frac{\partial f}{\partial y}\right)^2} dxdy
        \end{align*}

        \textbf{(c)} 
        Define the map \(g: U \to S \cap V\) by \(g(x,y) = (x,y,f(x,y))\) and define \(G: U \to \mathbb{R}, \; G(x,y) = (F \circ g)(x,y)\). Since \(g\) maps onto a level set of \(F\), we have
        \begin{align*}
            dG = dF \circ d g &= 0 \\
            0 = dG = dF \circ d g &= \begin{bmatrix} \parx F & \pary F & \parz F \end{bmatrix}\begin{bmatrix} 1 & 0 \\ 0 & 1 \\ \parx f & \pary f \end{bmatrix} \\
            &= \left( \parx F + \parz F \parx f \right)\parx + \left( \pary F + \parz F \pary f \right)\pary
        \end{align*}
        Which gives allows us to solve for the \(f\) partials in terms of the \(F\) partials
        \begin{align*}
            &\parx f = - \left(\parx F\right)\left(\parz F\right)^{-1} &\pary f = - \left(\pary F\right)\left(\parz F\right)^{-1}
        \end{align*}
        Now we can derive this formula from 
        \begin{align*}
            \text{Area}(S\cap V) &= \int_U \sqrt{1 + \left(\frac{\partial f}{\partial x}\right)^2 + \left( \frac{\partial f}{\partial y}\right)^2} dxdy \\
            &= \int_U \sqrt{\left(\frac{\parz F}{\parz F}\right)^2 + \left(\frac{\parx F}{\parz F}\right)^2 + \left(\frac{\pary F}{\parz F}\right)^2} dxdy \\
            &= \int_U \frac{\sqrt{\left(\parx F\right)^2 + \left(\pary F\right)^2 + \left(\parz F\right)^2}}{\abs{\parz F}} \\
            &= \int_U \frac{\abs{\nabla F}}{\abs{\parz F}}dxdy
        \end{align*}
        %The solution to this exercise relies on the crucial observation, that on \(S \cap V\), \[F(x,y,z) = 0 = z - f(x,y)\]
        % With this observation, the exercise is routine, first note that
        % \begin{align*}
        %     &\frac{\partial F}{\partial x} = \parx (z - f(x,y)) = -\frac{\partial f}{\partial x} \\
        %     &\frac{\partial F}{\partial y} = \pary (z - f(x,y)) = -\frac{\partial f}{\partial y} \\
        %     &\frac{\partial F}{\partial z} = \parz (z - f(x,y)) = 1
        % \end{align*}
        % Now to verify the formula:
        % \begin{align*}
        %     \int_U \frac{\abs{\nabla F}}{\abs{\parz F}}dxdy &= \int_U \frac{\sqrt{\left(\frac{\partial F}{\partial x}\right)^2 + \left(\frac{\partial F}{\partial y}\right)^2 + \left(\frac{\partial F}{\partial z}\right)^2 }}{\abs{\parz F}}dxdy \\
        %     &= \int_U \sqrt{1 + \left(\frac{\frac{\partial F}{\partial x}}{\frac{\partial F}{\partial z}}\right)^2 + \left(\frac{\frac{\partial F}{\partial y}}{\frac{\partial F}{\partial z}}\right)^2}dxdy \\
        %     &= \int_U \sqrt{1 + \left(\frac{\partial f}{\partial x}\right)^2 + \left( \frac{\partial f}{\partial y}\right)^2}dxdy
        % \end{align*}
        As in part (b). Now checking the formula:
        \begin{align*}
            \int_U \frac{\abs{\nabla F}}{\abs{\parz F}}dxdy &= \int_U 2\frac{\sqrt{x^2 + y^2 + z^2}}{2z}dxdy = \int_U \sqrt{1 + \frac{x^2}{z^2} + \frac{y^2}{z^2}}dxdy \\
            &= \int_U \sqrt{1 + \frac{x^2}{1-x^2-y^2} + \frac{y^2}{1-x^2-y^2}}dxdy \\
            &= \int_U \sqrt{\frac{1}{1-x^2-y^2}}dxdy \\
            &= \int_{y=-1}^1 \int_{x=-\sqrt{1-y^2}}^{\sqrt{1-y^2}} \frac{1}{\sqrt{1-x^2-y^2}}dxdy \\
            &= \int_{y=-1}^1 \arctan \frac{x}{\sqrt{1-x^2-y^2}}\vert_{x \to -\sqrt{1-y^2}}^{x \to \sqrt{1-y^2}} \\
            &= \int_{y=-1}^1 \lim_{x \to \infty}\arctan(x) - \lim_{x \to -\infty}\arctan(x) \\
            &= \int_{-1}^1 \pi dy = 2\pi
        \end{align*}
        As desired.
    \end{pb}
    \begin{pb}
        \textbf{(a)} We compute
        \begin{align*}
            A = \begin{bmatrix} \paru f^x & \paru f^y & \paru f^z \\
                \parv f^x & \parv f^y & \parv f^z  \end{bmatrix}
            = \begin{bmatrix} -b\sin u\cos v & -b\sin u \sin v &b\cos u\\
            -(a+b\cos u)\sin v & (a+b\cos u)\cos v & 0 \end{bmatrix}
        \end{align*}
        Then in \((u,v)\) coordinates, we have \(g = f^* g_{\text{EUC}}\), i.e.
        \begin{align*}
            g = AA^T = \begin{bmatrix} b^2 & 0 \\ 0 & (a+b\cos u)^2 \end{bmatrix}
        \end{align*}

        \textbf{(b)} Write \(c_1 = f\circ \gamma_1, \; c_2 = f\circ \gamma_2\). Then we have
        \(\dot{\gamma}_1^1 = 0, \dot{\gamma}_1^2 = 1\), so that \(\abs{\dot{\gamma_1}}^2_g = g_{2,2} = (a+b\cos \pi)^2 = (a-b)^2\). It follows that
        \begin{align*}
            L(c_1) = \int_0^{2\pi} a-b dt = 2\pi(a-b)
        \end{align*}
        Similarly, \(\dot{\gamma}_2^1 = 1, \dot{\gamma}_2^2 = 0\), so that \(\abs{\dot{\gamma_1}}^2_g = g_{1,1} = b^2\). It follows that
        \begin{align*}
            L(c_2) = \int_0^{2\pi}b = 2b\pi
        \end{align*}

        \textbf{(c)}
        \(dA = \sqrt{\det g}\;dudv = b(a+b\cos u)dudv\), hence the surface area is given by
        \begin{align*}
            SA = \int_0^{2\pi}\int_0^{2\pi}b(a+b\cos u)dudv = \int_0^{2\pi} uab + b^2\sin u\vert_0^{2\pi}dv = \int_0^{2\pi} 2\pi abdv = (2\pi)^2ab
        \end{align*}
    \end{pb}
\end{document}