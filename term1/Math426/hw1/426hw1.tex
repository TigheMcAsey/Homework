\documentclass[10.5pt]{article}
\usepackage{amsmath, amsfonts, amssymb,amsthm}
\usepackage{centernot}
\usepackage[includeheadfoot,margin=0.5in]{geometry} % For page dimensions
\usepackage{fancyhdr}
\usepackage{enumerate} % For custom lists

\fancyhf{}
\lhead{Math 426hw1}
\rhead{Tighe McAsey - 37499480}
\pagestyle{fancy}

% Page dimensions
\geometry{a4paper}

\theoremstyle{definition}
\newtheorem{pb}{}

% Commands:

\newcommand{\set}[1]{\{#1\}}
\newcommand{\abs}[1]{\lvert#1\rvert}
\newcommand{\norm}[1]{\lvert\lvert#1\rvert\rvert}
\newcommand{\tand}[1]{\text{ and }}
\newcommand{\tor}[1]{\text{ or }}

\begin{document}
    \begin{pb}
        \textbf{(a)} Let \(x \in \set{a\in X \vert \exists U \text{ open, such that } a \in U \subset A}\), then \(U \cup A^\circ\) is an open subset of \(A\) containing \(A^\circ\),
        hence by maximality \(x \in U \cup A^\circ = A^\circ\). If \(a \in A^o\), then \(a\) is in a open subset contained in \(A\) proving the other set inclusion.

        let \(x \in \set{x \in X \vert \forall U \text{ open with } x \in U \tand UU \cap A \neq \emptyset}^c\), then there exists some open \(U \subset A^c\) containing \(x\),
        so that \(A \subset U^c\) is closed this implies \(\overline{A} \subset U^c\) and hence \(x \not \in \overline{A}\). Conversely, if \(x \in \overline{A}^c\), then
        \(\overline{A}^c\) is an open set disjoint from \(A\) containing \(x\), so that \(x \in \set{x \in X \vert \forall U \text{ open with } x \in U \tand UU \cap A \neq \emptyset}^c\).

        \textbf{(b)} \(U^\circ\) is open by definition, so \(U^\circ = U\) implies \(U\) open. If \(U\) is open, then \(U\) is an open set contained in \(U\), so that \(U \subset U^\circ\) 
        and hence \(U = U^\circ\).

        \(\overline{A}\) is closed, hence \(A = \overline{A}\) implies \(A\) is closed. Now suppose that \(A\) is closed, then \(A\) is a closed set containing \(A\), hence
        \(A \supset \overline{A}\), which implies \(A = \overline{A}\).

        \textbf{(c)} The compliment of \(A^\circ\) is closed, and \(A^\circ \subset A\) implies that \(\left(A^\circ\right)^c \supset A^c\), implying that
        \(\overline{A^c} \subset \left(A^\circ\right)^c\). Conversely, \(\overline{A^c} \supset A^c\) implies that \(\overline{A^c}^c \subset A\), but this
        is the compliment of a closed set, hence open so that \(\overline{A^c}^c \subset A^\circ\), implying that \(\overline{A}^c \supset \left(A^\circ\right)\)

        \(\overline{A}^c\) is an open set contained in \(A^c\), hence \(\overline{A}^c \subset (A^c)^\circ\). Conversely, if \(x \in \overline{A}\), then from (a), any open set containing
        \(x\) has non-trivial intersection with \(A\), hence applying part (a) again we get that \(x \not \in (A^c)^\circ\), hence \(\overline{A} \subset \left((A^c)^\circ\right)^c\),
        contraposing this gives the desired equality.
    \end{pb}
    \begin{pb}
        Consider the collection \(\mathcal{I}\) of closed sets in \(X\), which are not finite unions of closed irreducibles. Every descending chain being eventually constant is equivalent to every descending chain having a lower bound
        (i.e. If \(\cap_i F_i = F_j\), then \(F_j\) is a lower bound on the chain).
        Thus we can apply Zorn's lemma which furnishes a minimal element \(Z\) in \(\mathcal{I}\), if \(Z\) were not irreducible, then it would need to be a union of closed subsets \(Z_1\cup Z_2\),
        since \(Z\) is not a finite union of irreducibles, the same must apply to one of \(Z_1 \tor ZZ_2\), but this contradicts the minimality of \(Z \in \mathcal{I}\). It follows that
        \(\mathcal{I} = \emptyset\), so that \(X\) is a finite union of irreducible elements.

        let \(\set{Y_i}_{i = 1}^m \neq \set{Z_i}_{i = 1}^n\) be two collections
        of irreducible sets, such that no set is contained in the union of the rest of the collection, and \(\bigcup_i Y_i = X = \bigcup_i Z_i\).
        Then there must exist some \(Y_i, Z_j\), such that \(Y_i \cap Z_j \neq \emptyset \tand YY_i \neq Z_j\)
        (explicitly choose some \(Y_i \not \in \set{Z_j}_j\), but \(\emptyset \neq Y_i = Y_i \cap \cup_j Z_j = \cup_j Y_i \cap Z_j\)
         cannot all be empty). We may assume WLOG \(Y_i \not \subset Z_j\), but this contradicts the Zarisky condition, since 
        \(Y_i = (Y_i \cap Z_j) \cup (Y_i \cap \cup_{i \neq j}Z_i)\) is a union of closed proper subsets of \(Y_i\).
    \end{pb}
    \begin{pb}
        \textbf{(a)} Suppose \(X\) is not connected, then there exists some \(X \subsetneq A \neq \emptyset\) which is clopen, it follows that
        \(A^c\) is also clopen. Let \(\pi: \begin{cases} x \mapsto 1 & x \in A \\ x \mapsto 0 &x\in A^c\end{cases}\) map \(X\) to \(\set{0,1}\) with the discrete topology.
        This is clearly a map, since the preimage of every set is open. Conversely, suppose there exists a surjective map \(\pi:X \to \set{0,1}\),
        we have that \(\pi^{-1}(1), \pi^{-1}(0) = (\pi^{-1}(1))^c\) are open, disjoint and non-empty. Since compliments of open sets are open, these sets are also closed.
        Hence \(\pi^{-1}(1)\) is a clopen set not equal to \(X\) or \(\emptyset\), since it is non-empty with non-empty compliment.

        \textbf{(b)} Let \(\pi:X\to \set{0,1}\) be a map, suppose WLOG \((0,1)\overset{\pi}{\mapsto}1\), it follows that \(\pi(\set{0}\times[-1,1]) = 1\), since
        assuming not we let \(\alpha = \sup\set{y \in [0,1]\vert \pi(y) = 0}\), implying that \(\pi\) is not continuous at \(\alpha\), since any open set
        \(U\) containing \(\alpha\) must contain some point \(\alpha > \beta, (0,\beta)\overset{\pi}{\mapsto}0\), and some point \(\alpha < \gamma, (0,\gamma)\overset{\pi}{\mapsto}1\) 
        by the supremum property. Let \(\beta = \inf \set{x\vert \pi(x,\sin\frac{1}{x}) = 0}\), if \(\beta=\infty\) we are done, and if \(\beta>0\), then the argument is
        identical to the case of the line, so assume \(\beta=0\). This implies that \(\pi(x,\sin \frac{1}{x}) = 0\) for all \(x > 0\), once again by the same argument
        as for the line. Now consider any open set \(U\) containing the point \((0,0)\) and take some neighbourhood \(N_\epsilon(0,0) \subset U\), taking
        \(N\) so large that \(\frac{1}{N\pi} < \epsilon\), we can see that \((\frac{1}{N\pi},\sin N\pi) = (\frac{1}{N\pi},0) \in U\). 
        \(\pi(\frac{1}{N\pi},0) = 0\) implies no open set containing \((0,0)\) is a subset of \(\pi^{-1}(1)\), contradicting \(\pi\) being a map implying that \(\beta = \infty\).
        This argument results in \(\pi(X) = 1\), so that \(X\) is connected by (a).

        
        Assume for contradiction there exists a path \(\gamma\) between \((0,0)\) and \((\frac{1}{\pi},0)\), assume \((0,y) \in \set{0}\times[0,1]\) is in \(\gamma([0,1])\),
        and let \(a \in \gamma^{-1}(y)\). By continuity, we may pick some \(\delta>0\), such that
        \(\abs{a - x}<\delta\) implies \(d(\gamma(a), \gamma(x)) < \frac{1}{2}\). Let \(x \in N_{\delta}(a)\) with \(\gamma(x) = (x_1,x_2)\) 
        and assume for the sake of contradiction \(x_1 \neq 0\).
        The projection map \(\pi:(t,s)\mapsto t\) is continuous since \(d(\pi(x),\pi(y))\leq d(x,y)\) and it is clear that the composition of continuous functions is continuous.
        We can choose \(N\) so large that \(\frac{2}{(2N + 1)\pi} < \frac{1}{N \pi} < x_1\), IVT guaruntees existence of some \(x',x''\) in between \(a\) and \(x\), 
        such that \(\pi \gamma:x' \mapsto  \frac{1}{N\pi}, x'' \mapsto \frac{2}{(2N + 1)\pi}\).
        \begin{align*}
            \abs{a - x''} \tand '\abs{a - x'}<\delta \tand dd(\gamma(a),\gamma(x'')) + d(\gamma(a),\gamma(x')) \geq d(\gamma(x'),\gamma(x'')) = \sqrt{\left(\frac{1}{N \pi}-\frac{2}{(2N + 1)\pi}\right)^2 + 1} \geq 1
        \end{align*}
        contradicting \(d(\gamma(a),\gamma(x'')), d(\gamma(a),\gamma(x'))\) both being less than \(\frac{1}{2}\).
        Hence \(x_1 = 0\), so that \(S = \set{x \in [0,1] \vert \pi \gamma(x) = 0}\) is open. Now suppose that \(\pi\gamma(y) > 0\), then there exists some
        \(\delta > 0\), so that \(\abs{y-x} < \delta\) implies \(\abs{\pi\gamma(x) - \pi\gamma(y)} < \pi\gamma(y)\), so for any \(x \in N_{\delta}(y)\)
        \begin{align*}
            \pi \gamma(x) \geq \pi \gamma(y) - \abs{\pi\gamma(x) - \pi\gamma(y)} > \pi \gamma(y) - \pi \gamma(y) = 0
        \end{align*}
        Hence \(\pi\gamma(x) > 0\), so that \(S^c\) is open, hence \(S \subset [0,1]\) is clopen. Since \(0 \in S\), and \(1 \not \in S\), \(\emptyset \neq S \neq [0,1]\),
        but \(S\) is connected by the same argument that \(\set{0}\times[-1,1]\) is connected, so this is a contradiction and \(X\) is not path connected.
    \end{pb}
\end{document}