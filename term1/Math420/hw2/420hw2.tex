\documentclass[10.5pt]{article}
\usepackage{amsmath, amsfonts, amssymb,amsthm}
\usepackage[includeheadfoot]{geometry} % For page dimensions
\usepackage{fancyhdr}
\usepackage{enumerate} % For custom lists

\fancyhf{}
\lhead{Math 420hw2}
\pagestyle{fancy}

% Page dimensions
\geometry{a4paper, margin=1in}

\theoremstyle{definition}
\newtheorem{pb}{}

% Commands:

\newcommand{\set}[1]{\{#1\}}
\newcommand{\abs}[1]{\lvert#1\rvert}
\newcommand{\norm}[1]{\lvert\lvert#1\rvert\rvert}
\newcommand{\tand}{\text{ and }}
\newcommand{\tor}{\text{ or }}



\begin{document}
    \begin{pb}
        To show that \(F\) is increasing, first note that \(\mu\) is nonnegative, so that for any \(x>0, \; y<0\) we have
        \begin{align*}
            F(y) = -\mu((y,0]) \leq F(0) = 0 \leq \mu((0,x]) = F(x)
        \end{align*}
        Then it suffices to show \(F\) is increasing on \([-\infty,0) \tand (0,\infty]\), both follow directly from monotonicity of \(\mu\),
        \begin{align*}
            &0 < x \leq y \implies (0,x] \subset (0,y] \implies \mu((0,x]) \leq \mu((0,y]) \implies F(x) < F(y) \\
            &y \leq x < 0 \implies (x,0] \subset (y,0] \implies -\mu((0,y]) \leq -\mu((0,x]) \implies F(y) < F(x)
        \end{align*}
        Let \(x_0 \geq 0\), then \(\lim_{x\searrow x_0}F(x) = \lim_{x\searrow x_0}\mu(0,x]\). It suffices to show for an arbitrary decreasing sequence \(\set{x_i}_1^\infty\), we have
        \(\lim_{i \to \infty}\mu(0,x_i] = \mu(0,x_0] = F(x_0)\) which follows from continuity from above, so the right sided limit exists and is equal to \(F(x_0)\).

        Now let \(x_0 < 0\), then \(\lim_{x\searrow x_0}F(x) = -\lim_{x\searrow x_0}\mu(x,0]\). It suffices to show for an arbitrary decreasing sequence \(\set{x_i}_1^\infty\), we have
        \(\lim_{i \to \infty}-\mu(x_i,0] = -\mu(x_0,0] = F(x_0)\), which follows from continuity from below, so the right hand sided limit exists and is equal to \(F(x_0)\).
    \end{pb}
    \begin{pb}
        \textbf{(a)} \(F\) bounded. Proof being, assume \(\abs{F} < M\), define sets \(\set{E_i}_1^\infty\), where \(E_i = (-i,-i+1] \cup (i-1,i]\)
        then we have
        \begin{align*}
           m^F(\mathbb{R}) = \sum_1^\infty m^F(E_i) &= \lim_{n\to\infty}\sum_1^n m^F(E_i) \\
            &= \lim_{n\to\infty}\sum_1^n F(i) - F(i-1) + F(-i + 1) - F(-i) \\
            &= \lim_{n\to\infty}F(n) - F(-n) \leq \lim_{n\to\infty}\abs{F(n)} + \abs{F(-n)} \leq 2M
        \end{align*}
        Conversely, assume that \(m^F(\mathbb{R}) = M < \infty\) (note that \(M > 0\)), then for any \(x \geq 0\), we have \(F(x) - F(0) \leq m^F(\mathbb{R}) = M\) by monotonicity, and \(F(0) < F(x)\).
        Similarly, if \(x < 0\), we have \(F(x) \leq F(0)\), and by monotonicity \(F(0) - F(x) \leq m^F(\mathbb{R}) = M\). Taken together for any \(x\) we have
        \begin{align*}
            F(0) - M \leq F(x) \leq M + F(0)
        \end{align*}
        implying \(F\) is bounded. 

        \textbf{(b)} \(F\) continuous at \(x_0\). Proof being, assume \(F\) is continuous at \(x_0\), then for some sequence \(\set{\delta_n}_1^\infty\), we have 
        \(\abs{x - x_0} \leq \delta_n \implies \abs{F(x) - F(x_0)} < \frac{1}{n}\), then continuity from above implies 
        \[0 \leq m^F(\set{x_0}) = m^F\left(\bigcap_1^\infty(\delta_n,x_0]\right) = \lim_{n\to\infty}m^F((\delta_n,x_0]) \leq \lim_{n\to\infty}\frac{1}{n} = 0\]
        Conversely we need only show right continuity. Suppose that \(m^F(\set{x_0}) = 0\), and let \(\epsilon > 0\), then continuity from above implies that
        \[\lim_{n\to\infty}m^F(x_0 - \frac{1}{n},x_0] = m^F\left(\bigcap_1^\infty (x_0 - \frac{1}{n},x_0]\right) = m^F(\set{x_0}) = 0\]
        so in particular, there exists \(N\) sufficiently large that \(m^F(x_0 - \frac{1}{N},x_0] = \abs{F(x_0) - F(x_0 - \frac{1}{N})} < \epsilon\), 
        and since \(F\) is increasing and right continuous this proves left continuity and hence continuity.

        \textbf{(c)} \(m^{F,*}\) is the point mass at \(0\). Proof being, let \(0 \in E \subset \mathbb{R}\), then for any collection of half open intervals \(\set{I_i}_1^\infty\), we have 
        \(0 \in I_n\) for some \(n\). Then we can write \(I_n = (a,b]\), for \(a < 0, \; b \geq 0\), then \(m^F_0(I_n) = 1\), so that \(1 \leq \sum_1^\infty m^{F}_0(I_i)\), and since this holds for all such
        covers of \(E\), we have \(1 \leq m^{F,*}(E)\), and for the reverse inequality note that \(E \subset \mathbb{R} \subset \bigcup_1^\infty (-i, -i+1] \cup (i-1, i+1]\), which is a countable union of
        half open intervals, all but \((-1,0]\) having \(m^F_0(I) = 0\), so \(1 \leq m^{F,*}(E) \leq m^{F,*}(\mathbb{R}) \leq 1\).

        Now suppose that \(0 \not \in E\), then \(E \subset (- \infty, 0) \cup (0,\infty)\), but then \((- \infty, 0) \cup (0,\infty) = \bigcup_1^\infty (-n, 1/n) \cup (0,n)\)
        implies that
        \begin{align*}
            0 \leq m^{F,*}(E) \leq m^{F,*}((- \infty, 0) \cup (0,\infty)) 
            \leq \sum_1^\infty m^F_0(-n, 1/n) + m^F_0(0,n) = 0
        \end{align*}
        Finally note that \(m^{F,*}(\emptyset) = 0\) by definition.

        I claim that \(M_F = \mathcal{P}(\mathbb{R})\), let \(A \subset \mathbb{R}\), and \(E \subset \mathbb{R}\). First assume that \(0 \not \in E\), then
        \[m^{F,*}(E) = 0 = m^{F,*}(E \cap A) + m^{F,*}(E \cap A^c)\]
        since neither of the sets measured on the right hand side of the equation contain 0. Now assume that \(0 \in E\), then \(0 \in E \cap A \tor E \cap A^c\) but not both.
        This suffices to show that
        \[m^{F,*}(E) = 1 = m^{F,*}(E \cap A) + m^{F,*}(E \cap A^c)\]
        So each \(A \in M_F\) is measurable.

        \textbf{(d)} \(m^F\) counts the number of integers in a set \(E\). Proof being, denote the floor function as \(F\).
        First apply theorem 1.16 of Folland (since \(F\) is increasing and right continuous), then \(m^F(a,b] = (F(b) - F(a))\) is a measure, so we may use measure properties.
        If \(z\) is an integer, then we can apply continuity from above:
        \[m^F(\set{z}) = m^F \left(\bigcap_1^\infty (z-\frac1n,z]\right) = \lim_{n\to\infty}m^F(z-\frac1n,z] = \lim_{n\to\infty}1 = 1\]
        So each integer singleton is a measurable set of measure \(1\). Now let \(E \subset \mathbb{R}\setminus\mathbb{Z}\), then \newline
        \(E \subset \bigcup_{n \in \mathbb{Z}}\bigcup_{k=1}^\infty(n-1,n-\frac{1}{k}]\) is a countable union of sets with measure \(0\), hence
        \[0 \leq m^F(E) \leq m^F (\bigcup_{n \in \mathbb{Z}}\bigcup_{k=1}^\infty(n-1,n-\frac{1}{k}]) \leq \sum_{n\in \mathbb{Z}}\sum_{k=1}^\infty m^F(n-1,n-\frac{1}{k}] = 0\]
        Note that singletons are borel sets, then if \(E \subset \mathbb{R}\), we can write \(E \cap \mathbb{Z} = \set{z_i}_i\), then if there are infinitely many \(z_i\):
        \begin{align*}
            \infty = m^F\left(\bigcup_i\set{z_i}\right) \leq m^F(E)
        \end{align*}
        and if there are finitely many \(z_i\):
        \begin{align*}
            m^F(E) = m^F\left(E \cap \mathbb{Z}\right) + m^F\left(E \cap \mathbb{Z}^c\right) = m^F\left(\bigcup_{i=1}^n \set{z_i}\right) + 0 = \sum_1^n m^F(\set{z_i}) = n
        \end{align*}
        We can once again cite Theorem 1.16 of Folland which says this is the unique measure induced by \(F\).
    \end{pb}
    \begin{pb}
        \textbf{(a)} First notice for any \(d\) and any \(\delta > 0\), given \(\set{l_j}_j\) with \(0 \leq l_j \leq \delta\) we have \(l_j^d \geq 0\), and hence
        \(0 \leq \sum_1^\infty l_j^d\). This suffices to show that \(H_d(E) \geq H_{d,\delta}(E) \geq 0\) for any set \(E\).

        Consider any \(d\) and let \(\delta > 0\), (remark we use the convention \(0^0\) = 0), then \(\emptyset \subset (0,0+0), \tand 0 \leq \delta\), so that \(0 \leq H_{d,\delta}(\emptyset) \leq 0^d = 0\)
        since \(\delta\) was arbitrary, this also proves \(H_{d}(\emptyset) = 0\).

        Now consider \(A \subset B \subset \mathbb{R}\). And consider any \(d\) and any \(\delta > 0\). Then for any cover of \(B\) of the form \(B \subset \bigcup_1^\infty (a_j,a_j + l_j)\) (\(0 \leq l_j \leq \delta\)),
        it is immediate \(A \subset \bigcup_1^\infty (a_j,a_j + l_j)\). Hence \(H_{\delta,d}(A) \leq \sum_1^\infty l_j^d\). And since this holds for an arbitrary cover of \(B\) of this form
        \[H_{\delta,d}(A) \leq \inf\set{\sum_1^\infty l_j^d \vert B \subset \bigcup_1^\infty (a_j,a_j + l_j), 0 \leq l_j \leq \delta} = H_{\delta,d}(B)\]
        Now since \(\delta\) was arbitrary, we have for any \(\delta > 0\), \(H_{\delta,d}(A) \leq H_d(B)\), and since this holds for each \(\delta\), \(H_{d}(A) \leq H_d(B)\).

        Now to verify subadditivity, let \(\epsilon > 0\), then let \(d\) be arbitrary and \(\delta > 0\),
        denote \(E := \bigcup_1^\infty E^i \subset \mathbb{R}\). For each \(i\) we can choose \(\set{(a_j^i,a_j^i + l_j^i)}_{j=1}^\infty\) covering \(E^i\), such that \(0 \leq l_j^i \leq \delta\)
        and \(\sum_{j=1}^\infty (l_i^j)^d < H_{d,\delta}(E^i) + 2^{-i}\epsilon\). Then \(E \subset \bigcup_{i,j}(a_j^i, a_j^i + l_j^i)\), so that
        \[H_{\delta,d}(E) \leq \sum_{i,j}(l_j^i)^d < \sum_i H_{\delta,d}(E^i) + 2^{-i}\epsilon = \epsilon + \sum_i H_{\delta,d}(E^i)\]
        and since the \(\epsilon\) was arbitrary, \(H_{\delta,d}(E) \leq \sum_i H_{\delta,d}(E^i)\). This implies that for arbitrary \(\delta > 0\) we have
        \[H_{\delta,d}(E) \leq \sup_{\delta > 0}\sum_i H_{\delta,d}(E^i) \leq \sum_i \sup_{\delta > 0}H_{\delta,d}(E^i) \overset{\text{def}}{=} \sum_i H_d(E^i)\]
        then since \(\delta\) was arbitrary, \(H_{d}(E) \leq \sum_i H_d(E^i)\).

        \textbf{(b)} \(H_0(E)\) is the counting measure. Proof being, first assume \(E\) is a finite collection of points, i.e. \(E = \set{z_i}_1^n \), then for any \(\delta > 0\), 
        we have \(E \subset \bigcup_1^n (z_i - \delta/2, (z_i - \delta/2) + \delta)\), so that \(H_0(E) \leq n\). To show the reverse inequality, let \(\delta = \frac12 \min_{i \neq j}\set{\abs{z_i - z_j}}\).
        Then we need atleast \(n\) intervals of length smaller or equal to \(\delta\) to cover \(E\), since no two points may lie in the same interval. 
        Hence if \(\set{(a_j, a_j + l_j)}_{j \in J}\) is a cover of \(E\), with each \(l_j \leq \delta\), we have \(\sum_{J} l_j^0  \geq \sum_1^n l_j^0 = n\), hence
        \[n \geq H_0 \geq H_{0,\delta} \geq n\]

        Now suppose \(E\)  is infinite, then \(E\) has a countable subset \(\set{z_i}_1^\infty\). I will show that for any natural number \(H_0(E) > N\), and hence \(H_0(E) = \infty\).
        Let \(N \in \mathbb{N}\), then similar to before, let \newline \(\delta_N = \frac12 \min_{i \neq j \tand 1 \leq i,j \leq N}\abs{z_i - z_j}\)
        then none of \(\set{z_i}_1^N\) are in the same interval having length less than or equal to \(\delta_N\), hence if \(\set{z_i}_1^N \subset E \subset \bigcup_{i \in I}(a_i,a_i + l_i)\)
        we must have \(\# I \geq N\), so that \(\sum_{i \in I} l_i^d \geq \sum_1^N l_i^d = N\), which suffices to show \(H_{0,\delta_N}(E) \geq N\). 
        Now since \(H_0 (E)\geq H_{0,\delta_N}(E) \geq N\) for all \(N \in \mathbb{N}\) we get \(H_0(E) = \infty\) as desired. 

        \textbf{(c)} Let \(d > \frac{\log2}{\log3}\), then \(3^d > 3^{\frac{\log2}{\log3}} = 3^{\log_32} = 2\), hence we have \(23^{-d} = r < 1\). With this book-keeping out of the way we can
        move on to solving the problem. Let \(\delta > 0\), then for some \(N \in \mathbb{N}\) we have for all \(n \geq N\), \(3^{-n} < \delta\). Then let \(K_j\) denote the \(j\)-th iteration in the construction of the cantor set, and
        \(\partial_L K_j\) by the left endpoints of the intervals in \(K_j\). Then 
        \[\mathcal{C} \subset K_{n+1} \subset \bigcup_{x \in \partial_L K_{n+1}}(x - 3^{-(n+1)}, x + 2\cdot3^{-(n+1)})\]
        Where \(\# \partial_L K_{n+1} = 2^{n+1}\) since \(K_j\) contains \(j\) intervals, note the subset relation follows by the length of the intervals being \(3^{-(n+1)}\).
        It follows that 
        \[H_{d,\delta}(\mathcal{C}) \leq \sum_1^{2^{n+1}}3^{-nd} = 2^{n+1}3^{-nd} = 2r^n\]
        and since this holds for all \(n\) sufficiently large it holds in the limit, proving that
        \[H_{d,\delta}(\mathcal{C}) \leq \lim_{n\to\infty}2r^n = 0\]
    \end{pb}
    \begin{pb}
        \textbf{(a)} It is immediate that \(\mathcal{C}_{\overrightarrow{\alpha}}\) is closed since it is the intersection of closed sets. Hence it will suffice to show that \(\mathcal{C}_{\overrightarrow{\alpha}}\)
        contains no open sets, and since each open set contains an open interval we can show this for open intervals. Suppose \(I\) is an open interval, then it has some length
        \(\ell > 0\), so choose \(n\) large enough that \(\ell > 2^{-n}\). Let \(\ell_j\) denote the length of the closed intervals making up \(K_j\), then \(\ell_{0} = 1\) and
        \(\ell_{j+1} = \frac12\ell_j(1- \alpha_{j+1})\), so by induction \(\ell_j \leq 2^{-j}\) (this is also the maximum length of an open interval contained in the set),
        hence \(I \not \subset K_n\), and since \(\mathcal{C}_{\overrightarrow{\alpha}}\subset K_n\) this implies that
        \(I \not \subset \mathcal{C}_{\overrightarrow{\alpha}}\). And since \(I\) was arbitrary it follows that \(\mathcal{C}_{\overrightarrow{\alpha}}\) contains no intervals.

        \textbf{(b)} Use the same definition of \(\ell_j\) as in (a). Then notice that \(m(k_j) = 2^j \ell_j\), then
        \[m(k_{j+1}) = 2^{j+1}(\ell_{j+1}) = 2^j \ell_j(1-\alpha_{j+1}) = m(k_j)(1-\alpha_{j+1})\]
        So continuity from above implies that 
        \[m(\mathcal{C}_{\overrightarrow{\alpha}}) = m\left(\bigcap_1^\infty K_n\right) = \lim_{n\to\infty}m(K_n) = \prod_1^\infty(1-\alpha_n)\]
        Now let \(\beta \in [0,1)\), then we can write \(\beta = e^{-x}\) for \(x > 0\), then choose \(0<\alpha_i := 1 - e^{-\frac{x}{2^i}}<1\).
        It follows that \(\log(1-\alpha_i) = -\frac{x}{2^i}\). Taken together
        \[\beta = e^{-x} = \text{exp}(\sum_1^\infty -\frac{x}{2^i}) = \text{exp}(\sum_1^\infty \log(1-\alpha_i)) = \prod_1^\infty(1-\alpha_i)\]
        As desired.

        \textbf{(c)} Assume for contraposition that \(E \subset [0,1]\), with \(m(E) = 1\). Then \(\overline{E} \subset [0,1]\) implies that \(1 \geq m(\overline{E}) \geq m(E) \geq 1\).
        Then the compliment of \(\overline{E}\) in \([0,1]\) is open with measure \(0\), hence empty (since any non-empty open set contains an interval and intervals have non-zero measure).
        It follows that \(\overline{E} = [0,1]\) contains the interval \((0,1)\), so \(E\) is not nowhere dense. Contraposition completes the proof.

        \textbf{(d)} Let \(\epsilon > 0\) but as close to \(0\) as the reader might like, then take \(\set{q_i}_1^\infty\) be an enumeration of
        \(\mathbb{Q} \cap [0,1]\). Then take open sets \(I_i := N_{\epsilon 2^{-n-1}}(q_i)\), it follows that \(I = \bigcup_1^\infty I_i\) is an open set containing all
        of the rationals in \([0,1]\), hence \(I^c\) is closed containing none of the rationals and hence no interval. Then
        \[1 = m([0,1]) \geq m(I^c \cap [0,1]) \geq 1-m(I) \geq 1 - \sum_1^\infty m(I_i) = 1 - \sum_1^\infty 2^{-n}\epsilon = 1 - \epsilon\]
    \end{pb}
    \begin{pb}
        \textbf{(a)} \(E \subset [0,1]\) implies that \(m^*(E) \leq 1\) by monotonicity. \(m^*(E) > 0\), since any set of outer measure \(0\) is measurable, by the Caratheodory
        theorem which says that the induced measure by \(m^*(E)\) is complete on its measurable sets.

        \textbf{(b)} Suppose for contradiction \(m(A) > 0\). Then by the Steinhaus theorem \(A - A\) contains an interval, but the only rational member of \(A - A\)
        is zero, since either elements are equal or in different rational translate cosets. This proves that \(A - A\) cannot contain an interval
        (since any interval contains countably many rationals), a contradiction.

        \textbf{(c)} Consider any countable collection of intervals \(\set{I_i}_1^\infty\), such that \([0,1]\setminus E \subset \bigcup_1^\infty I_i\), then
        \([0,1]\cap(\bigcup_1^\infty I_i)^c \subset E\), and is a borel set hence measurable, so it must have measure (by caratheodory theorem this is equal to outer measure) \(0\) by the previous subpart.
        \begin{align*}
            m^*\left([0,1]\cap(\bigcup_1^\infty I_i)^c\right) + m^*\left([0,1]\cap(\bigcup_1^\infty I_i)\right) &= m\left([0,1]\cap(\bigcup_1^\infty I_i)^c\right) + m\left([0,1]\cap(\bigcup_1^\infty I_i)\right) \\
            &= m[0,1] = 1
        \end{align*}
        Implying that \(m^*\left(\bigcup_1^\infty I_i\right) \geq m^*\left([0,1]\cap(\bigcup_1^\infty I_i)\right) = 1\), hence by definition \(m^*([0,1] \setminus E) \geq 1\). To show equality, let \(\epsilon > 0\), then
        \([0,1] \setminus E \subset (-\frac{\epsilon}{2},1+\frac{\epsilon}{2})\) implies that \(m^*([0,1] \setminus E) \leq 1 + \epsilon\), and since \(\epsilon\) was arbitrary,
        \[1 \leq m^*([0,1] \setminus E) \leq 1\]
        as desired.
    \end{pb}
\end{document}