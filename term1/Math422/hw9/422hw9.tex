\documentclass[11pt]{article}
\usepackage{amsmath, amsfonts, amssymb,amsthm}
\usepackage[includeheadfoot]{geometry} % For page dimensions
\usepackage{fancyhdr}
\usepackage{enumerate} % For custom lists

\fancyhf{}
\lhead{Math 422hw9}
\rhead{Tighe McAsey - 37499480}
\pagestyle{fancy}

% Page dimensions
\geometry{a4paper, margin=1in}

\theoremstyle{definition}
\newtheorem{pb}{}

% Commands:

\newcommand{\set}[1]{\{#1\}}
\newcommand{\abs}[1]{\lvert#1\rvert}
\newcommand{\norm}[1]{\lvert\lvert#1\rvert\rvert}
\newcommand{\gen}[1]{\left\langle #1 \right\rangle}
\newcommand{\tand}{\text{ and }}
\newcommand{\tor}{\text{ or }}
\newcommand{\falg}{F^{\text{alg}}}
\newcommand{\gal}{\text{Gal}}
\newcommand{\floor}[1]{\left\lfloor #1 \right\rfloor}

\begin{document}
    \begin{pb}
        Denote \(f := X^3 -aX +a\), by Gauss' lemma, we have that \(f\) is irreducible in \(\mathbb{Q}[X]\) if and only if it is reducible in \(\mathbb{Z}[X]\). Note that since \(f\) is monic, if it can be written as a product of polynomials \(g \tand h\) of smaller degrees, then taking \(\overline{f_p} := f \mod{p}\), we have that \(\overline{f_p} = \overline{g}\overline{h}\) is reducible. Thus it suffices to show \(f\) is irreducible when taken in \(\mathbf{F}_2\). Since the parity of \(k^2\) is equal to the parity of \(k\), \(a \equiv 7 \mod{2}\) is odd. It follows that \(\overline{f_2} = X^3 - X + 1\). Since \(\overline{f_2}\) is a cubic, it is reducible if and only if it finds a root, but \(\overline{f_2}(0) = 1\), and \(\overline{f_2}(1) = 1\) so that \(\overline{f_2}\) is irreducible, implying that \(f\) is irreducible.
        
        To examine the Galois group, we simply take the discriminant of \(f\),
        \begin{align*}
            D(f) &= -(4(-a)^3 + 27a^2) = -a^2(27 - 4a) = -a^2(27-4(k^2+k+7)) \\
            &= -a^2(-1-4k^2-4k) = a^2(4k^2 + 4k + 1) = a^2(2k+1)^2 \in \mathbb{Z}^2
        \end{align*}
        Since \(f\) is irreducible with square discriminant its Galois group is \(A_3\).
    \end{pb}
    \begin{pb}
        To see that \(f := T^3 - T - 1\) is irreducible , we apply Gauss' lemma, so that it suffices to check for irreducibility in \(\mathbb{Z}\). Furthermore, by the same argument as in problem one, to show that \(f\) is irreducible we may show that \(\overline{f_3}\) is irreducible. \(\overline{f_3}\) is irreducible over \(\mathbf{F_3}\) since it is an Artin-Schreier polynomial. Alternatively; every element of \(\mathbf{F_3}\) satisfies \(x^3 - x = 0\), so that \(\overline{f_3}(a) = -1\) for any \(a \in \mathbf{F_3}\), and since \(\overline{f_3}\) is a cubic finding no roots, it is irreducible. Thus implying irreducibility of \(f\). Now to describe the Galois group, it will suffice to determine the discriminant of \(f\),
        \begin{align*}
            D(f) = -(4(-1)^3 + 27(-1)^2) = -23 \not \in \mathbb{\mathbb{Q}}^2
        \end{align*}
        And hence \(\gal_f = S_3\)
    \end{pb}
    \begin{pb}
        We can identify \(f\) as the 5-th cyclotomic polynomial \(\Phi_5(X)\), we have shownn that the p-th cyclotomic polynomial is irreducible for \(p\) prime. Furthermore, we have shown that
        \begin{align*}
            \gal_{\Phi_n} \simeq \left(\mathbb{Z}/(n)\right)^\times \implies \gal_{\Phi_5} \simeq \left(\mathbb{Z}/(5)\right)^\times
        \end{align*}
        Which is cyclic and generated by \(2\), hence
        \begin{align*}
            \gal_{\Phi_5} \simeq \left(\mathbb{Z}/(5)\right)^\times \simeq C_4
        \end{align*}
    \end{pb}
    \begin{pb}
        To show that \(x \not \in \varphi^{-1}(F)\), assume for the sake of contradiction that \(\varphi(\frac{f}{g}) = x\) where \((f,g) = 1\), then
        \begin{align*}
            &x = \varphi(\frac{f}{g}) = \frac{f^p}{g^p} - \frac{f}{g} = \frac{f^p - fg^{p-1}}{g^p} \\
            \implies &f^p = fg^{p-1} + xg^p 
        \end{align*}
        So that \(g \vert f^p\), it follows by Euclid's lemma for Euclidean domains (in this case polynomial rings) that any irreducible factor \(q\) of \(g\) is such that \(q \vert f\). This contradicts \((f,g) = 1\).

        Now denote \(t := \varphi^{-1}(x)\), \(t\) satisfies the polynomial \(c(T) := T^p - T - x \in F[T]\), to show that \([F(t):F] = p\), it will suffice to show that \(T^p - T - x\) is irreducible which implies that \(c(T) = \min(t;F)\). First note that \(\min(t;F) \vert c(T)\), where \(c(T)\) has roots \(t + a\), for each \(a \in \mathbf{F_p}\). To see this, let \(a \in \mathbf{F_p}\), then
        \begin{align*}
            c(T + a) = T^p + a^p - T - a - x = T^p + a - T - a - x = T^p - t - x = c(T)
        \end{align*}
        so that \(c(T)\) is seperable, and thus so is \(\min(t;F)\), implying that \(F(t)\) is a seperable extension. Now to show that \(c(T)\) is indeed irreducible, write
        \begin{align*}
            c(T) = h_1(T),\hdots,h_k(T)
        \end{align*}
        as its irreducible factor decomposition. It follows, by the uniqueness of this decomposition, that the map \(T \overset{\sigma}{\mapsto} T+1\) acts by permuting these factors. It may only fix a factor if for each monic factor \((t + a) \vert h_i\), we also have \((t + a + 1) \vert h_i\), but continuing this inductively, the action only fixes a factor \(h_i\) if \((t + a) \vert h_i, \; \forall a \in \mathbf{F_p}\), implying that \(h_i(T) = c(T)\). Hence we have that \(\sigma\) acts on the \(k\) factors of \(c(T)\) via permutation for \(k < p\) (we get this inequality since \(c(T)\) cannot split linearly since \(t \not \in F\)). Then we have that \(\sigma \in S_k\), and \(\sigma^p = 1\), by Lagranges theorem we have \(o(\sigma) \vert k!\), so that in particular \(o(\sigma) \vert (p,k!) = 1\) since \(p\) is a prime greater than \(k\). This implies that \(\sigma\) acts via the identity so in particular \(\sigma\) fixes \(h_1\), which as described earler implies that \(h_1(T) = c(T)\). This suffices to show that \(c(t)\) is irreducible, since \(h_1(T) = c(T)\) was taken to be irreducible.

        Now we have shown that \(F(t)\) is a seperable of extension of degree \(p\), which is normal since it is the splitting field of \(c(T) = \prod_{j=0}^{p-1}(T-(t+j))\) over \(F\). This implies that the extension is Galois, and \(\# \gal(F(t)/F) = [F(t):F] = p\), so since this is a group of order \(p\), it is cyclic.
    \end{pb}
    \begin{pb}
        We first show that \(\min(a^{1/p}:\mathbb{Q}) = X^p - a\), proof being we can factor \(X^p - a = \prod_{k=1}^p (X - a^{1/p}\zeta_p^k)\) in \(\mathbb{C}[X]\). If this
        polynomial were reducible in \(\mathbb{Q}\), then if \(g\) were a factor, the last coefficient of \(g\) must be of the form \(\pm a^{k/p}\). This is impossible since
        \(a^{k/p} \in \mathbb{Q}, \; k < p\), then by bezouts identity, there exist \(u,v\) such that \(uk + vp = 1\), implying that \(a^{1/p} = a^{uk/p}a^{vp/p} \in \mathbb{Q}\).

        This gives the desired result for both \(L\) and \(F\) extensions, since by multiplicativity of degree,
        \begin{align*}
            &p \vert [F(a^{1/p}):\mathbb{Q}] \tand [F: \mathbb{Q}] \leq p-1 \\
            &p \vert [L(a^{1/p}):\mathbb{Q}] \tand [L: \mathbb{Q}] \leq p-1
        \end{align*}
        implying that \(p \vert [F(a^{1/p}):F], [L(a^{1/p}):L]\). Then since \(F \supset \cos(2\pi/p), -\sin^2(2\pi/p)\) (proven below), we get that \(L = F(\sqrt{-\sin^2(2\pi/p)})\), i.e. \([L:F] = 2\), also note that
        \(N = L(a^{1/p})\). So that \([L(a^{1/p}):F] = [L(a^{1/p}):L][L:F] = 2p\), implies \(\# \gal(N/F) = 2p\). Finally, we have that
        \(\gal(N/F) \supset \gen{\tau,\sigma} \simeq D_p\), where \(\tau\) is complex conjugation and \(\sigma\) is a generator of the cyclic group \(\gal(\mathbb{Q}(\zeta_p):\mathbb{Q})\). The
        isomorphism follows from \(\sigma\tau = \tau\sigma^{-1}, \tau^2 = 1, \sigma^p = 1\), meaning the multiplication rules of \(D_p\) are satisfied. Then since 
        \(\# \gal(N/F) = 2p = \# \gen{\tau,\sigma}\) we have equality.

        To show that \(F \supset \cos(2\pi/p), \sin^2(2\pi/p)\), we have \(\zeta_p, \zeta_p^{-1} \in F\), hence we have \(\frac12(\zeta_p + \zeta_p^{-1}) = \cos(2\pi/p) \in F\). This implies we also have
        \((\zeta_p - \cos(2\pi/p))^2 = -\sin^2(2\pi/p)\).
    \end{pb}
    \begin{pb}
        Since \([K:F] = n\), and \(x^n - a\) is a degree \(n\) polynomial satisfied by \(\alpha\), we have that \(\min(\alpha;F) = x^n - a\). Note that for each \(0 \leq i < n\), we have that \(a\zeta_n^i\) is also a root of \(x^n - a\), so that in particular \(K\) is the splitting field of \(x^n - a\), a seperable polynomial so that \(K/F\) is Galois with Galois group \(G\) permuting the \(n\) factors of \(\min(\alpha;F)\).

        We have that 
        \begin{align*}
            Tr_{K/F}(\alpha) = \sum_{\sigma \in G} \sigma(\alpha) = \sum_{i=0}^{n-1} \zeta_n^i \alpha
        \end{align*}
        is the first elementary symmetric polynomial in the factors of \(\min(\alpha;F)\), in particular it is equal to \(-1 c_{n-1}\), where \(c_{n-1}\) is the coefficient of \(x^{n-1}\) in \(\min(\alpha;F)\) which is equal to \(0\). And hence \(Tr_{K/F}(\alpha) = -c_{n-1} = 0\)

        Similarly, we have that
        \begin{align*}
            N_{K/F}(\alpha) = \prod_{\sigma \in G}\sigma(\alpha) = \prod_{i=0}^{n-1}\zeta_n^i \alpha
        \end{align*}
        Which is the \(n\)-th elementary symmetric polynomial in the factors of \(\min(\alpha;F)\), in particular it is equal to \((-1)^nc_0\), where \(c_0\) is the constant term in \(\min(\alpha;F)\), in this case we have that \(c_0 = -a\), implying that \(N_{K/F}(\alpha) = (-1)^n c_n = (-1)^{n+1}a\)
    \end{pb}
\end{document}